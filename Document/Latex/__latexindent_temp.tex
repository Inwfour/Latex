\chapter{การพัฒนาระบบ}
ในบทนี้จะกล่าวถึงการสร้างแอปพลิเคชันสูงวัยมายเฟรนด์ โดยนำผลที่ได้จากการวิเคราะห์และออกแบบระบบมาสร้างเป็นระบบงานซึ่งจะอธิบายถึงตัวอย่างการเขียน โปรแกรมการทำงานของระบบในส่วนต่างๆดังต่อไปนี้

% start สมัครสมาชิก
\section{การพัฒนาในส่วนสมัครสมาชิก}
เมื่อผู้ใช้กรอกข้อมูลการสมัครเสร็จแล้วทำการกดปุ่มสมัครสมาชิก ระบบจะมีการทำงาน แสดงดังรูปที่ \ref{Fig:4-Register}
\begin{figure}[H]
{\setstretch{1.0}\lstset{language=Pascal}
\begin{lstlisting}
save(user) {
	if (this.user.email == null || this.user.password == null) {
		this.toastCtrl.create({
		message: "กรุณาระบุชื่อผู้ใช้หรือรหัสผ่านให้ถูกต้อง",
		duration: 3000,
		position: top,
	}).present();
	} else {
		this.registerService.SaveUser(user).then(async (data) => {
		await firebase.auth().signOut();
		this.user.email = "";
		this.user.password = "";
		this.toastCtrl.create({
			message: "บันทึกข้อมูลสำเร็จ",
			duration: 3000,
			position: top,
		}).present();
		}).catch(() => {
		this.toastCtrl.create({
			message: "ชื่อผู้ใช้หรือรหัสผ่านไม่ถูกต้อง",
			duration: 3000,
			position: 'top'
		}).present();
		})
	}
}
\end{lstlisting}}
\caption{การทำงานของระบบเมื่อกดปุ่มสมัครสมาชิก}
\label{Fig:4-Register}
\end{figure}
\newpage
จากภาพที่ \ref{Fig:4-Register} โครงสร้างของไฟล์ register.ts สามารถอธิบายการทำงานได้ดังนี้
\begin{itemize}[label={--}]
\item บรรทัดที่ 1 ฟังก์ชัน save() เป็นฟังก์ชันที่ใช้ควบคุมการทำงานในการสมัครสมาชิก
\item บรรทัดที่ 2 - 7 เช็คอีเมลล์และพาสเวิร์ดที่เรากรอกว่าว่างหรือไม่ ถ้าช่องใดช่องหนึ่งจะแสดงข้อความว่า "กรุณาระบุชื่อผู้ใช้หรือรหัสผ่านให้ถูกต้อง" ถ้าไม่ว่างจะนำไปเช็คข้อมูลต่อไป
\item บรรทัดที่ 9 เป็นการเรียกใช้งานฟังก์ชัน saveUser() ในคลาส RegisterProvider เพื่อบันทึกข้อมูลผู้ใช้ลง Authentication Firebase 
\item บรรทัดที่ 10 กำหนดให้ user logout เนื่องจากหลังจาก register เรียบร้อยแล้วฐานข้อมูลจะถูกเก็บ user โดยอัตโนมัติ จึงต้อง logout ออกให้อยู่ในสถานะไม่มี user
\item บรรทัดที่ 11 - 12 กำหนดค่าของช่องอีเมล์และพาสเวิร์ดให้ว่างหลังจากบันทึกข้อมูลเรียบร้อยแล้ว
\item บรรทัดที่ 13 - 17 ถ้าระบบบันทึกข้อมูลลงฐานข้อมูลเรียบร้อยแล้ว จะแสดงข้อความว่า "บันทึกข้อมูลสำเร็จ"
\item บรรทัดที่ 18 - 23 ถูกเรียกใช้กรณีสมัครสมาชิกไม่สำเร็จจะแสดงข้อความว่า "ชื่อผู้ใช้หรือรหัสผ่านไม่ถูกต้อง"
\end{itemize}
\newpage
% end สมัครสมาชิก

% start เข้าสู่ระบบ
\section{การพัฒนาในส่วนการเข้าสู่ระบบ}
เมื่อผู้ใช้กรอกข้อมูลทั้งหมดเสร็จทำการกดปุ่มเข้าสู่ระบบ ระบบจะมีการทำงาน แสดงดังรูปที่ \ref{Fig:4-login}

\begin{figure}[H]
{\setstretch{1.0}\lstset{language=Pascal}
\begin{lstlisting}
login(user) {
let loader = this.loadingCtrl.create({
	spinner: 'hide',
	content: `<img src="assets/imgs/loading.svg">`
})
loader.present();
this.loginservice.loginUser(user).then(async(data) => {
	await this.fireinfo.doc(firebase.auth().currentUser.uid).get().then((res) => {
	if (res.data() === undefined) {
		this.fireinfo.doc(firebase.auth().currentUser.uid).set({
		owner: firebase.auth().currentUser.uid,
		email: firebase.auth().currentUser.email,
		created: firebase.firestore.FieldValue.serverTimestamp(),
		}).then(() => {
		loader.dismiss();
			this.toastCtrl.create({
			message: "เข้าสู่ระบบสำเร็จ",
			duration: 3000,
			position: 'top',
			}).present();
			this.navCtrl.push(RegisterPhotoPage);
		})
		.catch(() => {
			loader.dismiss();
			this.toastCtrl.create({
			message: "บันทึกข้อมูลไม่สำเร็จ",
			duration: 3000,
			position: 'top'
			}).present();
		});
\end{lstlisting}}
\caption{การทำงานของระบบเมื่อกดเข้าสู่ระบบ}
\label{Fig:4-login}
\end{figure}
\newpage

จากภาพที่ \ref{Fig:4-login} โครงสร้างของไฟล์ login.ts สามารถอธิบายการทำงานได้ดังนี้
\begin{itemize}[label={--}]
\item บรรทัดที่ 2 - 6 เป็นการใช้ loadingController เพื่อเริ่มใช้โหลดหน้าจอเมื่อฟังก์ชันถูกกดใช้
\item บรรทัดที่ 7 เป็นการเรียกใช้ฟังก์ชัน loginUser(user) ที่อยู่ในคลาส LoginService เพื่อตรวจสอบการเข้าสู่ระบบว่ามีอยู่ในฐานข้อมูลหรือไม่
\item บรรทัดที่ 8 - 22 เป็นการเช็คข้อมูลในฐานข้อมูลว่ามีข้อมูลหรือไม่ ถ้าไม่มีจะสร้างรายละเอียดข้อมูลผู้ใช้ไว้ในฐานข้อมูล ถ้ามีจะไม่ทำในส่วนนี้
\item บรรทัดที่ 23 - 30 กรณีไม่สามารถทำการสร้างข้อมูลได้ระบบจะแจ้งว่าบันทึกข้อมูลไม่สำเร็จ 
\end{itemize}
\newpage

% end เข้าสู่ระบบ

% start เข้าสู่ระบบต่อ

\begin{figure}[H]
	{\setstretch{1.0}\lstset{language=Pascal}
	\begin{lstlisting}
} else {
	if(res.data().owner_name == undefined && res.data().age == undefined && res.data().phone == undefined && res.data().photoURL == undefined && res.data().disease == undefined){
	console.log(data);
	loader.dismiss();
	this.toastCtrl.create({
		message: "เข้าสู่ระบบสำเร็จ",
		duration: 3000,
		position: 'top',
	}).present();
	this.navCtrl.push(RegisterPhotoPage);
	}else {
		loader.dismiss();
		this.toastCtrl.create({
		message: "เข้าสู่ระบบสำเร็จ",
		duration: 3000,
		position: 'top',
		}).present();
		this.navCtrl.push(TabsPage)
	}
	}
})
}).catch((err) => {
console.log(err);
loader.dismiss();
this.toastCtrl.create({
message: "กรุณากรอกชื่อผู้ใช้หรือรหัสผ่านให้ถูกต้อง",
duration: 3000,
position: 'top',
}).present();
})
}
	\end{lstlisting}}
	\caption{การทำงานของระบบเมื่อกดเข้าสู่ระบบ (ต่อ)}
	\label{Fig:4-logincon1}
	\end{figure}
\newpage

จากภาพที่ \ref{Fig:4-logincon1} โครงสร้างของไฟล์ login.ts (ต่อ) สามารถอธิบายการทำงานได้ดังนี้
\begin{itemize}[label={--}]
\item บรรทัดที่ 2 - 9 ทำการเช็คข้อมูลที่อยู่ในฐานข้อมูล ถ้าหากข้อมูลบางอย่างไม่ถูกสร้าง ระบบจะหยุด LoadingController และจะแสดงข้อความเข้าสู่ระบบสำเร็จ หลังจากนั้นจะเรียกใช้งานคลาส RegisterPhotoPage
\item บรรทัดที่ 11 - 18 ถ้าหากข้อมูลถูกสร้างทั้งหมดแล้ว ระบบหยุด LoadingController และจะแสดงข้อความเข้าสู่ระบบสำเร็จ จากนั้นจะเรียกใช้คลาส TabsPage เพื่อไปยังหน้าหลัก
\item บรรทัดที่ 22 - 29 ถ้าหากผู้ใช้กรอกข้อมูลไม่ครบหรือไม่ถูกต้องระบบจะแสดงข้อความว่ากรุณากรอกชื่อผู้ใช้หรือรหัสผ่านให้ถูกต้อง
\end{itemize}
\newpage

% end เข้าสู่ระบบต่อ

% start คุยกับแชทบอท

\section{การพัฒนาในส่วนของคุยกับแชทบอท}
เมื่อผู้ใช้กดเลือกเมนูปู่จอห์น ระบบจะมีการทำงาน แสดงดังรูปที่ \ref{Fig:4-addchatbot}
\begin{enumerate}
	\item ติดตั้ง api.ai plugin กับ cordova cli
	\begin{figure}[H]
		{\setstretch{1.0}\begin{lstlisting}
$  cordova plugin add cordova-plugin-apiai
		\end{lstlisting}}
		\caption{คำสั่งเพิ่ม api.ai plugin}
		\label{Fig:4-addchatbot}
	\end{figure}
	จากรูปที่ \ref{Fig:4-addchatbot} การเรียกใช้งานต้องทำการติดตั้ง api.ai plugin ก่อนจึงจะเรียกใช้ได้
	\newline

	\item app.component.ts
	\begin{figure}[H]
		{\setstretch{1.0}\begin{lstlisting}
window["ApiAIPlugin"].init(
{
	clientAccessToken: "xxxxxxxxxxxxxxxxxxxxxxx",
	lang: "en"
}, 
);
		\end{lstlisting}}
		\caption{โค้ดส่วนของการเพิ่ม clientAccessToken ของแชทบอท}
		\label{Fig:4-configchatbot}
	\end{figure}
	จากรูปที่ \ref{Fig:4-configchatbot} ในการเชื่อมต่อเพื่อพูดคุยกับ Diagloflow ต้องกำหนด clientAccessToken ก่อนเพื่อเชื่อมต่อกัน
	\newpage

	\item chatbot.ts
	\begin{figure}[H]
		{\setstretch{1.0}\begin{lstlisting}
sendText() {
window["ApiAIPlugin"].requestText({
query: messages
}, (response) => {
this.ngZone.run(() => {

if(response.result.fulfillment.speech){
try {
	console.log(response);
	this.textres = JSON.parse(response.result.fulfillment.speech);
	this.messages.push({
	text: this.textres.text,
	img: this.textres.img,
	button: this.textres.button,
	list: this.textres.list,
	sender: "api"
	})
} catch(e) {
	this.messages.push({
	text: response.result.fulfillment.speech,
	sender: "api"
	})
}
this.content.scrollToBottom();
}  
})
}, (error) => {
alert(JSON.stringify(error))
})
}
		\end{lstlisting}}
		\caption{การทำงานของการคุยกับแชทบอท}
		\label{Fig:4-chatbot}
	\end{figure}
	\newpage

	จากภาพที่ \ref{Fig:4-chatbot} โครงสร้างของไฟล์ chatbot.ts (ต่อ) สามารถอธิบายการทำงานได้ดังนี้
	\begin{itemize}[label={--}]
	\item บรรทัดที่ 2 - 3 เป็นการส่งตัวแปร messages ไป query ที่ Dialogflow เพื่อหาคำตอบ
	\item บรรทัดที่ 4 - 17 คือคำตอบที่ถูก response กลับมาจาก Dialogflow จากนั้นนำ response.result.fulfillment.speech เก็บไว้ในตัวแปร textres แล้วส่งไปแสดงหน้าจอในรูปแบบข้อความหรือรูปภาพหรือลิสท์
	\item บรรทัดที่ 18 - 22 ถ้าข้อมูลที่ถูกส่งกลับมามีข้อผิดพลาด push ข้อความมาแสดงหน้าจอทันที
	\end{itemize}
	\newpage

	\begin{figure}[H]
		{\setstretch{1.0}\begin{lstlisting}
startListening() {
let options = {
	language: 'th-TH'
}
this.speechRecognition.startListening(options).subscribe((matches) => {
	this.text = matches[0];
	this.sendText();
}),
}
		\end{lstlisting}}
		\caption{การทำงานของการใช้งานพิมพ์ด้วยเสียง}
		\label{Fig:4-chatbotsound}
	\end{figure}
	\newline
	จากภาพที่ \ref{Fig:4-chatbotsound} แสดงการทำงานการพิมพ์ด้วยเสียง สามารถอธิบายการทำงานได้ดังนี้
	\begin{itemize}[label={--}]
	\item บรรทัดที่ 1 เป็นการเรียกใช้ฟังก์ชันท์ startListening()
	\item บรรทัดที่ 2 - 4 เป็นการกำหนด options ของเสียงที่ถูกรับเข้ามา
	\item บรรทัดที่ 5 - 7 จะเรียกใช้ speechRecognition plugin เพื่อใช้การทำงานของการรับเสียบแปลงเป็นข้อความ จากนั้นนำ
	matches[0] เก็บไว้ในตัวแปร text แล้วจึงเรียกใช้ฟังก์ชัน sendText() เพื่อส่งข้อความรูปแบบ text
	\end{itemize}
\end{enumerate}
\newpage
% end คุยกับแชทบอท

% start ส่วนเพิ่มโพสท์

\section{การพัฒนาในส่วนของการโพสท์}
เมื่อผู้ใช้กรอกข้อความหรือเพิ่มรูปภาพแล้วกดปุ่มโพสท์ ระบบจะมีการทำงาน แสดงดังรูปที่ \ref{Fig:4-addpost}

\begin{figure}[H]
{\setstretch{1.0}\lstset{language=Pascal}
\begin{lstlisting}

\end{lstlisting}}
\caption{การพัฒนาในส่วนของการโพสท์}
\label{Fig:4-addpost}
\end{figure}
\newpage

จากภาพที่ \ref{Fig:4-addpost} โครงสร้างของไฟล์ feed.ts สามารถอธิบายการทำงานได้ดังนี้
\begin{itemize}[label={--}]
\item บรรทัดที่ 2 - 6 เป็นการใช้ loadingController เพื่อเริ่มใช้โหลดหน้าจอเมื่อฟังก์ชันถูกกดใช้
\item บรรทัดที่ 7 เป็นการเรียกใช้ฟังก์ชัน loginUser(user) ที่อยู่ในคลาส LoginService เพื่อตรวจสอบการเข้าสู่ระบบว่ามีอยู่ในฐานข้อมูลหรือไม่
\item บรรทัดที่ 8 - 22 เป็นการเช็คข้อมูลในฐานข้อมูลว่ามีข้อมูลหรือไม่ ถ้าไม่มีจะสร้างรายละเอียดข้อมูลผู้ใช้ไว้ในฐานข้อมูล ถ้ามีจะไม่ทำในส่วนนี้
\item บรรทัดที่ 23 - 30 กรณีไม่สามารถทำการสร้างข้อมูลได้ระบบจะแจ้งว่าบันทึกข้อมูลไม่สำเร็จ 
\end{itemize}
\newpage

% end ส่วนเพิ่มโพสท์

% start ส่วนลบโพสท์
















\begin{figure}[H]
{\setstretch{1.0}\begin{lstlisting}
.subscribe((_res: any) => {
_res.result.hits.map(_travel => {
let total_place = 0; let avg = 0;
this.http.get(`${KUZZLE}:${KPORT}/${KINDEX}/${travels.review}/${_travel._id}`)
.subscribe((res: any) => {
res.result._source.review.map(_item => {
total_place += _item.score
})
avg = total_place / res.result._source.review.length
this.results_populartravel.push({
..._travel, avg,
review: travels.review
})
this.results_populartravel.sort(
(a, b) => (a.avg > b.avg) ? 1 : ((b.avg > a.avg) ? -1 : 0)
).reverse()
}, err => {
this.results_populartravel.push({
..._travel, avg: 0,
review: travels.review
})
this.results_populartravel.sort(
(a, b) => (a.avg > b.avg) ? 1 : ((b.avg > a.avg) ? -1 : 0)
).reverse()
})
})
})
})
break;
\end{lstlisting}}
\caption{การทำงานของระบบเมื่อผู้ใช้กดเลือกหมวด (ต่อ)}
\label{Fig:4-pupular1}
\end{figure}
\newpage

จากภาพที่ \ref{Fig:4-pupular1} โครงสร้างของไฟล์ popular.component.ts สามารถอธิบายการทำงานได้ดังนี้(ต่อ)
\begin{itemize}[label={--}]
\item บรรทัดที่ 1 เป็นการ subscribe เพื่อดูข้อมูลข้างในที่ได้จากการ query ข้อมูล
\item บรรทัดที่ 2 เป็นการจิ้มข้อมูลเข้าไปชั้นข้างในเพื่อให้ได้ข้อมูลที่ต้องการและทำการวนฟอดูข้อมูลข้างใน
\item บรรทัดที่ 3 เป็นการประกาศตัวแปรtotalPlace ใช้เก็บค่าผลรวม ส่วน avg ใช้เก็บค่าเฉลี่ยของการให้คะแนนสถานที่
\item บรรทัดที่ 4 - 6 เป็นการ ค้นหาข้อมูลจากฐานข้อมูล เมื่อได้ข้อมูลมาจะใช้ .subscribe ในการเข้าดูข้อมูลโดยจะประกาศ res ขึ้นมาเก็บค่าข้อมูลที่ได้ทำการค้นหาและทำการจิ้มข้อมูลใน res เข้าไปถึงส่วนของข้อมูลที่เราต้องการ
\item บรรทัดที่ 7 เป็นการนำค่าที่ได้ในการวนฟอในแต่ละรอบมาเก็บไว้ที่ totalplace
\item บรรทัดที่ 9 เป็นการคำนวณคะแนนเฉลี่ยในการรีวิวของแต่ละสถานที่และค่าที่จะนำมาเก็บไว้ในตัวแปร avg
\item บรรทัดที่ 10 - 13 เป็นการ push ค่าที่ได้ไปเก็บไว้ที่ resultspopulartravel 
\item บรรทัดที่ 14 - 16 เป็นการนำค่าที่ถูกเก็บไว้มาทำการจัดอันดับโดยจะเช็คว่า ค่าเฉลี่ยของตัวที่ 1 มีค่ามากกว่าตัวที่ 2 ถ้าใช่ก็ให้มีค่าเป็น 1 ถ้าค่าเฉลี่ยตัวที่ 2 มากกว่าตัวที่ 1 ก็ให้มีค่าเป็น -1 และทำการ reverse() ข้อมูลโดยจะเช็คไปแบบนี้จนครบถึงจะหยุดทำงาน
\item บรรทัดที่ 17 - 21 ถ้าข้อมูลสถานที่นั้นยังไม่มีรีวิว ก็จะทำการ push ข้อมูลสถานที่โดยจะกำหนดให้ค่าของ avg = 0
\item บรรทัดที่ 22 - 24 เป็นการนำค่าที่ถูกเก็บไว้มาทำการจัดอันดับโดยจะเช็คว่า ค่าเฉลี่ยของตัวที่ 1 มีค่ามากกว่าตัวที่ 2 ถ้าใช่ก็ให้มีค่าเป็น 1 ถ้าค่าเฉลี่ยตัวที่ 2 มากกว่าตัวที่ 1 ก็ให้มีค่าเป็น -1 และทำการ reverse() ข้อมูลโดยจะเช็คไปแบบนี้จนครบถึงจะหยุดทำงาน
\end{itemize}
\newpage

\section{การพัฒนาในส่วนของเขียนรีวิว}
เมื่อผู้ใช้กดปุ่มบันทึกรีวิว ระบบจะมีการทำงาน แสดงดังรูปที่ \ref{Fig:4-review} - \ref{Fig:4-review2}
\begin{figure}[H]
{\setstretch{1.0}\begin{lstlisting}
onClick(rating: number): void {
this.rating = rating;
}
reviewPlace() {
const now = moment(moment().valueOf()).format("DD/MM/YYYY H:mm")
if (this.id_review === undefined) {
this.dataReview = {
review: [
{
id: this.id_place,
create_by: JSON.parse(localStorage.getItem("UserInfo")).username,
imageUser: JSON.parse(localStorage.getItem("UserInfo")).imageSrc,
title: this.title,
description: this.description,
image: this.imageReview,
score: this.rating,
timeReview: now
}
]
}
this.http.post(`${KUZZLE}:${KPORT}/${KINDEX}/${this.idReview}/${this.id_place}/_create`, this.dataReview)
.subscribe((response) => {
console.log(response);
});
this.reviews = [...this.reviews, {
id: this.id_place,
type: this.type_place,
review: this.idReview
 }]
 this.http.put(`${KUZZLE}:${KPORT}/users/${this.id_user}/_update`, {
reviews: JSON.stringify(this.reviews)
 })
\end{lstlisting}}
\caption{การทำงานของระบบเมื่อผู้ใช้กดปุ่มเขียนรีวิว}
\label{Fig:4-review}
\end{figure}
\newpage

จากภาพที่ \ref{Fig:4-review} โครงสร้างของไฟล์ dialog-review.component.ts สามารถอธิบายการทำงานได้ดังนี้(ต่อ)
\begin{itemize}[label={--}]
\item บรรทัดที่ 1 - 3 ฟังก์ชัน onClick() เป็นฟังก์ชันของการให้คะแนนสถานที่ โดยจะเก็บคะแนนที่ผู้ใช้ให้ไว้ในต rating
\item บรรทัดที่ 4 ฟังก์ชัน reviewPlace() จะทำงานเมื่อผู้ใช้กดปุ่ม บันทุกรีวิว โดยจะทำการบันทึกข้อมูลรีวิว
\item บรรทัดที่ 5 เป็นการประกาศ now ขึ้นมาเก็บค่าของวันที่ในการเขียนรีวิว โดยจะมี format วันที่เป็น วัน/เดือน/ปี
\item บรรทัดที่ 6 เป็นเงื่อนไขในการเช็คว่า id ของรีวิวสถานที่นี้ว่า มีค่าเท่ากับ undefined หรือไม่ถ้าเท่าจะเข้ามาทำงานในเงื่อนไขนี้
\item บรรทัดที่ 7 - 8 เป็นการประกาศตัวแปรชื่อ dataReview มาเก็บข้อมูลการเขียนรีวิวโดยใน dataReview จะมีการเก็บเป็นอาเรย์ในตัวแปร review
\item บรรทัดที่ 10 เป็นการเก็บ id ของสถานที่รีวิวไว้ในตัวแปรชื่อ id 
\item บรรทัดที่ 11 เป็นการเก็บชื่อผู้ใช้ที่เขียนรีวิวโดยจะไปอ่านค่ามาจาก localStoragemใน filed ชื่อ username และเก็บไว้ในตัวแปรชื่อ createby
\item บรรทัดที่ 12 เป็นการเก็บรูปภาพของผู้ใช้ที่เขียนรีวิวโดยจะไปอ่านค่ามาจาก localStoragemใน filed ชื่อ imageSrc และเก็บไว้ในตัวแปรชื่อ imageUser
\item บรรทัดที่ 13 เป็นการเก็บหัวข้อของรีวิวไว้ในตัวแปรชื่อ title
\item บรรทัดที่ 14 เป็นการเก็บรายละเอียดของการเขียนรีวิวไว้ในตัวแปรชื่อ description
\item บรรทัดที่ 15 เป็นการเก็บรูปภาพรีวิวไว้ในตัวแปรชื่อ image
\item บรรทัดที่ 16 เป็นการเก็บค่าของคะแนนรีวิวไว้ในตัวแปร score
\item บรรทัดที่ 17 เป็นการเก็บเวลาของงการเขียนรีวิวไว้ในตัวแปรชื่อ timeReview
\item บรรทัดที่ 21 - 23 เป็นการทำงานโดยจะเอาข้อมูลที่ได้จากการเขียนรีวิวในตัวแปรชื่อ dataReview ไปทำการสร้าง document เพื่อเก็บข้อมูลรีวิวของสถานที่นั้นโดยใช้ method post ในการสร้าง หลังจากสร้างเสร็จจะ log ค่าออกมาดูเพื่อดูผลลัพธ์
\item บรรทัดที่ 25 - 29 เป็นการประกาศตัวแปรชื่อ reviews มาเก็บประวัติของผู้ใช้ว่าเคยเขียนรีวิวสถานที่ไหนบ้าง โดยจะเก็บ id คือเก็บชื่อสถานที่ type คือประเภทสถานที่และ review คือประเภทของสถานที่ที่เขียนรีวิว
\item บรรทัดที่ 30 -32 เป็นการนำข้อมูลที่อยู่ในก้อน reviews ไปทำการอัปเดทข้อมูลผู้ใช้ในส่วนของการเขียนรีวิว
\end{itemize}

\begin{figure}[H]
{\setstretch{1.0}\begin{lstlisting}
} else {
this.dataReview = {
review: [
...this.result_ReviewPlace,
{
id : this.id_place,
create_by: JSON.parse(localStorage.getItem("UserInfo")).username,
imageUser: JSON.parse(localStorage.getItem("UserInfo")).imageSrc,
title: this.title,
description: this.description,
image: this.imageReview,
score: this.rating,
timeReview: now
}
]
}
this.http.put(`${KUZZLE}:${KPORT}/${KINDEX}/${this.idReview}/${this.id_place}/_update`, this.dataReview)
.subscribe((response) => {
console.log(response);
}, err => {
console.log({ ...err });
})
\end{lstlisting}}
\caption{การทำงานของระบบเมื่อผู้ใช้กดปุ่มเขียนรีวิว(ต่อ)}
\label{Fig:4-review1}
\end{figure}
\newpage

จากภาพที่ \ref{Fig:4-review1} โครงสร้างของไฟล์ dialog-review.component.ts สามารถอธิบายการทำงานได้ดังนี้(ต่อ)
\begin{itemize}[label={--}]
\item บรรทัดที่ 1 เป็นเงื่อนไขเช็คว่าถ้า id ของสถานที่ไม่เท่ากับ undefined ก็จะเข้ามาทำงานในส่วนนี้
\item บรรทัดที่ 2 - 3 เป็นการประกาศตัวแปรชื่อ dataReview มาเก็บข้อมูลการเขียนรีวิวโดยใน dataReview จะมีการเก็บเป็นอาเรย์ในตัวแปร review
\item บรรทัดที่ 4 เป็นนำค่ารีวิวเดิมเดิมที่มีอยู่ มาต่อกับรีวิวตัวใหม่
\item บรรทัดที่ 6 เป็นการเก็บ id ของสถานที่รีวิวไว้ในตัวแปรชื่อ id
\item บรรทัดที่ 7 เป็นการเก็บชื่อผู้ใช้ที่เขียนรีวิวโดยจะไปอ่านค่ามาจาก localStoragemใน filed ชื่อ username และเก็บไว้ในตัวแปรชื่อ createby
\item บรรทัดที่ 8 เป็นการเก็บรูปภาพของผู้ใช้ที่เขียนรีวิวโดยจะไปอ่านค่ามาจาก localStoragemใน filed ชื่อ imageSrc และเก็บไว้ในตัวแปรชื่อ imageUser
\item บรรทัดที่ 9 เป็นการเก็บหัวข้อของรีวิวไว้ในตัวแปรชื่อ title
\item บรรทัดที่ 10 เป็นการเก็บรายละเอียดของการเขียนรีวิวไว้ในตัวแปรชื่อ description
\item บรรทัดที่ 11 เป็นการเก็บรูปภาพรีวิวไว้ในตัวแปรชื่อ image
\item บรรทัดที่ 12 เป็นการเก็บค่าของคะแนนรีวิวไว้ในตัวแปร score
\item บรรทัดที่ 13 เป็นการเก็บเวลาของงการเขียนรีวิวไว้ในตัวแปรชื่อ timeReview
\item บรรทัดที่ 17 - 19 เป็นการทำงานโดยเอาข้อมูลที่ได้จาก dataReview ไปทำการอัปเดทข้อมูลการเขียนรีวิวของสถานที่นั้้นในฐานจข้อมูลโดยทำผ่าน method put และเมื่อทำสำเร็จจะแสดงข้อความบอกว่า เพิ่มคอมเม้นสำเร็จ ใน log
\item บรรทัดที่ 21 - 22 ถ้าทำงานล้มเหลวจะให้แสดง error ออกทาง log
\end{itemize}
\newpage

\begin{figure}[H]
{\setstretch{1.0}\begin{lstlisting}
const check = this.reviews.find(item => item.id == this.id_place) 
if (check) {
this.reviews.map((item, index) => {
item.id === this.id_place ? this.reviews.splice(index, 1) : null 
})
this.reviews = [...this.reviews, {
id: this.id_place,
type: this.type_place,
review: this.idReview
]
this.http.put(`${KUZZLE}:${KPORT}/users/${this.id_user}/_update`, {
reviews: JSON.stringify(this.reviews)
})
.subscribe(res => {
console.log(res);
 }, err => {
console.log(err);
 })
}

onNoClick(): void {
this.dialogRef.close();
}
\end{lstlisting}}
\caption{การทำงานของระบบเมื่อผู้ใช้กดปุ่มเขียนรีวิว(ต่อ)}
\label{Fig:4-review2}
\end{figure}
\newpage

จากภาพที่ \ref{Fig:4-review2} โครงสร้างของไฟล์ dialog-review.component.ts สามารถอธิบายการทำงานได้ดังนี้(ต่อ)
\begin{itemize}[label={--}]
\item บรรทัดที่ 1 เป็นการประกาศ check ขึ้นมาเพื่อค้นหาข้อมูลการเขียนรีวิวทั้งหมดที่มีอยู่โดยใช้ item เป็นรับข้อมูลเข้าไปค้าหาต่อ
\item บรรทัดที่ 2 - 3 เป็นการเช็คเงื่อนไขว่าถ้าเช็ค เมื่อได้ข้อมูล reviews ก็จะทำการวนฟอเพื่อดูข้อมูล
\item บรรทัดที่ 4 - 5 เมื่อมันผ่านเงื่อนไขก็จะเข้ามาเช็คดูอีดครั้งว่าค่าของ id มีค่าเท่ากับ idplaceห ไหม ถ้าตรงกันก็จะให้รีวิวที่มีอยู่ออกไปแต่ถ้าไม่ใช้ก็จะให้ค่าตัวนั้นมี่ค่าเป็น null
\item บรรทัดที่ 6 - 10 เป็นการประกาศตัวแปรชื่อ review ขึ้นมาเพื่อเก็บประวัติข้อมูลการเขียนรีวิวของผูัใช้
\item บรรทัดที่ 11 - 12 นำข้อมูลที่ได้จาก reviews ไปทำการอัพเดทข้อมูลของผูัใช้
\item บรรทัดที่ 14 - 19 เป็นการ log ดูค่าเมื่ออัพเดทสำเร็จก็ และเมื่ออัพดเทไม่สำเ้จจะแสเง error
\item บรรทัดที่ 21 - 23 ฟังก์ชัน onNoClick เมื่อผู้ใช่ไม่ต้องทีจะเขียนเรีวิว ระบบก็จะปิดหน้าเขียนรีวิวลง
\end{itemize}
\newpage

\section{การพัฒนาในส่วนของการเพิ่มสถานที่}
เมื่อผู้ใช้กดปุ่มเพิ่มสถานที่ ระบบจะมีการทำงาน แสดงดังรูปที่ \ref{Fig:4-addplace} - \ref{Fig:4-addplace2}
\begin{figure}[H]
{\setstretch{1.0}\begin{lstlisting}
onFileSelected(e) {
var file = e.dataTransfer ? e.dataTransfer.files[0] : e.target.files[0];
var pattern = /image-*/;
var reader = new FileReader();
if (!file.type.match(pattern)) {
alert('invalid format');
return;
}
reader.onload = this._handleReaderLoaded.bind(this);
reader.readAsDataURL(file);
}
_handleReaderLoaded(e) {
let reader = e.target;
this.imageSrc = reader.result;
}
\end{lstlisting}}
\caption{การทำงานของระบบเมื่อผู้ใช้กดปุ่มเพิ่มสถานที่}
\label{Fig:4-addplace}
\end{figure}

จากภาพที่ \ref{Fig:4-addplace} โครงสร้างของไฟล์ dialog-add.component.ts สามารถอธิบายการทำงานได้ดังนี้
\begin{itemize}[label={--}]
\item บรรทัดที่ 1 - 15 เป็นฟังก์ชันการการอัพโหลดรูปภาพแนวนำรูปภาพที่อัพโหลดมาแปลงเป็น base64 เพื่อที่จะไปเก็บไว้ใในฐานข้อมูล 
\end{itemize}
\newpage

\begin{figure}[H]
{\setstretch{1.0}\lstset{language=Pascal}
\begin{lstlisting}
addPlace() {
var selectplace = ''
if(this.typeplace == "hotelvilla"){
selectplace = "วิลล่า"; }
else if(this.typeplace == "hotelhostel"){
selectplace = "โรงแรม";}
else if(this.typeplace == "hotelresort"){
selectplace = "รีสอร์ท";}
else if(this.typeplace == "hoteltent"){
selectplace = "กางเต้นท์";}
else if(this.typeplace == "hotelgateshouse"){
selectplace = "บังกะโล/เกสท์เฮาส์";}
else if(this.typeplace == "hotelhomestay"){
selectplace = "โฮมสเตย์";}
else if(this.typeplace == "travelwaterfall"){
selectplace = "น้ำตก";}
else if(this.typeplace == "travelwaterpark"){
selectplace = "สวนน้ำ/สวนสัตว์";}
else if(this.typeplace == "traveltemple"){
selectplace = "สถานที่ทางศาสนา";}
else if(this.typeplace == "travelnational"){
selectplace = "อุทยานและธรรมชาติ";}
else if(this.typeplace == "travelshop"){
selectplace = "ถนนคนเดิน";}
else if(this.typeplace == "travelbar"){
selectplace = "สถานบันเทิง";}
else if(this.typeplace == "eatgrill"){
selectplace = "บุฟเฟ่ต์/ปิ้งย่าง";}
else if(this.typeplace == "eatdrinks"){
selectplace = "คาเฟ่";}
else if(this.typeplace == "eatnoodle"){
selectplace = "เส้น";}
else if(this.typeplace == "eatesan"){
selectplace = "อาหารท้องถิ่น";}
else if(this.typeplace == "eatjapan"){
selectplace = "อาหารญี่ปุ่น";}
else if(this.typeplace == "eatsteak"){
selectplace = "สเต็ก";}
\end{lstlisting}}
\caption{การทำงานของระบบเมื่อผู้ใช้กดปุ่มเพิ่มสถานที่(ต่อ)}
\label{Fig:4-addplace1}
\end{figure}
\newpage

จากภาพที่ \ref{Fig:4-addplace1} โครงสร้างของไฟล์ dialog-add.component.ts สามารถอธิบายการทำงานได้ดังนี้
\begin{itemize}[label={--}]
\item บรรทัดที่ 1 ฟังก์ชัน addPlace() เป็นฟังก์ชันที่ใช้ทำการเพิ่มข้อมูลสถานที่จะทำงานก็ต่อเมื่อผู้ใช้กดปุ่มบันทึก
\item บรรทัดที่ 2 เป็นการประกาศตัวแปร selectplace ที่มี type เป็น string เพื่อที่จะใช้เก็บค่าประเภทของสถานที่ที่ผู้ใช้เลือก
\item บรรทัดที่ 3 - 18 เป็นการเช็คเงื่อนไขเมื่อผู้ใช้เลือกประเภทของสถานที่เพื่อที่จะกำหนดค่าให้กับ selectplace 
\end{itemize}

\begin{figure}[H]
{\setstretch{1.0}\lstset{language=Pascal}
\begin{lstlisting}
this.data = {
"image": this.imageSrc,
"namelocation": this.namelocation,
"group": selectplace,
"address": this.address,
"cityprovince": this.cityprovince,
"lat": this.lat,
"lng": this.lng,
"openclose": this.openclose,
"phonenumber": this.phonenumber,
"price": this.price,
"description": this.description,
"facebook": this.facebook,
"website": this.website
};
this.http.post(`${KUZZLE}:${KPORT}/${KINDEX}/${this.typeplace}/_create`,this.data)
.subscribe((res:any) =>{
const idname =res.result._id
this.places = [...this.places, {
id: idname,
type: this.typeplace,
review: this._typeReview
}]
this.http.put(`${KUZZLE}:${KPORT}/users/${this.id_user}/_update`, {
places: JSON.stringify(this.places)
})
\end{lstlisting}}
\caption{การทำงานของระบบเมื่อผู้ใช้กดปุ่มเพิ่มสถานที่(ต่อ)}
\label{Fig:4-addplace2}
\end{figure}
\newpage
จากภาพที่ \ref{Fig:4-addplace2} โครงสร้างของไฟล์ dialog-add.component.ts สามารถอธิบายการทำงานได้ดังนี้
\begin{itemize}[label={--}]
\item บรรทัดที่ 1 เป็นการประกาศตัวแปรชื่อ data โดยมี type เป็น object เพื่อที่จะมาเก็บค่าของรายละเอียดข้อมูลสถานที่ที่ผู้ใช้กรอกเข้ามา
\item บรรทัดที่ 2 เป็นการประกาศ image ขึ้นมาเก็บรูปภาพของสถานที่โดยมีรูปแบการเก็บเป็น base64
\item บรรทัดที่ 3 เป็นการประกาศ namelocation ขึ้นมาเพื่อที่จะเก็บชื่อสถานที่
\item บรรทัดที่ 4 เป็นการเก็บประเภทของสถานที่ที่ได้จากการเลือกของผู้ใช้ไว้ใน group 
\item บรรทัดที่ 5 เป็นการเก็บที่อยู่ของสถานที่ที่ได้จากการกรอกเข้ามาจากผู้ใช้ไว้ใน address
\item บรรทัดที่ 6 เป็นการเก็บชื่ออำเภอและจังหวัดที่ได้จากการข้อมูลเข้ามาจากผู้ใชเไว้ใน cityprovince
\item บรรทัดที่ 7 เป็นการเก็บค่าละติจูดที่ได้จากการกรอกข้อมูลเข้ามาจากผู้ใช้ไว้ใน lat
\item บรรทัดที่ 8 เป็นการเก็บค่าลองจิจูดที่ได้จากการกรอกข้อมูลเข้ามาจากผู้ใช้ไว้ใน lng
\item บรรทัดที่ 9 เป็นการเก็บข้อมูลเวลาเปิด ปิด ของสถานที่ที่ได้จากการกรอกเข้ามาจากผู้ใช้ไว้ใน openclose
\item บรรทัดที่ 10 เป็นการเก็บข้อมูลเบอร์โทรของสถานที่ที่ได้จากการกรอกเข้ามาจากผู้ใช้ไว้ใน phonenumber
\item บรรทัดที่ 11 เป็นการเก็บข้อมูลราคาของอาหาร ราคาเข้าชมสถานที่ ที่ได้จากกการกรอกเข้ามาจากผู้ใช้ไว้ใน price
\item บรรทัดที่ 12 เป็นการเก็บข้อมูลรายละเอียดของสถานที่ที่ได้จากการกรอกเข้ามาจากผู้ใช้ไว้ใน description
\item บรรทัดที่ 13 เป็นการเก็บชื่อ facebook ของสถานที่ทีได้จากการกรอกเข้ามาจากผู้ใช้ไว้ใน facebook
\item บรรทัดที่ 14 เป็นการเก็บชื่อ website ของสถานที่ที่ได้จากกการกรอกเข้ามาจากผู้ใช้ไว้ใน website
\item บรรทัดที่ 16 เป็น การนำข้อมูลสถานที่ที่ได้จาก data ไปทำการเพิ่มข้อมูลลงในฐานข้อมูลตามเป็นเภทที่ผู้ใช้เลือก
\item บรรทัดที่ 18 เป็นการประกาศตัวแปรชื่อ idname เพื่อมาเก็บค่าของ id สถานที่
\item บรรทัดที่ 20 เป็นการเรียกใช้ places เพื่อที่จะนำมาเก็บค่าข้อมูล
\item บรรทัดที่ 21 เป็นการเก็บชื่อ id สถานที่ไว้ใน id
\item บรรทัดที่ 22 เป็นการเก็บประเภทของสถานที่ไว้ใน type
\item บรรทัดที่ 23 เป็นการเก็บประเภทของรีวิวสถานที่ไว้ใน review
\item บรรทัดที่ 25 - 27 เป็นการนำข้อมูลที่ได้จาก places มาทำการอัพเดทข้อมูลในส่วนของผู้ใช้เพื่อที่จะเก็บประวัติการเพิ่มสถานที่ของผู้ใช้ไว้และนำข้อมูลที่ได้ไปแสดงในหน้าโปรไฟล์ผู้ใช้
\end{itemize}
\newpage

\section{การพัฒนาในส่วนของการลบรีวิว}
เมื่อผู้ใช้กดปุ่มลบรีวิว ระบบจะมีการทำงาน แสดงดังรูปที่ \ref{Fig:4-deletere}
\begin{figure}[H]
{\setstretch{1.0}\begin{lstlisting}
deleteReview() {
this.http.delete(`${KUZZLE}:${KPORT}/${KINDEX}/${this.type_review}/${this.id_place}`)
.subscribe((response) =>{
});
const check = this.reviews.find(item => item.id == this.id_place)
if (check) {
this.reviews.map((item, index) => {
if( item.id === this.id_place ){
this.reviews.splice(index, 1)
this.http.put(`${KUZZLE}:${KPORT}/users/${this.id_user}/_update`, {
reviews: JSON.stringify(this.reviews)
})
.subscribe(res => {
}, err => {
console.log(err);
})
}
})
}
}
\end{lstlisting}}
\caption{การทำงานของระบบเมื่อผู้ใช้กดปุ่มลบรีวิว}
\label{Fig:4-deletere}
\end{figure}
\newpage
จากภาพที่ \ref{Fig:4-addplace} โครงสร้างของไฟล์ delete-review.component.ts สามารถอธิบายการทำงานได้ดังนี้
\begin{itemize}[label={--}]
\item บรรทัดที่ 1 ฟังก์ชัน deleteReview()g เป็นฟังก์ชันที่ใช้ในการลบรีวิวจะถูกเรียกใช้เมื่อผู้ใช้กดปุ่มลบรีวิว
\item บรรทัดที่ 2 - 5 เป็นการลบข้อมูลรีวิวออกจากฐานข้อมูลโดยจะลบตามประเภทของสถานที่และ id ของสถานที่ที่ โดยใช้ method delete
\item บรรทัดที่ 6 เป็นการประกาศ check ขึ้นมาเพื่อค้นหาข้อมูลการเขียนรีวิวทั้งหมดที่มีอยู่ในฐานข้อมูลของผู้ใช้โดยใช้ item เป็นรับข้อมูลเข้าไปค้าหาต่อ
\item บรรทัดที่ 7 - 10  เป็นการเช็คเงื่อนไขเมื่อเข้าเงื่อนไขก็จะเข้ามาทำงานในส่วนนี้โดยใช้ข้อมูลที่เก็บไว้ใน reviews ทำการวนฟอเพื่อดูข้อมูล และก็เข้าเงื่อนไขก็จะเช็คดูอีกครั้งว่าค่าของ id มีค่าเท่ากับ idplaceห ไหม ถ้าตรงกันก็จะให้รีวิวลบรีวิวที่มีชื่อ id ตรงกับตัวที่ค้นหาออก
\item บรรทัดที่ 11 - 15 เป็นการอัพเดทข้อมูลในส่วนของผู้ใช้อีกครั้ง โดยใช้ method put เพื่อที่จะทำการอัพเดทข้อมูลการเขียนรีวิว
\item บรรทัดที่ 16 - 18 เป็นการ log ดูค่า error ในกรณีที่อัพเดทข้อมูลไม่สำเร็จ
\end{itemize}
\newpage
\section{การพัฒนาในส่วนของการบันทึกสถานที่}
เมื่อผู้ใช้กดปุ่มบันทึก ระบบจะมีการทำงาน แสดงดังรูปที่ \ref{Fig:4-bookmark}
\begin{figure}[H]
{\setstretch{1.0}\lstset{language=Pascal}
\begin{lstlisting}
changeBookMark(place: any) {
if (this.loginData.isAuth === false) {
this.modalService.info({
nzTitle: 'กรุณาล็อคอินเข้าสู่ระบบก่อนกดปุ่มบันทึกสถานที่'
});
}else {
this.dataplace = {
id: place._id,
type: place._type,
type_review: this.type_Review
}
const check = this.bookmarks.find(item => item.id == place._id)
if (check) {
this.bookmarks.map((item, index) => {
item.id === place._id ? this.bookmarks.splice(index, 1) : null 
})
}
else {
this.bookmarks.push({
id: place._id,
type: place._type,
review: this.type_Review
})
}
this.http.put(`${KUZZLE}:${KPORT}/users/${this.id_User}/_update`, {
bookmarks: JSON.stringify(this.bookmarks)
})
.subscribe(res => {
console.log(res);
this.bookmark = !this.bookmark
 })
}
}
\end{lstlisting}}
\caption{การทำงานของระบบเมื่อผู้ใช้กดปุ่มบันทึกสถานที่}
\label{Fig:4-bookmark}
\end{figure}
\newpage
จากภาพที่ \ref{Fig:4-addplace} โครงสร้างของไฟล์ detail-place.component.ts สามารถอธิบายการทำงานได้ดังนี้
\begin{itemize}[label={--}]
\item บรรทัดที่ 1 ฟังก์ชัน changeBookMark() ฟังก์ชันนี้จะถูกเรียกใช้ก็ต่อเมื่อผู้ใช้กดปุ่มบันทึก
\item บรรทัดที่ 2 - 5 เป็นการเช็คดูสถานะการล็อคอิน ถ้าสถานะการเข้าสู่ระบบเป็น false จะมีข้อความแจ้งผู้ใช้ให้เข้าสู่ระบบก่อนกดปุ่มบันทึกสถานที่
\item บรรทัดที่ 6 - 11 เป็นการเช็คดูสถานะการล็อคอิน ในกรณีเข้าสู่ระบบสำเร็จแล้วผู้ใช้สมารถกดปุ่มบันทึกสถานที่ได้ โดยจะเก็บค่าของ id สถานที่ type สถานที่ type ของรีวิว ไว้ในตัวแปรชื่อ dataplace 
\item บรรทัดที่ 12 เป็นการประกาศ check ขึ้นมาเพื่อเก็บค่าผลลัพธ์การค้นหาข้อมูลการบันทึกสถานที่ทั้งหมดที่มีอยู่ในฐานข้อมูลของผู้ใช้โดยใช้ item เป็นรับข้อมูลในการค้นหา
\item บรรทัดที่ 13 - 15 เป็นการเช็คเงื่อนไขเมื่อเข้าเงื่อนไขก็จะเข้ามาทำงานในส่วนนี้โดยใช้ข้อมูลที่เก็บไว้ใน bookmarks มาทำการวนฟอเพื่อดูข้อมูล และก็เข้าเงื่อนไขจะเช็คดูอีกครั้งว่าค่าของ item.id มีค่าเท่ากับ place.id ไหม ถ้าตรงกันก็จะให้ลบชื่อสถานที่ตัวนั้นทิ้ง
\item บรรทัดที่ 18 - 24 เป็นการเช็คเงื่อนไขในกรณีที่ข้อมูลการบันทึกสถานที่นั้นยังไม่เคยกดบันทึกก็จะทำการเพิ่มข้อมูลสถานที่นั้นลงไปใน bookmarks
\item บรรทัดที่ 25 - 29 เป็นการอัพเดทข้อมูลในส่วนของ filed ที่ชื่อ bookmarks ในฐานข้อมูลของผู้ใช้ โดยอัพเดทผ่าน method put 
\item บรรทัดที่ 30 เป็นการรีเซตค่าตรงปุ่มกดบันทึกให้เปลี่ยนเป็นบันทึก
\end{itemize}
\newpage

\section{การพัฒนาในส่วนของการแจ้งแก้ไขข้อมูล}
เมื่อผู้ใช้กดปุ่มแจ้งแก้ไข ระบบจะมีการทำงาน แสดงดังรูปที่ \ref{Fig:4-edit} - \ref{Fig:4-edit1}
\begin{figure}[H]
{\setstretch{1.0}\lstset{language=Pascal}
\begin{lstlisting}
editDataPlace() {
if (this.id_place !== undefined) {
this.dataPlace = {
edit_by: this.edit_by,
img_user: this.img_user,
namelocation: this.editForm.value.namelocation,
group: this.group,
address: this.editForm.value.address,
cityprovince: this.editForm.value.cityprovince,
description: this.editForm.value.description,
phonenumber: this.editForm.value.phonenumber,
price: this.editForm.value.price,
openclose: this.editForm.value.openclose,
website: this.editForm.value.website,
facebook: this.editForm.value.facebook,
lat: this.editForm.value.lat,
lng: this.editForm.value.lng,
image: this.imgs,
typeplace: this.typeplace
}
this.http.put(`${KUZZLE}:${KPORT}/${KINDEX}/dataedit/${this._id}/_update`,this.dataPlace)
.subscribe((response) => {
console.log(response);})
}
\end{lstlisting}}
\caption{การทำงานของระบบเมื่อผู้ใช้กดปุ่มแจ้งแก้ไขสถานที่}
\label{Fig:4-edit}
\end{figure}
จากภาพที่ \ref{Fig:4-edit} โครงสร้างของไฟล์ detail-place.component.ts สามารถอธิบายการทำงานได้ดังนี้
\begin{itemize}[label={--}]
\item บรรทัดที่ 1 ฟังก์ชัน editDataPlace() ฟังก์ชันนี้จะทำงานก็ต่อเมื่อผู้ใช้กดปุ่มแจ้งแก้ไข
\item บรรทัดที่ 2 - 20 เป็นการเช็คเงือนไขดูว่าสถานที่จะแจ้งแก้ไขเคยมีประวัติการแจ้งแก้ไขไหม ถ้า this.idplace != undefined ก็จะเข้ามาทำงานในส่วนนี้โดยจะทำการเก็บข้อมูลรายละเอียดการแก้ไขไว้ในตัวแปรชื่อ dataplace
\item บรรทัดที่ 21 - 24 เป็นการส่งข้อมูลที่ได้จาก dataplace ไปทำการอัพเดทข้อมูลการแจ้งแก้ไขเพื่อที่จะส่งไปให้แอดมินเป็นผู้อนุมัติแก้ไขข้อมูล
\end{itemize}


\begin{figure}[H]
{\setstretch{1.0}\lstset{language=Pascal}
\begin{lstlisting}
else{
this.dataPlace = {
edit_by:this.edit_by,
img_user: this.img_user,
namelocation: this.editForm.value.namelocation,
group: this.group,
address: this.editForm.value.address,
cityprovince: this.editForm.value.cityprovince,
description: this.editForm.value.description,
phonenumber: this.editForm.value.phonenumber,
price: this.editForm.value.price,
openclose: this.editForm.value.openclose,
website: this.editForm.value.website,
facebook: this.editForm.value.facebook,
lat: this.editForm.value.lat,
lng: this.editForm.value.lng,
image: this.imgs,
typeplace: this.typeplace
 }
 this.http.post(`${KUZZLE}:${KPORT}/${KINDEX}/dataedit/${this._id}/_create`,this.dataPlace)
 .subscribe((response) => {
console.log("เก็บข้อมูลการแจ้งแก้ไขเรียบร้อย ", response);})
}
\end{lstlisting}}
\caption{การทำงานของระบบเมื่อผู้ใช้กดปุ่มแจ้งแก้ไขสถานที่}
\label{Fig:4-edit1}
\end{figure}

จากภาพที่ \ref{Fig:4-edit1} โครงสร้างของไฟล์ detail-place.component.ts สามารถอธิบายการทำงานได้ดังนี้
\begin{itemize}[label={--}]
\item บรรทัดที่ 1 - 19 เป็นเช็คต่อจากภาพที่ \ref{Fig:4-edit} โดยในเงื่อนไขนี้คือถ้าของ idสถานที่ที่แจ้งแก้ไขมาเป็น undefined หมายความว่าสถานที่นี้ยังไม่เคยมีประวัติการแจ้งแก้ไขโดยจะทำการเก็บข้อมูลการแจ้งแก้ไขที่ได้จากผู้ใช้มาเก็บไว้ใน dataplace
\item บรรทัดที่ 20 - 23 เป็นการเรียกใช้ dataplace ที่เก็บข้อมูลการแจ้งแก้ไขไว้มาทำการเพิ่มลงไปในฐานข้อมูลของการแจ้งแก้ไข 
\end{itemize}
\newpage

\section{การพัฒนาในส่วนของการค้นหาสถานที่}
เมื่อผู้ใช้กดปุ่มค้นหา ระบบจะมีการทำงาน แสดงดังรูปที่ \ref{Fig:4-search} - \ref{Fig:4-search1}
\begin{figure}[H]
{\setstretch{1.0}\begin{lstlisting}
dataPlace(){
if (this.typeplace == "all") {
this._resultsPlace = []
const list = ["eatgrill", "eatdrinks", "eatnoodle", "eatesan", "eatjapan", "eatsteak"]
list.map(item => {
const query = {
query: {
}, size: 100
}
this.http.post(`${KUZZLE}:${KPORT}/${KINDEX}/${item}/_search`, query)
.subscribe((res: any) => {
this._resultsPlace = [ ...this._resultsPlace,
...res.result.hits]
})
})
} else {
const query = {
query: {
}, size: 100
}
this.http.post(`${KUZZLE}:${KPORT}/${KINDEX}/${this.typeplace}/_search`, query)
.subscribe((res: any) => {
this._resultsPlace = res.result.hits
})
}
}
\end{lstlisting}}
\caption{การทำงานของระบบเมื่อผู้ใช้กดปุ่มเลือกประเภท}
\label{Fig:4-search}
\end{figure}

จากภาพที่ \ref{Fig:4-search} สามารถอธิบายการทำงานของฟังก์ชันการค้นหาสถานที่ได้ดังนี้
\begin{itemize}[label={--}]
\item บรรทัดที่ 1 ฟังก์ชัน dataPlace() จะทำงานก็ต่อเมื่อผู้ใช้กดปุ่มเลือกประเภทสถานที่
\item บรรทัดที่ 2 เป็นการเช็คเงื่อนไขดูว่า ประเภทที่ผู้ใช้เลือกนั้นเป็นประเภทไหน ถ้าผู้ใช้เลือกทั้งหมด ก็จะมาเข้าเงื่อนไขนี้
\item บรรทัดที่ 4 เป็นการประกาศ list ขึ้นมาเก็บค่าของประเภท โดยมี type เป็น array
\item บรรทัดที่ 5- 10 เป็นกานำข้อมูลที่อยู่ใน list มาทำการวนฟอเพื่อนำไปค้นหาข้อมูลในฐานข้อมูลตามประเภททั้งหมดที่มีอยู่ใน list
\item บรรทัดที่ 12 เป็นการทำงานหลังจากที่ค้นหาข้อมูลเสร็จจะนำข้อมูลที่ได้มาเก็บไว้ใน resultsPlace
\item บรรทัดที่ 16 - 24 เป็นการเช็คเงื่อนไขในกรณีที่ประเภทที่ผู้ใช้เลือกมาไม่ใช่ทั้งหมด ก็จะทำการค้นหาข้อมูลในฐานข้อมูลตามประเภทที่ผู้ใช้เลือก เมื่อค้นหาเสร็จก็จะนำข้อมูลที่ได้มาเก็บไว้ใน resultsPlace 
\end{itemize}


\begin{figure}[H]
{\setstretch{1.0}\lstset{language=Pascal}
\begin{lstlisting}
onClickplace(){
if(this.name == null){
alert("กรุณากรอกข้อมูลก่อนกดปุ่มค้นหา");
}else{
let data = {
id_place : this.dataplace.id ,
type_place : this.dataplace.type,
type_review : this.dataplace.type_review,
name: this.name,
}
localStorage.setItem("dataPlace",JSON.stringify(data))
this.router.navigate(['/list-place'])
}
}
\end{lstlisting}}
\caption{การทำงานของระบบเมื่อผู้ใช้กดปุ่มค้นหา(ต่อ)}
\label{Fig:4-search1}
\end{figure}

จากภาพที่ \ref{Fig:4-search1} สามารถอธิบายการทำงานของฟังก์ชันการค้นหาสถานที่ได้ดังนี้
\begin{itemize}[label={--}]
\item บรรทัดที่ 1 ฟังก์ชัน onClickplace() จะทำงานก็ต่อเมื่อผู้ใช้กดปุ่มค้นหา
\item บรรทัดที่ 2 - 3 เป็นการเช็คเงื่อนไขว่าช่องให้กรอกชื่อสถานทีจะต้องไม่เป็น !null ถ้าเป็น null จะเข้าเงื่อนไขนี้ โดยจะมีความแจ้งผู้ใช้ว่ากรุณากรอกข้อมูลก่อนกดปุ่มค้นหา 
\item บรรทัดที่ 4 - 12 เป็นการเช็คเงื่อนไขในกรณีที่ผู้ใช้กรอกชื่อสถานที่เรียบร้อยก่อนกดปุ่มค้นหา ก็จะเข้ามาทำงานในส่วนนี้โดยจะประกาศต้วแปรชื่อ data ขึ้นมาเพื่อเก็บค่าของ id สถานที่ type สถานที่ type ของรีวิว โดยนำข้อมูลที่ได้ไปเขียนลง localStorage ที่มีชื่อว่า dataPlace หลังจากนั้นก็ router ไปหน้า  list-place  
\end{itemize}
\newpage

\section{การพัฒนาในส่วนของการทำแบบประเมิน}
เมื่อผู้ใช้ทำการกรอกแบบประเมิน ระบบจะมีการทำงาน แสดงดังรูปที่ \ref{Fig:4-question} - \ref{Fig:4-question2}
\begin{figure}[H]
{\setstretch{1.0}\begin{lstlisting}
userForm = new FormGroup({
gender: new FormControl('', Validators.required),
age: new FormControl('', Validators.required),
job: new FormControl('', Validators.required),
income: new FormControl('', Validators.required),
live: new FormControl('', Validators.required),
travel_purpose: new FormControl('', Validators.required),
travel_style: new FormControl('', Validators.required),
vehicle: new FormControl('', Validators.required),
info: new FormControl('', Validators.required),
frequency: new FormControl('', Validators.required),
travelUbon: new FormControl('', Validators.required),
goal: new FormControl('', Validators.required),
environment: new FormControl('', Validators.required),
decision: new FormControl('', Validators.required),
convenience: new FormControl('', Validators.required),
want: new FormControl('', Validators.required),
});
\end{lstlisting}}
\caption{การทำงานของระบบเมื่อผู้ใช้กรอกแบบประเมิน}
\label{Fig:4-question}
\end{figure}

จากภาพที่ \ref{Fig:4-question} โครงสร้างของไฟล์ questionnaire.component.ts สามารถอธิบายการทำงานได้ดังนี้
\begin{itemize}[label={--}]
\item บรรทัดที่ 1 - 18 เป็นฟอมที่ใช้เก็บค่าจากผู้ใช้เมื่อผู้ใช้เลือกช้อยส์ของแต่ละคำถามซึ่งจะมีด้วยกันทั้งหมด 16 ข้อ
\end{itemize}
\newpage

\begin{figure}[H]
{\setstretch{1.0}\begin{lstlisting}
predict() {
this.isValidFormSubmitted = false;
if (this.userForm.invalid) {
return;
}
this.isValidFormSubmitted = true;
this.user.q1 = this.userForm.get('gender').value;
this.user.q2 = this.userForm.get('age').value;
this.user.q3 = this.userForm.get('job').value;
this.user.q4 = this.userForm.get('income').value;
this.user.q5 = this.userForm.get('live').value;
this.user.q6 = this.userForm.get('travel_purpose').value;
this.user.q7 = this.userForm.get('travel_style').value;
this.user.q8 = this.userForm.get('vehicle').value;
this.user.q9 = this.userForm.get('info').value;
this.user.q10 = this.userForm.get('frequency').value;
this.user.q11 = this.userForm.get('travelUbon').value;
this.user.q12 = this.userForm.get('goal').value;
this.user.q13 = this.userForm.get('decision').value;
this.user.q14 = this.userForm.get('environment').value;
this.user.q15 = this.userForm.get('convenience').value;
this.user.q16 = this.userForm.get('want').value;

this.resultModel = [
this.user.q1,this.user.q2,this.user.q3,
this.user.q4,this.user.q5,this.user.q6,
this.user.q7,this.user.q8,this.user.q9,
this.user.q10,this.user.q11,this.user.q12,
this.user.q13,this.user.q14,this.user.q15,this.user.q16
]
\end{lstlisting}}
\caption{การทำงานของระบบเมื่อผู้ใช้กรอกแบบประเมินเสร็จกดปุ่มส่งข้อมูล(ต่อ)}
\label{Fig:4-question1}
\end{figure}
\newpage

จากภาพที่ \ref{Fig:4-question1} โครงสร้างของไฟล์ questionnaire.component.ts สามารถอธิบายการทำงานได้ดังนี้
\begin{itemize}[label={--}]
\item บรรทัดที่ 1 ฟังก์ชัน predict() ฟังก์ชันนี้จะทำงานก็ต่อเมื่อผู้ใช้กดปุ่มส่งข้อมูลโดยจะส่งข้อมูลที่ได้จากการกรอกแบบประเมินไปให้ฝั่ง server
\item บรรทัดที่ 2 - 5 เป็นการเช็คดูว่าฟอมที่ผู้ใช้กรอกมาตอบครบทุกข้อหรือไม ถ้ายังไม่ไม่ครบทุกข้อก็จะไม่สามารถกดส่งข้อมูลได้โดยระบบจะแจ้งข้อที่ผู้ใช้ข้าม
\item บรรทัดที่ 6 ในกรณีที่ผู้ใช้ตอบครบทุกข้อก็จะสามารถกดปุ่มส่งข้อมูลได้
\item บรรทัดที่ 7 - 22 เป็นการเก็บค่าข้อมูลคำตอบที่ผู้ใช้เลือกจาก userForm ไว้ใน user.q1 - user.q16 โดยเรียงตามข้อเริ่มจากข้อ 1 ถึง 16
\item บรรทัดที่ 24 - 30 เป็นการเรียกใช้ resultModel เพื่อเก็บค่าผลลัพธ์ของคำตอบที่ผู้ใช้เลือกซึ่งมีการเก็บเป็นแบบ array
\end{itemize}

\begin{figure}[H]
{\setstretch{1.0}\begin{lstlisting}
this.http.post('http://10.80.29.22:8000/api/predict',{
questionnaire : this.resultModel
}).subscribe((res:any) =>{
let resultmodels = {
result: this.resultModel1,
name : res.results,
number : res._predictOut
}
this.resultpredict.sendpredict(resultmodels) 
this.dialog.open(DialogResultComponent, {
});
})
}
\end{lstlisting}}
\caption{การทำงานของระบบเมื่อผู้ใช้กรอกแบบประเมินเสร็จกดปุ่มส่งข้อมูล(ต่อ)}
\label{Fig:4-question2}
\end{figure}

จากภาพที่ \ref{Fig:4-question2} โครงสร้างของไฟล์ questionnaire.component.ts สามารถอธิบายการทำงานได้ดังนี้
\begin{itemize}[label={--}]
\item บรรทัดที่ 1 - 2 เป็นการนำข้อมูลที่ได้จาก resultModel ส่งค่าไปให้ server ที่ port 8000 ทำงานต่อในส่วนของการ predict ข้อมูล
\item บรรทัดที่ 4 - 8 เป็นการนำเรียกข้อมูลได้จากจากฝั่ง server มาเก็บไว้ใน resultmodels โดยจะเก็บผลลัพธ์ของการกรอกแบบสอบถามไว้ใน result เก็บ ข้อมูลผลลัพธ์การ predict ไว้ที่ number
\item บรรทัดที่ 9 เป็นการเรียกใช้ service ของ sendpredict เพื่อที่จะเอาค่าข้อมูลที่เก็บไว้ที่ resultmodels ไปบันทึกไว้ที่ sendpredict 
\item บรรทัดที่ 10 เป็นการเรียกใช้ dialog.open เพื่อเปิดหน้าของ DialogResultComponent มาแสดงผลลัพธ์การแนะนำ
\end{itemize}


\section{การพัฒนาในส่วนของ Server ที่ใช้ในการ train model}
เมื่อผู้ดูแลระบบกดปุ่ม train จะมีกระบวนการทำงาน แสดงดังรูปที่ \ref{Fig:4-train} - \ref{Fig:4-train2}
\begin{figure}[H]
{\setstretch{1.0}\begin{lstlisting}
app.get("/api/trian", (req, res) => {
axios.post('http://10.80.29.22:7512/dbpom/model/_search?size=1500',{})
.then( (response) => {
const datamodel = response.data.result.hits.map((item, index) => {
var xTrains = [];
xTrains.push(item._source.q1)
xTrains.push(item._source.q2)
xTrains.push(item._source.q3)
xTrains.push(item._source.q4)
xTrains.push(item._source.q5)
xTrains.push(item._source.q6)
xTrains.push(item._source.q7)
xTrains.push(item._source.q8)
xTrains.push(item._source.q9)
xTrains.push(item._source.q10)
xTrains.push(item._source.q11)
xTrains.push(item._source.q12)
xTrains.push(item._source.q13)
xTrains.push(item._source.q14)
xTrains.push(item._source.q15)
xTrains.push(item._source.q16)
xTrain.push(xTrains);
yTrain.push(item._source.q17);
})
\end{lstlisting}}
\caption{ขั้นตอนกระบวนการ train model}
\label{Fig:4-train}
\end{figure}
\newpage
จากภาพที่ \ref{Fig:4-train} โครงสร้างของไฟล์ app.js สามารถอธิบายการทำงานได้ดังนี้
\begin{itemize}[label={--}]
\item บรรทัดที่ 1 เมื่อผู้ดูแลระบบกดปุ่ม train ก็จะมาเรียกใช้งานที่ path:"api/train"
\item บรรทัดที่ 2 เป็นการค้นหาข้อมูลจากฐานข้อมูลของ model เพื่อนำข้อมูลที่ผู้ใช้กรอกแบบประเมินมาทำการ train
\item บรรทัดที่ 4 - 24 เป็นการใช้ .map ในการวนฟอค้นหาข้อมูลทั้งหมดที่มีอยู่ในฐานข้อมูลและทำการแยกค่าของ xTrain yTrain โดย xTrain จะเก็บค่าของ q1 - q16 ส่วน yTrain จะเก็บค่าของ q17
\end{itemize}


\begin{figure}[H]
{\setstretch{1.0}\begin{lstlisting}
const XTrain = tf.tensor2d(xTrain)
const YTrain = tf.tensor2d(yTrain.map(item => [
item === 0 ? 1 : 0,
item === 1 ? 1 : 0,
item === 2 ? 1 : 0,
item === 3 ? 1 : 0,
item === 4 ? 1 : 0,
item === 5 ? 1 : 0
])); 

model = tf.sequential();
model.add(tf.layers.dense({
inputShape: [16],
activation: 'sigmoid',
units: 17,
}));
model.add(tf.layers.dense({
inputShape: [17],
activation: 'sigmoid',
units: 6,
}));
model.add(tf.layers.dense({
activation: 'softmax',
units: 6,
}));
model.summary();
\end{lstlisting}}
\caption{ขั้นตอนกระบวนการ train model(ต่อ)}
\label{Fig:4-train1}
\end{figure}
\newpage
จากภาพที่ \ref{Fig:4-train1} โครงสร้างของไฟล์ app.js สามารถอธิบายการทำงานได้ดังนี้
\begin{itemize}[label={--}]
\item บรรทัดที่ 1 - 9 เริ่ม load ข้อมูล xTrain , yTrain โดยค่าของ xTrain จะถูกเก็บไว้ที่ XTrain และค่าของ yTrain จะถูกเก็บไว้ที่ YTrain ซึ่งค่าของ yTrain จะมี 0,1,2,3,4,5
\item บรรทัดที่ 11 เป็นการสร้าง model จาก tf.sequential()
\item บรรทัดที่ 12 - 25 เป็นการเพิ่ม layer sigmoid และ layer softmax
\item บรรทัดที่ 26 เป็นการแสดง summary ของ model
\end{itemize}

\begin{figure}[H]
{\setstretch{1.0}\begin{lstlisting}
const learningRate = 0.06 
const optimizer = tf.train.adam(learningRate);
model.compile({
  optimizer: optimizer,
  loss: 'meanSquaredError', 
  metrics: ['accuracy'],
});
model.fit(XTrain, YTrain).then((history) => {
});
res.send({
xTrain: xTrain,
yTrain : yTrain,
model:model
  })
})
\end{lstlisting}}
\caption{ขั้นตอนกระบวนการ train model(ต่อ)}
\label{Fig:4-train2}
\end{figure}

จากภาพที่ \ref{Fig:4-train2} โครงสร้างของไฟล์ app.js สามารถอธิบายการทำงานได้ดังนี้
\begin{itemize}[label={--}]
\item บรรทัดที่ 1 - 7 เป็นการเริ่ม train model โดยใช้ adam learning
\item บรรทัดที่ 8 สิ้นสุดการ train และได้ model ออกมาเพื่อที่จะนำโมเดลที่ได้ไทำการ predict ต่อ
\item บรรทัดที่ 10 - 15 เป็นการส่งค่ากลับไปให้ฝั่ง Client เพื่อให้รู้ว่าทำงานเสร็จสิ้นแล้ว 
\end{itemize}
\newpage


\section{การพัฒนาในส่วนของ Server ที่ใช้ในการ predict ข้อมูลจากผู้ใช้เมื่อกรอกแบบประเมินเสร็จ}
เมื่อผู้ใช้กดปุ่มส่งข้อมูล จะมีกระบวนการทำงาน แสดงดังรูปที่ \ref{Fig:4-predict}
\begin{figure}[H]
{\setstretch{1.0}\begin{lstlisting}
app.post("/api/predict",(req, res) => {
const result = req.body.questionnaire.map(item => parseInt(item)) 
const xTest = tf.tensor2d([result])
const predictOut = model.predict(xTest)
const category = predictOut.dataSync().map(_item => Math.floor(_item * 100));
category.map((item,index)=> {
dataShow.push({
label : nameResult[index]+' '+item+'%',
value : item,index
  })
})
dataShow = dataShow.sort((a, b) => (a.value > b.value) ? 1 : ((b.value > a.value) ? -1 : 0)).reverse()
return res.send({
  xTest: xTest,
  model : model,
  predictOut: predictOut,
  category: category,
  data : req.body.questionnaire,
  results : dataShow
})
})
\end{lstlisting}}
\caption{ขั้นตอนกระบวนการ predict ข้อมูล}
\label{Fig:4-predict}
\end{figure}
\newpage
จากภาพที่ \ref{Fig:4-predict} โครงสร้างของไฟล์ app.js สามารถอธิบายการทำงานได้ดังนี้
\begin{itemize}[label={--}]
\item บรรทัดที่ 1 เมื่อผู้ใช้กดปุ่มส่งข้อมูล ก็จะมาเรียกใช้งานที่ path:"api/predict" เพื่อนำข้อมูลที่ได้ส่งให้ฝั่ง server ทำการประมวลให้
\item บรรทัดที่ 2 เป็นการประการ result ขึ้นมาเพื่อมาเก็บข้อมูลแบบประเมินที่ผู้ใช้กรอกเข้ามาในฝั่งของ Client ซึ่งค่าที่ส่งมาเป็น string เราจะต้องมาแปลงให้เป็นตัวเลขก่อนโดยใช้คำสั่ง parseInt()
\item บรรทัดที่ 3 เป็นการนำเอา result มาทำให้เป็นข้อมูล 2 มิติ และเก็บไว้ที่ xTest
\item บรรทัดที่ 4 เป็นการนำเอาตัวแปร xTest มาทำการ predict ข้อมูลและเก็บไว้ที่ predictOut
\item บรรทัดที่ 5 เป็นการเรียกใช้ predictOut มาทำการวิเคราะห์ผลลัพธ์การแนะนำโคยค่าทีได้จะออกมาเป็นตัวเลขทศนิยมเราจะต้องนำค่าที่ๆได้มาคิดเป็นเปอร์เซ็นต์และเก็บไว้ที่ตัวแปร category
\item บรรทัดที่ 6 - 11 เป็นการนำค่าของ category มาค้นหาและไป push เก็บไว้ในรูปแบบ label และ value ซึ่ง label จะเก็บชื่อของหมวดและค่าเปอร์เซ็นต์ ส่วน value จะเก็บตำแหน่งของชื่อหมวด และเก็บไว้ใน dataShow
\item บรรทัดที่ 12 เป็นการนำเอาค่าข้อมูลที่เก็บไว้ใน dataShow มาทำการจัดอันดับซึ่งเรียงจากค่าที่มีเปอร์เซ็นต์มากสุดไปน้อยสุด
\item บรรทัดที่ 13 - 20 เป็นการส่งค่าผลลัพธ์ที่ได้กลับไปให้ฝั่ง Client เพื่อทำการแสดงผลข้อมูลการแนะนำให้กับผู้ใช้ได้ทราบ
\end{itemize}
\newpage