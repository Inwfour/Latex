\chapter{การพัฒนาระบบ}

ในบทนี้จะกล่าวถึงการสร้างแอปพลิเคชันสูงวัยมายเฟรนด์ โดยนำผลที่ได้จากการวิเคราะห์และออกแบบระบบมาสร้างเป็นระบบงานซึ่งจะอธิบายถึงตัวอย่างการเขียน โปรแกรมการทำงานของระบบในส่วนต่างๆดังต่อไปนี้

\begin{enumerate}[label=4.\arabic*]
	\item การสมัครสมาชิก
	\item การเข้าสู่ระบบ
	\item ส่วนจัดการแชทบอท
	\item การจัดการส่วนของ Dialogflow
	\item การจัดการโพสท์และคอมเมนท์
	\begin{enumerate}[label=4.5.\arabic*]
		\item การเพิ่มโพสท์
		\item การลบโพสท์
		\item การกดถูกใจโพสท์
		\item การเลือกหมวดโพสท์
		\item การจัดการคอมเมนท์
	\end{enumerate}	
	\item การค้นหาเพื่อน
	\item การเพิ่มการแจ้งเตือนการทานยา
\end{enumerate}	


% start สมัครสมาชิก
\section{การสมัครสมาชิก}
เมื่อผู้ใช้กรอกข้อมูลการสมัครเสร็จแล้วทำการกดปุ่มสมัครสมาชิก ระบบจะมีการทำงาน แสดงดังรูปที่ \ref{Fig:4-Register}
\begin{figure}[H]
{\setstretch{1.0}\lstset{language=Pascal}
\begin{lstlisting}
save(user) {
	if (this.user.email == null || this.user.password == null) {
		this.toastCtrl.create({
		message: "กรุณาระบุชื่อผู้ใช้หรือรหัสผ่านให้ถูกต้อง",
		duration: 3000,
		position: top,
	}).present();
	} else {
		this.registerService.SaveUser(user).then(async (data) => {
		await firebase.auth().signOut();
		this.user.email = "";
		this.user.password = "";
		this.toastCtrl.create({
			message: "บันทึกข้อมูลสำเร็จ",
			duration: 3000,
			position: top,
		}).present();
		}).catch(() => {
		this.toastCtrl.create({
			message: "ชื่อผู้ใช้หรือรหัสผ่านไม่ถูกต้อง",
			duration: 3000,
			position: 'top'
		}).present();
		})
	}
}
\end{lstlisting}}
\caption{Code การทำงานของระบบเมื่อกดปุ่มสมัครสมาชิก}
\label{Fig:4-Register}
\end{figure}
\newpage
จากรูปที่ \ref{Fig:4-Register} โครงสร้างของไฟล์ register.ts สามารถอธิบายการทำงานได้ดังนี้
\begin{itemize}[label={--}]
\item บรรทัดที่ 1 ฟังก์ชัน save() เป็นฟังก์ชันที่ใช้ควบคุมการทำงานในการสมัครสมาชิก
\item บรรทัดที่ 2 - 7 เช็คอีเมลล์และพาสเวิร์ดที่เรากรอกว่าว่างหรือไม่ ถ้าช่องใดช่องหนึ่งจะแสดงข้อความว่า "กรุณาระบุชื่อผู้ใช้หรือรหัสผ่านให้ถูกต้อง" ถ้าไม่ว่างจะนำไปเช็คข้อมูลต่อไป
\item บรรทัดที่ 9 เป็นการเรียกใช้งานฟังก์ชัน saveUser() ในคลาส RegisterProvider เพื่อบันทึกข้อมูลผู้ใช้ลง Authentication Firebase 
\item บรรทัดที่ 10 กำหนดให้ user logout เนื่องจากหลังจาก register เรียบร้อยแล้วฐานข้อมูลจะถูกเก็บ user โดยอัตโนมัติ จึงต้อง logout ออกให้อยู่ในสถานะไม่มี user
\item บรรทัดที่ 11 - 12 กำหนดค่าของช่องอีเมล์และพาสเวิร์ดให้ว่างหลังจากบันทึกข้อมูลเรียบร้อยแล้ว
\item บรรทัดที่ 13 - 17 ถ้าระบบบันทึกข้อมูลลงฐานข้อมูลเรียบร้อยแล้ว จะแสดงข้อความว่า "บันทึกข้อมูลสำเร็จ"
\item บรรทัดที่ 18 - 23 ถูกเรียกใช้กรณีสมัครสมาชิกไม่สำเร็จจะแสดงข้อความว่า "ชื่อผู้ใช้หรือรหัสผ่านไม่ถูกต้อง"
\end{itemize}
\newpage
% end สมัครสมาชิก

% start เข้าสู่ระบบ
\section{การเข้าสู่ระบบ}
เมื่อผู้ใช้กรอกข้อมูลทั้งหมดเสร็จ แล้วทำการกดปุ่มเข้าสู่ระบบ ระบบจะมีการทำงาน แสดงดังรูปที่ \ref{Fig:4-login} - \ref{Fig:4-logincon1}

\begin{figure}[H]
{\setstretch{1.0}\lstset{language=Pascal}
\begin{lstlisting}
login(user) {
let loader = this.loadingCtrl.create({
	spinner: 'hide',
	content: `<img src="assets/imgs/loading.svg">`
})
loader.present();
this.loginservice.loginUser(user).then(async(data) => {
	await this.fireinfo.doc(firebase.auth().currentUser.uid).get().then((res) => {
	if (res.data() === undefined) {
		this.fireinfo.doc(firebase.auth().currentUser.uid).set({
		owner: firebase.auth().currentUser.uid,
		email: firebase.auth().currentUser.email,
		created: firebase.firestore.FieldValue.serverTimestamp(),
		}).then(() => {
		loader.dismiss();
			this.toastCtrl.create({
			message: "เข้าสู่ระบบสำเร็จ",
			duration: 3000,
			position: 'top',
			}).present();
			this.navCtrl.push(RegisterPhotoPage);
		})
		.catch(() => {
			loader.dismiss();
			this.toastCtrl.create({
			message: "บันทึกข้อมูลไม่สำเร็จ",
			duration: 3000,
			position: 'top'
			}).present();
		});
\end{lstlisting}}
\caption{การทำงานของระบบเมื่อกดเข้าสู่ระบบ}
\label{Fig:4-login}
\end{figure}
\newpage

จากรูปที่ \ref{Fig:4-login} โครงสร้างของไฟล์ login.ts สามารถอธิบายการทำงานได้ดังนี้
\begin{itemize}[label={--}]
\item บรรทัดที่ 2 - 6 เป็นการใช้ loadingController เพื่อเริ่มใช้โหลดหน้าจอเมื่อฟังก์ชันถูกกดใช้
\item บรรทัดที่ 7 เป็นการเรียกใช้ฟังก์ชัน loginUser(user) ที่อยู่ในคลาส LoginService เพื่อตรวจสอบการเข้าสู่ระบบว่ามีอยู่ในฐานข้อมูลหรือไม่
\item บรรทัดที่ 8 - 22 เป็นการเช็คข้อมูลในฐานข้อมูลว่ามีข้อมูลหรือไม่ ถ้าไม่มีจะสร้างรายละเอียดข้อมูลผู้ใช้ไว้ในฐานข้อมูล ถ้ามีจะไม่ทำในส่วนนี้
\item บรรทัดที่ 23 - 30 กรณีไม่สามารถทำการสร้างข้อมูลได้ระบบจะแจ้งว่าบันทึกข้อมูลไม่สำเร็จ 
\end{itemize}
\newpage

% end เข้าสู่ระบบ

% start เข้าสู่ระบบต่อ

\begin{figure}[H]
	{\setstretch{1.0}\lstset{language=Pascal}
	\begin{lstlisting}
} else {
	if(res.data().owner_name == undefined && res.data().age == undefined && res.data().phone == undefined && res.data().photoURL == undefined && res.data().disease == undefined){
	console.log(data);
	loader.dismiss();
	this.toastCtrl.create({
		message: "เข้าสู่ระบบสำเร็จ",
		duration: 3000,
		position: 'top',
	}).present();
	this.navCtrl.push(RegisterPhotoPage);
	}else {
		loader.dismiss();
		this.toastCtrl.create({
		message: "เข้าสู่ระบบสำเร็จ",
		duration: 3000,
		position: 'top',
		}).present();
		this.navCtrl.push(TabsPage)
	}
	}
})
}).catch((err) => {
console.log(err);
loader.dismiss();
this.toastCtrl.create({
message: "กรุณากรอกชื่อผู้ใช้หรือรหัสผ่านให้ถูกต้อง",
duration: 3000,
position: 'top',
}).present();
})
}
	\end{lstlisting}}
	\caption{Code การทำงานของระบบเมื่อกดเข้าสู่ระบบ (ต่อ)}
	\label{Fig:4-logincon1}
	\end{figure}
\newpage

จากรูปที่ \ref{Fig:4-logincon1} โครงสร้างของไฟล์ login.ts (ต่อ) สามารถอธิบายการทำงานได้ดังนี้
\begin{itemize}[label={--}]
\item บรรทัดที่ 2 - 9 ทำการเช็คข้อมูลที่อยู่ในฐานข้อมูล ถ้าหากข้อมูลบางอย่างไม่ถูกสร้าง ระบบจะหยุด LoadingController และจะแสดงข้อความเข้าสู่ระบบสำเร็จ หลังจากนั้นจะเรียกใช้งานคลาส RegisterPhotoPage
\item บรรทัดที่ 11 - 18 ถ้าหากข้อมูลถูกสร้างทั้งหมดแล้ว ระบบหยุด LoadingController และจะแสดงข้อความเข้าสู่ระบบสำเร็จ จากนั้นจะเรียกใช้คลาส TabsPage เพื่อไปยังหน้าหลัก
\item บรรทัดที่ 22 - 29 ถ้าหากผู้ใช้กรอกข้อมูลไม่ครบหรือไม่ถูกต้องระบบจะแสดงข้อความว่ากรุณากรอกชื่อผู้ใช้หรือรหัสผ่านให้ถูกต้อง
\end{itemize}
\newpage

% end เข้าสู่ระบบต่อ

% start คุยกับแชทบอท

\section{ส่วนจัดการแชทบอท}
เมื่อผู้ใช้กดเลือกเมนูปู่จอห์น ระบบจะมีการทำงาน แสดงดังรูปที่ \ref{Fig:4-addchatbot}
\begin{figure}[H]
	{\setstretch{1.0}\begin{lstlisting}
sendText() {
window["ApiAIPlugin"].requestText({
query: messages
}, (response) => {
this.ngZone.run(() => {

if(response.result.fulfillment.speech){
try {
	console.log(response);
	this.textres = JSON.parse(response.result.fulfillment.speech);
	this.messages.push({
	text: this.textres.text,
	img: this.textres.img,
	button: this.textres.button,
	list: this.textres.list,
	sender: "api"
	})
} catch(e) {
	this.messages.push({
	text: response.result.fulfillment.speech,
	sender: "api"
	})
}
this.content.scrollToBottom();
}  
})
}, (error) => {
alert(JSON.stringify(error))
})
}
	\end{lstlisting}}
	\caption{Code การทำงานของการคุยกับแชทบอท}
	\label{Fig:4-chatbot}
\end{figure}
\newpage

จากรูปที่ \ref{Fig:4-chatbot} โครงสร้างของไฟล์ chatbot.ts สามารถอธิบายการทำงานได้ดังนี้
\begin{itemize}[label={--}]
\item บรรทัดที่ 2 - 3 เป็นการส่งตัวแปร messages ไป query ที่ Dialogflow เพื่อหาคำตอบ
\item บรรทัดที่ 4 - 17 คือคำตอบที่ถูก response กลับมาจาก Dialogflow จากนั้นนำ response.result.fulfillment.speech เก็บไว้ในตัวแปร textres แล้วส่งไปแสดงหน้าจอในรูปแบบข้อความหรือรูปภาพหรือลิสท์
\item บรรทัดที่ 18 - 22 ถ้าข้อมูลที่ถูกส่งกลับมามีข้อผิดพลาด push ข้อความมาแสดงหน้าจอทันที
\end{itemize}
\newpage

% end คุยกับแชทบอท

% start การพัฒนาในส่วนของ Dialogflow ที่ใช้ fulfillment ในการเชื่อมต่อกับ Firebase

\section{การพัฒนาในส่วนของ Dialogflow ที่ใช้ fulfillment ในการเชื่อมต่อกับ Firebase}
เมื่อผู้ใช้ส่งข้อความมายัง Dialgflow เพื่อหาคำตอบที่่ถูกต้อง ระบบจะมีการทำงาน แสดงดังรูปที่ \ref{Fig:4-addchatbot}
\begin{figure}[H]
	{\setstretch{1.0}\begin{lstlisting}
function disease(agent) {
var dis = request.body.queryResult.parameters.disease;
	var dis_detail = request.body.queryResult.parameters.desease_detail;
var result = "";
if(dis === "เบาหวานชนิดที่1" || dis === "เบาหวานชนิดที่2" || dis === "เบาหวานชนิดที่3"){
	return admin.firestore().collection("disease").doc("เบาหวาน")
.collection(dis).doc(dis_detail).get().then((data) => {
	result = JSON.stringify(data.data());
if(result){
	agent.add(result);
}else{
agent.add("ปู่ไม่รู้เลย ปู่ขอหาข้อมูลก่อนนะ");
} 
	});
}else {
return admin.firestore().collection("disease").doc(dis)
.collection("รายละเอียด").doc(dis_detail).get().then((data) => {
	result = JSON.stringify(data.data());
if(result){
	agent.add(result);
}else{
agent.add("ปู่ไม่รู้เลย ปู่ขอหาข้อมูลก่อนนะ");
}
});
}
}
	\end{lstlisting}}
	\caption{Code ส่วนของ Dialogflow ที่ใช้ fulfillment ในการเชื่อมต่อกับ Firebase}
	\label{Fig:4-addchatbot}
\end{figure}
\newpage

จากรูปที่ \ref{Fig:4-addchatbot} โครงสร้างของไฟล์ index.js อยู่ใน fulfillment ของ Dialgflow สามารถอธิบายการทำงานได้ดังนี้
\begin{itemize}[label={--}]
\item บรรทัดที่ 2 เก็บค่า parameters ข้อมูลของชื่อโรคจาก Intent มาเก็บไว้ในตัวแปร dis
\item บรรทัดที่ 3 เก็บค่า parameters ข้อมูลของรายละเอียดโรคจาก Intent มาเก็บไว้ในตัวแปร disdetail
\item บรรทัดที่ 4 สร้างตัวแปร result ให้เท่ากับสตริงเปล่าเพื่อรอรับค่าที่ถูก return กลับมา
\item บรรทัดที่ 5 เช็คชื่อโรคว่าเป็นข้อมูลชนิดเบาหวานหรือไม่
\item บรรทัดที่ 6 - 8 เป็นการส่งชื่อโรคและรายละเอียดไปเช็คในฐานข้อมูลแล้วคืนค่ากลับมา จากนั้นแปลงข้อมูลเป็นสตริงแล้วเก็บไว้ในตัวแปร result
\item บรรทัดที่ 9 - 10 ส่งค่าตัวแปร result กลับมายังแชทบอท
\item บรรทัดที่ 11 - 12 ถ้าข้อมูลที่ถูกส่งกลับคืนมาเป็น null ระบบจะส่งข้อความว่าปู่ไม่รู้เลย ปู่ขอหาข้อมูลก่อนนะ กลับมา
\item บรรทัดที่ 15 - 22 ถ้าหากข้อมูลไม่ตรงกับที่ if จะทำการค้นหาและส่งกลับมาเก็บใน result และส่งกลับมายังแชทบอท
\end{itemize}
\newpage

% end การพัฒนาในส่วนของ Dialogflow ที่ใช้ fulfillment ในการเชื่อมต่อกับ Firebase

% start ส่วนเพิ่มโพสท์
\section{การจัดการโพสท์}

\subsection{การเพิ่มโพสท์}
เมื่อผู้ใช้กรอกข้อความหรือเพิ่มรูปภาพแล้วกดปุ่มโพสท์ ระบบจะมีการทำงาน แสดงดังรูปที่ \ref{Fig:4-addpost} - \ref{Fig:4-addpostcon1}

\begin{figure}[H]
{\setstretch{1.0}\lstset{language=Pascal}
\begin{lstlisting}
post() {
const alert = this.alertCtrl.create({
title: 'ประเภทโพสท์',
inputs: [
{
	name: 'ทั่วไป',
	type: 'radio',
	label: 'ทั่วไป',
	value: 'other',
	},
	{
	name: 'กีฬา',
	type: 'radio',
	label: 'กีฬา',
	value: 'sport',
	},
{
	name: 'ดนตรี',
	type: 'radio',
	label: 'ดนตรี',
	value: 'music'
},
\end{lstlisting}}

\caption{Code การสร้าง Alert เมื่อคลิกโพสท์}
\label{Fig:4-addpost}
\end{figure}

จากรูปที่ \ref{Fig:4-addpost} โครงสร้างของไฟล์ feed.ts สามารถอธิบายการทำงานได้ดังนี้
\begin{itemize}[label={--}]
\item บรรทัดที่ 2 - 17 เป็นการสร้าง alert หลังจากที่กดปุ่มโพสท์จะมีให้ผู้ใช้เลือกประเภทของโพสท์ดังนี้ ทั่วไป กีฬา ดนตรี
\end{itemize}

% end ส่วนเพิ่มโพสท์ ต่อ

\begin{figure}[H]
	{\setstretch{1.0}\lstset{language=Pascal}
	\begin{lstlisting}
	{
		name: 'ศาสนา',
		type: 'radio',
		label: 'ศาสนา',
		value: 'regilion'
	},
	],
	buttons: [
	{
		text: 'ยกเลิก',
		role: 'cancel',
		cssClass: 'secondary',
		handler: (data) => {
		console.log('Confirm Cancel');
		}
	}, {
	\end{lstlisting}}
	
	\caption{Code การสร้าง Alert เมื่อคลิกโพสท์ (ต่อ)}
	\label{Fig:4-addpost}
	\end{figure}
	
	จากรูปที่ \ref{Fig:4-addpost} โครงสร้างของไฟล์ feed.ts สามารถอธิบายการทำงานได้ดังนี้
	\begin{itemize}[label={--}]
	\item บรรทัดที่ 1 - 6 เป็นการกำหนดรายละเอียดด่านในปุ่ม alert ที่มีตัวเลือกเป็น ศาสนา
	\item บรรทัดที่ 8 - 15 เป็นการกำหนดปุ่มยกเลิก
	\end{itemize}
	\newpage
	
	% end ส่วนเพิ่มโพสท์ ต่อ

% start ส่วนเพิ่มโพสท์ ต่อ

\begin{figure}[H]
	{\setstretch{1.0}\lstset{language=Pascal}
	\begin{lstlisting}
	text: 'โพสท์',
	handler: (data) => {        
	let loader = this.loadingCtrl.create({
		spinner: 'hide',
		content: `<img src="assets/imgs/loading.svg">`
	});
loader.present();
this.CollectionService.PostsCollection().add({
	text: this.text,
	created: firebase.firestore.FieldValue.serverTimestamp(),
	owner: firebase.auth().currentUser.uid,
	owner_name: firebase.auth().currentUser.displayName,
	likes: {
		[`${firebase.auth().currentUser.uid}`]: false
	},
	likesCount: 0,
	photoUser: firebase.auth().currentUser.photoURL,
	type: data
	}).then(async (doc) => {
	if (this.image) {
		await this._POST.uploadImgPost(doc.id, this.image);
	}
	this.text = "";
	this.image = undefined;
	loader.dismiss();
	this.getPosts();
	}).catch((err) => {
	loader.dismiss();
	console.log(err);
	})
}
}
]
});
alert.present();
}
	\end{lstlisting}}
	\caption{Code การเพิ่มโพสท์ (ต่อ)}
	\label{Fig:4-addpostcon1}
	\end{figure}
	\newpage
	
	จากรูปที่ \ref{Fig:4-addpostcon1} โครงสร้างของไฟล์ feed.ts สามารถอธิบายการทำงานได้ดังนี้
	\begin{itemize}[label={--}]
	\item บรรทัดที่ 1 - 2 เป็นปุ่มโพสท์สำหรับยืนยันการโพสท์ จะส่งตัวแปร data ที่เป็นข้อ value ของ alert ที่เราได้เลือก
	\item บรรทัดที่ 3 เป็นการสร้าง loadingController สำหรับรอข้อมูล
	\item บรรทัดที่ 4 ปิดการใช้งาน spinner ของ loading
	\item บรรทัดที่ 5 กำหนด content ของตัวโหลด
	\item บรรทัดที่ 7 เปิดใช้งาน loading ให้ทำงาน
	\item บรรทัดที่ 8 - 18 เป็นการเพิ่มข้อมูลโพสท์ไปยังฐานข้อมูล
	\item บรรทัดที่ 19 - 26 เมื่อโพสท์ถูกสร้างเรียบร้อยแล้วจะตรวจสอบว่าโพสท์มีรูปภาพหรือไม่ ถ้ามีจะเรียกใช้ uploadImgPost() ในคลาส PostProvider จะส่งค่า id และ image ไปสร้างใน Storage และเรียกใช้ฟังก์ชัน getPosts() เพื่อรีเฟรชหน้า
	\item บรรทัดที่ 27 - 29 ถ้าหากโพสท์ไม่สามารถสร้างได้จะแสดงข้อผิดพลาดกับระบบ
	\end{itemize}
	\newpage

% end ส่วนเพิ่มโพสท์ ต่อ

% start ส่วนการลบโพสท์

\subsection{การลบโพสท์}
เมื่อเลือกที่ปุ่มลบโพสท์ ระบบจะมีการทำงาน แสดงดังรูปที่ \ref{Fig:4-deletepost}

\begin{figure}[H]
{\setstretch{1.0}\lstset{language=Pascal}
\begin{lstlisting}
delete(post) {
let alert = this.actionSheetCtrl.create({
title: "คุณต้องการที่จะลบโพสท์ ?",
buttons: [
{
text: "ยืนยัน",
handler: () => {
let loader = this.loadingCtrl.create({
spinner: 'hide',
content: `<img src="assets/imgs/loading.svg">`
});
loader.present();
this.CollectionService.PostsCollection().doc(post.id).delete()
.then(() => {
console.log("Success Uid Posts = " + post.id);
this.CollectionService.CommentsCollection().where("post", "==", post.id)
.get()
.then((data) => {
data.forEach(function (doc) {
	firebase.firestore().collection("comments").doc(doc.id).delete()
	.then(() => {
		console.log("Success Uid Comments = " + doc.id);
		loader.dismiss();
	}).catch((err) => {
		loader.dismiss();
		console.log(err);
})
\end{lstlisting}}
\caption{Code การลบโพสท์}
\label{Fig:4-deletepost}
\end{figure}
\newpage

จากรูปที่ \ref{Fig:4-deletepost} โครงสร้างของไฟล์ feed.ts สามารถอธิบายการทำงานได้ดังนี้
\begin{itemize}[label={--}]
\item บรรทัดที่ 2 เป็นการสร้างแอคชั่นชีทเพื่อแสดงข้อความยืนยันการลบโพสท์
\item บรรทัดที่ 13 เป็นการกำหนด collection และ documents เพื่อส่ง id ของโพสท์ไปลบที่ฐานข้อมูล
\item บรรทัดที่ 16 ค้นหาตัวแปร post ในฐานข้อมูลที่เท่ากับ id ของโพสท์
\item บรรทัดที่ 19 เป็นการวนลูปตัวแปร data ที่ถูกส่งกลับมาไว้ในตัวแปร doc
\item บรรทัดที่ 20 เป็นการลบคอมเมนท์ตาม id ของโพสท์นั้น
\end{itemize}
\newpage

% end ส่วนการลบโพสท์

% start ส่วนการลบโพสท์ ต่อ

\begin{figure}[H]
	{\setstretch{1.0}\lstset{language=Pascal}
	\begin{lstlisting}
this.comments = data.docs.length;
console.log(this.comments);
loader.dismiss();
this.getPosts();
}).catch((err) => {
console.log(err);
})
}).catch((error) => {
console.error("Error removing document: ", error);
this.getPosts();
});
}
},
{
text: "กลับ",
handler: () => {
console.log("ไม่ได้ลบข้อมูล");
}
}
]
});
alert.present();
}
	\end{lstlisting}}
	\caption{การพัฒนาในส่วนของลบโพสท์ (ต่อ)}
	\label{Fig:4-deletepostcon1}
	\end{figure}
	
	จากรูปที่ \ref{Fig:4-deletepostcon1} โครงสร้างของไฟล์ feed.ts สามารถอธิบายการทำงานได้ดังนี้
	\begin{itemize}[label={--}]
	\item บรรทัดที่ 4 เป็นการเรียกใช้ข้อมูลในฐานข้อมูลหลังจากที่โพสท์ถูกลบเรียบร้อยแล้ว
	\item บรรทัดที่ 6 แสดงข้อผิดพลาดถ้าหากไม่สามารถลบโพสท์จากฐานข้อมูลได้
	\item บรรทัดที่ 15 - 18 เป็นปุ่มสำหรับยกเลิกการลบโพสท์
	\end{itemize}
	\newpage
	
% end ส่วนการลบโพสท์ ต่อ

% start ส่วนของการถูกใจ

\subsection{การกดถูกใจโพสท์}
เมื่อผู้ใช้เลือกปุ่มถูกใจ ระบบจะมีการทำงาน แสดงดังรูปที่ \ref{Fig:4-likepost}

\begin{figure}[H]
{\setstretch{1.0}\lstset{language=Pascal}
\begin{lstlisting}
like(post) {
let loader = this.loadingCtrl.create({
	spinner: 'hide',
	content: `<img src="assets/imgs/loading.svg">`
});
loader.present();
let body = {
	postId: post.id,
	userId: firebase.auth().currentUser.uid,
	action: post.data().likes && post.data().likes[firebase.auth().currentUser.uid] == true ? "unlike" : "like"
}
this.http.post("https://us-central1-oldmyfriends.cloudfunctions.net/updateLikesCount", JSON.stringify(body)
	, {
	responseType: "text"
	}).subscribe((data) => {
	loader.dismiss();
	console.log(data);
	}, (error) => {
	loader.dismiss();
	console.log(error);
	})
}
\end{lstlisting}}
\caption{Code การกดถูกใจโพสท์}
\label{Fig:4-likepost}
\end{figure}

จากรูปที่ \ref{Fig:4-likepost} โครงสร้างของไฟล์ feed.ts สามารถอธิบายการทำงานได้ดังนี้
\begin{itemize}[label={--}]
\item บรรทัดที่ 2 - 6 เป็นการสร้างการโหลดเพื่อรอการส่งกลับของข้อมูล ใช้งานเมื่อฟังก์ชัน like() ทำงาน
\item บรรทัดที่ 7 - 11 เป็นการกำหนดตัวแปรที่จะส่งไปแบบ post
\item บรรทัดที่ 12 - 21 เป็นการกำหนดเส้นทางเพือส่งข้อมูล JSON ไปตรวจสอบที่ฐานข้อมูล จากนั้นฐานข้อมูลจะเช็คข้อมูลการถูกใจแล้วบันทึก หรือแก้ไขลงฐานข้อมูล
\end{itemize}
\newpage

% end ส่วนของการถูกใจ

% start ส่วนของการเลือกหมวดโพสท์

\subsection{การเลือกหมวดโพสท์}
เมื่อผู้ใช้กรอกข้อความแล้วกดปุ่มคอมเมนท์ ระบบจะมีการทำงาน แสดงดังรูปที่ \ref{Fig:4-selectpost}

\begin{figure}[H]
{\setstretch{1.0}\lstset{language=Pascal}
\begin{lstlisting}
onChange($event) {
console.log($event);
if($event === "all"){
	this.getPosts();
}else {
	this.posts = [];
	let loader = this.loadingCtrl.create({
	spinner: 'hide',
	content: `<img src="assets/imgs/loading.svg">`
	}); loader.present();
	firebase.firestore().collection("posts").where("type", "==", $event).get().then((data) => {
	data.forEach((doc) => {
		loader.dismiss();
		this.posts.push(doc);
	})
	})
}
}
\end{lstlisting}}
\caption{ส่วนของการเลือกหมวดโพสท์}
\label{Fig:4-selectpost}
\end{figure}

จากรูปที่ \ref{Fig:4-selectpost} โครงสร้างของไฟล์ feed.ts สามารถอธิบายการทำงานได้ดังนี้
\begin{itemize}[label={--}]
\item บรรทัดที่ 1 ใช้งานเมื่อผู้ใช้เลือกหมวดและกดยืนยัน
\item บรรทัดที่ 3 - 4 เป็นการเช็คข้อมูลที่รับเข้ามาถ้าแสดงโพสท์ทั้งหมด จะเรียกฟังก์ชัน geetPosts() เพื่อแสดงโพสท์ทั้งหมด
\item บรรทัดที่ 6 ถ้าเลือกอย่างอื่นจะกำหนดตัวแปร posts ให้เป็นอาเรย์เปล่า
\item บรรทัดที่ 7 - 10 ตัวโหลดถูกเปิดใช้งาน
\item บรรทัดที่ 11 - 14 ค้นหาที่ฐานข้อมูลตามประเภทที่เราส่งไป แล้วปิดตัวโหลด จากนั้นเก็บตัวแปร doc ใส่ลงในตัวแปร posts
\end{itemize}
\newpage

% end ส่วนของการเลือกหมวดโพสท์

% start ส่วนของการคอมเมนท์

\subsection{การจัดการคอมเมนท์}
เมื่อผู้ใช้กรอกข้อความแล้วกดปุ่มคอมเมนท์ ระบบจะมีการทำงาน แสดงดังรูปที่ \ref{Fig:4-commentspost}

\begin{figure}[H]
{\setstretch{1.0}\lstset{language=Pascal}
\begin{lstlisting}
sendComment() {
let loader = this.loadingCtrl.create({
	spinner: 'hide',
	content: `<img src="assets/imgs/loading.svg">`
});
loader.present();
if (typeof (this.text) == "string" && this.text.length > 0) {
	firebase.firestore().collection("comments").add({
	text: this.text,
	post: this.post.id,
	owner: firebase.auth().currentUser.uid,
	owner_name: firebase.auth().currentUser.displayName,
	created: firebase.firestore.FieldValue.serverTimestamp(),
	photoUser: this.photoDisplay
	}).then((doc) => {
	console.log(doc);
	this.text = "";
	loader.dismiss();
	this.getComment();
	}).catch((err) => {
	loader.dismiss();
	console.log(err);
	})
} else {
	this.toastCtrl.create({
	message: "กรุณากรอกให้ช่องไม่ว่าง",
	duration: 2000,
	}).present();
}
}
\end{lstlisting}}
\caption{Code การจัดการคอมเมนท์}
\label{Fig:4-commentspost}
\end{figure}
\newpage

จากรูปที่ \ref{Fig:4-commentspost} โครงสร้างของไฟล์ comments.ts สามารถอธิบายการทำงานได้ดังนี้
\begin{itemize}[label={--}]
\item บรรทัดที่ 2 - 6 เป็นการเปิดใช้งานตัวโหลด
\item บรรทัดที่ 7 เช็คตัวแปร text ว่าว่างหรือไม่
\item บรรทัดที่ 8 - 14 เป็นการเก็บข้อมูลที่คอมเมนท์ไปยังฐานข้อมูล
\item บรรทัดที่ 15 - 19 ถ้าหาก text จะบันทึกข้อมูลลงฐานข้อมูล หลังจากบันทึกข้อมูลเรียบร้อยแล้ว จะกำหนดตัวแปร text เท่ากับช่องว่าง และปิดตัวโหลด จากนั้นเรียกฟังก์ชัน getComment() เพื่อรีเฟรชหน้าจอ
\item บรรทัดที่ 20 - 22 ถ้าหากไม่สามารถบันทึกข้อมูลได้ ระบบจะปิดตัวโหลดและแจ้งข้อความผิดพลาดแก่ผู้ใช้
\item บรรทัดที่ 24 - 28 ถ้าหาตัวแปร text ว่างจะแสดงข้อความว่ากรุณากรอกให้ช่องไม่ว่าง
\end{itemize}
\newpage

% end ส่วนของการคอมเมนท์



% start ส่วนของการค้นหาเพื่อน

\section{การค้นหาเพื่อน}
เมื่อผู้พิมพ์เพื่อค้นหาเพื่อน ระบบจะมีการทำงาน แสดงดังรูปที่ \ref{Fig:4-addfamily}

\begin{figure}[H]
{\setstretch{1.0}\lstset{language=Pascal}
\begin{lstlisting}
searchuser(searchbar) {
let tempfriends = this.tempmyfriends;
var q = searchbar.target.value;
if (q.trim() == "") {
	this.myfriends = this.tempmyfriends;
	return;
}
tempfriends = tempfriends.filter((v) => {
	if (v.data().owner_name.toLowerCase().indexOf(q.toLowerCase()) > -1) {
	return true;
	}
	return false;
})
this.myfriends = tempfriends;
}
\end{lstlisting}}
\caption{Code การค้นหาเพื่อน}
\label{Fig:4-addfamily}
\end{figure}
จากรูปที่ \ref{Fig:4-addfamily} โครงสร้างของไฟล์ FamilybuddysPage.ts สามารถอธิบายการทำงานได้ดังนี้
\begin{itemize}[label={--}]
\item บรรทัดที่ 1 ถูกเรียกใช้เมื่อผู้ใช้กรอกข้อมูลเพื่อนโดยจะส่งมาในตัวแปร searchbar
\item บรรทัดที่ 2 เก็บข้อมูลเพื่อนทั้งหมดไว้ในตัวแปร tempfriends
\item บรรทัดที่ 3 เก็บข้อความที่ผู้ใช้พิมพ์ไว้ที่ตัวแปร q
\item บรรทัดที่ 4 - 7 ถ้าช่องค้นหาว่างจะแสดงเพื่อนทั้งหมดใน List
\item บรรทัดที่ 8 - 12 กำหนด tempfriends เท่ากับชื่อเพื่อนอย่างน้อยหนึ่งตัว จะ return เป็น true ถ้าไม่ตรงจะ return เป็น false
\item บรรทัดที่ 14 ให้ตัวแปร myfriends เท่ากับ tempfriends คือเพื่อนที่ถูกค้นหาเจอ
\end{itemize}

% end ส่วนของการเพิ่มสมาชิกครอบครัว

% \section{การจัดการแจ้งเตือน}

% % start พัฒนาในส่วนของการแจ้งเตือนรอบบริเวณ

% \subsection{การแจ้งเตือนรอบบริเวณ}
% เมื่อผู้ใช้กดปุ่มแจ้งเตือนรอบบริเวณ ระบบจะมีการทำงาน แสดงดังรูปที่ \ref{Fig:4-danger}

% \begin{figure}[H]
% {\setstretch{1.0}\lstset{language=Pascal}
% \begin{lstlisting}
% constructor() {
% this.nativeAudio.preloadComplex('1', 'assets/sound/help.mp3',1,1,0);
% }
% sound() {
% this.nativeAudio.play('1').then(()=>{console.log('Playing')});
% }
% \end{lstlisting}}
% \caption{Code การแจ้งเตือนรอบบริเวณ}
% \label{Fig:4-danger}
% \end{figure}

% จากรูปที่ \ref{Fig:4-danger} โครงสร้างของไฟล์ HelpPage.ts สามารถอธิบายการทำงานได้ดังนี้
% \begin{itemize}[label={--}]
% \item บรรทัดที่ 2 เรียกใช้ nativeaudio plugin เป็นการโหลดเสียงสำหรับรอเปิดการใช้งานเสียง จะถูกเรียกเมื่อหน้าฉุกเฉินถูกเรียกใช้งาน
% \item บรรทัดที่ 5 เรียกใช้ nativeAudio plugin เพื่อเล่นเสียง
% \end{itemize}
% \newline

% % end พัฒนาในส่วนของการแจ้งเตือนรอบบริเวณ

% % start พัฒนาใส่ของการเพิ่มแจ้งเตือนทานยา

\section{การเพิ่มการแจ้งเตือนการทานยา}
เมื่อผู้ใช้กรอกข้อมูลยาที่ต้องการแจ้งเตือนแล้วกดปุ่มบันทึกการแจ้งเตือน ระบบจะมีการทำงาน แสดงดังรูปที่ \ref{Fig:4-alarm}

\begin{figure}[H]
{\setstretch{1.0}\lstset{language=Pascal}
\begin{lstlisting}
savemedicine() {
if(this.selecttype == "eat"){
	this.type = "สำหรับรับประทาน"
}else {
	this.type = "สำหรับฉีด"
}
if(this.name == undefined || this.number == undefined){
	this.toastCtrl.create({
	message: "กรุณากรอกข้อมูลให้ถูกต้อง",
	duration: 2000,
	position: 'top',
	}).present();
}else{
firebase.firestore().collection("medicine").add({
	uid: firebase.auth().currentUser.uid,
	name: this.name,
	number: this.number,
	time: this.bangkokTime,
	type: this.selecttype,
	timestamp: firebase.firestore.FieldValue.serverTimestamp()
}).then(() => {
	this.localNotifications.schedule({
	title: "ชื่อยา : " + this.name,
	text: "ประเภท : " + this.type + ", จำนวน : " + this.number,
	trigger: {at: new Date(this.bangkokTime)},
	led: 'FF0000',
	sound: this.setSound()
	});

	this.toastCtrl.create({
	message: "บันทึกแจ้งเตือนสำเร็จ",
	duration: 2000,
	position: 'top',
	}).present();
	this.viewCtrl.dismiss();
})
}
}
\end{lstlisting}}
\caption{Code การเพิ่มการแจ้งเตือนการทานยา}
\label{Fig:4-alarm}
\end{figure}
\newpage

จากรูปที่ \ref{Fig:4-alarm} โครงสร้างของไฟล์ SettingAlarmPage.ts สามารถอธิบายการทำงานได้ดังนี้
\begin{itemize}[label={--}]
\item บรรทัดที่ 2 - 5 เช็คประเภทข้อมูลว่าเป็นสำหรับรับประทาน หรือสำหรับฉีด จากนั้นเก็บไว้ในตัวแปร type
\item บรรทัดที่ 7 - 12 ถ้าหากชื่อยาหรือจำนวนยาไม่ได้ระบุจะแจ้งเตือนข้อความ กรุณากรอกข้อมูลให้ถูกต้อง
\item บรรทัดที่ 14 - 20 เป็นการบันทึกข้อมูลการทานยาลงฐานข้อมูล
\item บรรทัดที่ 22 - 27 หลังจากบันทึกข้อมูลเสร็จแล้ว ระบบจะเรียกใช้ localNotifications plugin เพื่อเก็บข้อมูลสำหรับการแจ้งเตือนตามเวลา
\item บรรทัดที่ 30 - 35 หลังจากบันทึกข้อมูลเรียบร้อยแล้ว จะแสดงข้อความว่าบันทึกแจ้งเตือนสำเร็จ และกลับไปยังหน้าจอแสดงรายการแจ้งเตือนทั้งหมด
\end{itemize}
\newpage

% end พัฒนาใส่ของการเพิ่มแจ้งเตือนทานยา


% \section{การจัดการครอบครัว}

% % start ส่วนของการแสดงผลครอบครัว

% \subsection{ส่วนของการแสดงผลครอบครัว}
% เมื่อผู้ใช้เรียกใช้งานหน้าครอบครัว ระบบจะมีการทำงาน แสดงดังรูปที่ \ref{Fig:4-family}

% \begin{figure}[H]
% {\setstretch{1.0}\lstset{language=Pascal}
% \begin{lstlisting}
% getfamilys() {
% this.familyservice.getmyfamilys().then((data) => {
% 	this.myfamilys = data;
% })
% }
% \end{lstlisting}}
% \caption{Code ส่วนการดึงข้อมูลครอบครัวมาแสดง}
% \label{Fig:4-family}
% \end{figure}

% จากรูปที่ \ref{Fig:4-family} โครงสร้างของไฟล์ family.ts สามารถอธิบายการทำงานได้ดังนี้
% \begin{itemize}[label={--}]
% \item บรรทัดที่ 2 เป็นการเรียกใช้ฟังก์ชัน getmyfamilys() ในคลาส FamilyProvider
% \item บรรทัดที่ 3 หลังจากทำฟังก์ชันเสร็จเรียบร้อยแล้วจะส่งข้อมูลคืนมาไว้ในตัวแปร data จากนั้นเก็บข้อมูลครอบครัวไว้ในตัวแปร myfalilys
% \end{itemize}
% \newpage
% % html familys

% family.html เป็นไฟล์ที่ใช้แสดงหน้าแสดงครอบครัว ดังรูปที่ \ref{Fig:4-showfamily}
% \begin{figure}[H]
% {\setstretch{1.0}\lstset{language=Pascal}
% \begin{lstlisting}
% <ion-content padding>
% <button ion-button block round color="headertop" (click)="familybuddys()">เพิ่มสมาชิกครอบครัว</button>
% <br>
% <ion-list no-lines>
% <ion-card style="border-radius: 10px">
% <ion-item *ngFor="let key of myfamilys" >

% <ion-thumbnail item-start *ngIf="key.data().photoURL">
% 	<img src="{{key.data().photoURL}}">
% </ion-thumbnail>
% <ion-thumbnail item-start *ngIf="!key.data().photoURL">
% 	<img src="assets/imgs/user.png">
% </ion-thumbnail>
% <ion-label item-start>{{key.data().owner_name}}</ion-label>

% 	<ion-label item-center>{{key.data().type}}</ion-label>

% <ion-icon name="settings" item-right (click)="settings(key)"></ion-icon>
% </ion-item>
% </ion-card>
% </ion-list>
% </ion-content>
% \end{lstlisting}}
% \caption{โค้ดส่วนการแสดงหน้าแสดงครอบครัว}
% \label{Fig:4-showfamily}
% \end{figure}
% \newpage

% % end ส่วนของการแสดงผลครอบครัว

% % start พัฒนาในส่วนของการโทรหาสมาชิกในครอบครัว

% \subsection{การโทรหาสมาชิกในครอบครัว}
% เมื่อผู้ใช้เลือกปุ่มโทรหาสมาชิกในครอบครัว ระบบจะมีการทำงาน แสดงดังรูปที่ \ref{Fig:4-call}

% \begin{figure}[H]
% {\setstretch{1.0}\lstset{language=Pascal}
% \begin{lstlisting}
% call(key){
% if(key.data().phone != undefined){
% this.callNumber.callNumber(key.data().phone, true).then(() => {
% 	console.log("Worked");
% }).catch((err) => {
% 	alert(JSON.stringify(err));
% })
% }else {
% this.toast.create({
% 	message: "ไม่สามารถโทรได้สมาชิกไม่ได้ระบุเบอร์ไว้",
% 	duration: 2000
% }).present();
% }
% }
% \end{lstlisting}}
% \caption{Code การโทรหาสมาชิกในครอบครัว}
% \label{Fig:4-call}
% \end{figure}

% จากรูปที่ \ref{Fig:4-call} โครงสร้างของไฟล์ HelpfamilyPage.ts สามารถอธิบายการทำงานได้ดังนี้
% \begin{itemize}[label={--}]
% \item บรรทัดที่ 2 เป็นการเช็คว่าข้อมูลที่รับเข้ามามีข้อมูลมือถือหรือไม่
% \item บรรทัดที่ 3 เรียกใช้ callNumber plugin กำหนดเบอร์มือถือและอนุญาติการใช้งานการโทร
% \item บรรทัดที่ 4 เมื่อการโทรสามารถใช้งานได้จะแจ้งข้อความว่า Worked
% \item บรรทัดที่ 6 ถ้าไม่สามารถใช้งานได้จะแสดง alert ข้อผิดพลาด
% \item บรรทัดที่ 9 - 12 ถ้าหากข้อมูลที่รับเข้ามาไม่มีเบอร์มือถือระบบจะแจ้งว่าไม่สามารถทดรได้สามาชิกไม่ได้ระบุเบอร์ไว้
% \end{itemize}
% \newpage
% % end พัฒนาในส่วนของการโทรหาสมาชิกในครอบครัว

% % start พัฒนาในส่วนของการส่งข้อความหาสมาชิกในครอบครัว

% \subsection{การส่งข้อความหาสมาชิกในครอบครัว}
% เมื่อผู้ใช้เลือกปุ่มส่งข้อความหาสมาชิกในครอบครัว ระบบจะมีการทำงาน แสดงดังรูปที่ \ref{Fig:4-sms}

% \begin{figure}[H]
% {\setstretch{1.0}\lstset{language=Pascal}
% \begin{lstlisting}
% sendSMS(key){
% firebase.firestore().collection("informationUser").doc(firebase.auth().currentUser.uid).get().then((res) => {
% 	let message = "ช่วยด้วย !!! นี่ " + key.data().owner_name + " เอง ตอนนี้มีปัญหาช่วยติดต่อกลับมาที่ " + res.data().phone + " ด้วยนะ ด่วน ๆ"   
% this.socialSharing.shareViaSMS(message,key.data().phone).then(() => {
% 	console.log("SMS Success");
% }).catch((err) => {
% 	console.log(err);
% })
% }).catch((err) => {
% 	console.log(err);
% })
% }
% \end{lstlisting}}
% \caption{Code การส่งข้อความหาสมาชิกในครอบครัว}
% \label{Fig:4-sms}
% \end{figure}

% จากรูปที่ \ref{Fig:4-sms} โครงสร้างของไฟล์ HelpfamilyPage.ts สามารถอธิบายการทำงานได้ดังนี้
% \begin{itemize}[label={--}]
% \item บรรทัดที่ 2 เป็นการเรียกใช้ข้อมูลผู้ใช้จากฐานข้อมูลเก็บไว้ในตัวแปร res
% \item บรรทัดที่ 3 กำหนดตัวแปร message เก็บข้อความที่จะส่ง
% \item บรรทัดที่ 4 - 5 เป็นการเรียกใช้ socialSharing plugin เพื่อส่งข้อความที่กำหนดไว้ใน message
% \item บรรทัดที่ 6 - 11 แจ้งข้อความถ้าหากเกิดข้อผิดพลาด
% \end{itemize}
% \newpage

% % end พัฒนาในส่วนของการส่งข้อความหาสมาชิกในครอบครัว

% % start ส่วนของการแสดงตำแหน่งสมาชิกในครอบครัว

% \subsection{การแสดงตำแหน่งสมาชิกในครอบครัว}
% เมื่อผู้ใช้เข้าเมนูตำแหน่งครอบครัว ระบบจะมีการทำงาน แสดงดังรูปที่ \ref{Fig:4-googlemap}

% \begin{figure}[H]
% {\setstretch{1.0}\lstset{language=Pascal}
% \begin{lstlisting}
% showMap() {
% firebase.firestore().collection("informationUser").doc(firebase.auth().currentUser.uid).get().then( (res) => {
% const locationme = new google.maps.LatLng(res.data().location.latitude, res.data().location.longitude);
% const options = {
% center: locationme,
% zoom: 14,
% streetViewControl: false,
% mapTypeId: google.maps.MapTypeId.ROADMAP
% };
% this.map = new google.maps.Map(this.mapRef.nativeElement, options);

% var marker = new google.maps.Marker({
% position: locationme,
% map: this.map,
% draggable: false,
% animation: google.maps.Animation.DROP,
% title: res.data().owner_name,
% icon: {
% 	url: "http://maps.google.com/mapfiles/ms/icons/pink-dot.png",
% },
% })
% var infowindow = new google.maps.InfoWindow({
% content: res.data().owner_name,
% maxWidth: 200
% });
% infowindow.open(this.map,marker);
% \end{lstlisting}}
% \caption{Code การแสดงตำแหน่งสมาชิกในครอบครัว}
% \label{Fig:4-googlemap}
% \end{figure}
% \newpage

% จากรูปที่ \ref{Fig:4-googlemap} โครงสร้างของไฟล์ GoogleMap.ts สามารถอธิบายการทำงานได้ดังนี้
% \begin{itemize}[label={--}]
% \item บรรทัดที่ 2 เป็นการค้นหาข้อมูลผู้ใช้ที่อยู่ในฐานข้อมูลไว้ในตัวแปร res
% \item บรรทัดที่ 3 เป็นการกำหนด latitude และ longitude ของผู้ใช้งานในตัวแปร locationme
% \item บรรทัดที่ 5 กำหนดจุดศูนย์กลางของหน้าจอ Google Maps หลังจากผู้ใช้เรียกใช้งานหน้าจอแผนที่
% \item บรรทัดที่ 6 กำหนดขนาดหน้าจอที่จะแสดงในแผนที่
% \item บรรทัดที่ 8 เป็นการกำหนดประเภทของแผนที่
% \item บรรทัดที่ 10 กำหนดให้ตัวแปร map เก็บข้อมูลของแผนที่ที่จะแสดง
% \item บรรทัดที่ 12 - 19 เป็นการกำหนดตำแหน่ง marker ของผู้ใช้
% \item บรรทัดที่ 22 - 25 กำหนดตัวแปร infowindow เพื่อเรียกใช้รายละเอียดของ marker
% \item บรรทัดที่ 26 เป็นคำสั่งในการเปิดใช้งาน infowindow
% \end{itemize}
% \newpage

% % end ส่วนของการแสดงตำแหน่งสมาชิกในครอบครัว

% % start ส่วนของการแสดงตแหน่งสมาชิกในครอบครัว (ต่อ)

% เมื่อผู้ใช้เข้าเมนูตำแหน่งครอบครัว (ต่อ) ระบบจะมีการทำงาน แสดงดังรูปที่ \ref{Fig:4-googlemapcon1}

% \begin{figure}[H]
% {\setstretch{1.0}\lstset{language=Pascal}
% \begin{lstlisting}
% this.familyservice.getinfofamilys();
% this.events.subscribe('infofamilys', () => {
% this.allfamilys;
% this.zone.run(() => {
% this.allfamilys = this.familyservice.infofamilys;
% for(var key in this.allfamilys){
% if(this.allfamilys[key].location){
% 	this.allresfamilys = this.allfamilys[key];
% 	console.log(this.allresfamilys.location);
% 	var marker = new google.maps.Marker({
% 	position: new google.maps.LatLng(this.allresfamilys.location.latitude,this.allresfamilys.location.longitude),
% 	map: this.map,
% 	draggable: false,
% 	animation: google.maps.Animation.DROP,
% 	title: this.allresfamilys.owner_name,
% 	icon: {
% 		url: "http://maps.google.com/mapfiles/ms/icons/blue-dot.png",
% 	},
% 	})
% 	var infowindow = new google.maps.InfoWindow({
% 	content: this.allresfamilys.owner_name,
% 	maxWidth: 200
% 	});
% 	infowindow.open(this.map,marker);
% }
% }
% })
% })
% }) 
% }
% \end{lstlisting}}
% \caption{Code การแสดงตำแหน่งสมาชิกในครอบครัว (ต่อ)}
% \label{Fig:4-googlemapcon1}
% \end{figure}
% \newpage

% จากรูปที่ \ref{Fig:4-googlemapcon1} โครงสร้างของไฟล์ GoogleMap.ts สามารถอธิบายการทำงานได้ดังนี้
% \begin{itemize}[label={--}]
% \item บรรทัดที่ 1 เรียกใช้ฟังก์ชัน getinfofamilys() ในคลาส FamilyProvider
% \item บรรทัดที่ 2 - 5 เรียกใช้ตัวแปรที่อยู่ในฟังก์ชัน getinfofamilys() เก็บไว้ในตัวแปร allfamilys
% \item บรรทัดที่ 6 วนลูปตัวแปร allfamilys
% \item บรรทัดที่ 7 เช็คตัวแปร allfamilys ว่ามี location หรือไม่
% \item บรรทัดที่ 8 เก็บข้อมูลที่มีตำแหน่งไว้ในตัวแปร allresfamilys
% \item บรรทัดที่ 10 - 11 นำข้อมูลที่อยู่ในตัวแปร allresfamilys มาสร้าง marker
% \item บรรทัดที่ 20 - 24 เป็นการสร้างรายละเอียดของ marker แต่ละอัน
% \end{itemize}
% \newpage

% % end ส่วนของการแสดงตำแหน่งสมาชิกในครอบครัว (ต่อ)

