\chapter{การพัฒนาระบบ}
หลังจากที่ได้มีการเตรียมความพร้อมสำหรับการพัฒนาในด้านต่าง ไม่ว่าจะเป็นที่มาและความสำคัญของปัญหา เทคโนโลยีที่มีความเหมาะสมกับระบบ และการออกแบบระบบการทำงานรวมไปถึงโครงสร้างของข้อมูล ในบทนี้จะเป็นการพูดถึงการสร้างระบบที่ได้มีการออกแบบไว้ในบทที่แล้วจะถูกนำเสนอในบทนี้ โดยการพัฒนาระบบแบ่งได้เป็นส่วนต่าง ๆ ดังนี้
	\begin{enumerate}[label=4.\arabic*]
		\item การพัฒนาเว็บแอปพลิเคชัน
		\item การพัฒนาเว็บแอปพลิเคชันแอนดรอยด์แอปพลิเคชัน
	\end{enumerate}	

\section{การพัฒนาเว็บแอปพลิเคชัน}
	การพัฒนาระบบกองทุนเงินให้กู้ยืมเพื่อการศึกษาสำหรับเว็บแอปพลิเคชันนั้นวัตถุประสงค์หลังเพื่อสร้างความสะดวกต่อการกำงานของเจ้าหน้าที่อันเนื่องมาจากข้อจำกัดบางประการหากใช้ระบบทำงานบนอุปกรณ์สมาร์ทโฟนเพียงอย่างเดียว โดยตัวเว็บแอปพลิเคชันนี้ถูกพัฒนาขึ้นด้วย Vue.js มีรายละเอียดการทำงานดังนี้
	
	\subsection{การเชื่อมต่อ Cloud Firestore}bsection,
	ในการเชื่อมต่อเว็บแอปพลิเคชันกับไฟร์เบสเพื่อใช้บริการต่างๆ ของไฟร์เบส ทำได้ดังนี้
	\begin{figure}[H]
		{\setstretch{1.0}\begin{lstlisting}
export default {
	apiKey: "XXXXXXXXXXXXXXXXXXXXXXXXXXXXXXXXXXXXXXX",
	authDomain: "e-student-698a5.firebaseapp.com",
	databaseURL: "https://e-student-698a5.firebaseio.com",
	projectId: "e-student-698a5",
	storageBucket: "e-student-698a5.appspot.com",
	messagingSenderId: "000000000000"
}
			\end{lstlisting}}
		\caption{ไฟล์ firebaseConfig.js}
		\label{Fig:firebaseConfig}
	\end{figure}
	จากภาพที่ \ref{Fig:firebaseConfig} โครงสร้างของไฟล์ firebaseConfig.js สามารถอธิบายการทำงานได้ดังนี้
	\begin{itemize}[label={--}]
		\item บรรทัดที่  1	     เป็นการส่งออกโมดูลเพื่อใช้งานในไฟล์อื่น
		\item บรรทัดที่  2 -7	เป็นการตั้งค่าระบุตัวตนเพื่อใช้งานบริการไฟร์เบส
	\end{itemize}
	
	\begin{figure}[H]
		{\setstretch{1.0}\begin{lstlisting}
import firebase from 'firebase'
import 'firebase/firestore'
import firebaseConfig from './firebaseConfig'

export default firebase.initializeApp(firebaseConfig)
			\end{lstlisting}}
		\caption{ไฟล์ firebaseInit.js}
		\label{Fig:firebaseConfig}
	\end{figure}
	จากภาพที่ \ref{Fig:firebaseConfig} โครงสร้างของไฟล์ firebaseInit.js สามารถอธิบายการทำงานได้ดังนี้
	\begin{itemize}[label={--}]
		\item บรรทัดที่  1	     เป็นการนำเข้าไลบรารีของไฟร์เบส
		\item บรรทัดที่  2 	    เป็นการนำเข้าบริการ Cloud Firestore ของไฟร์เบส 
		\item บรรทัดที่  8       เป็นการนำเข้าโมดูลตั้งค่าที่ได้จากรูปภาพที่ \ref{Fig:firebaseConfig}
		\item บรรทัดที่  5 เป็นการส่งออกโมดูลไฟร์เบสเพื่อใช้ในไฟล์อื่น ๆ ซึ่งเมื่อถึงขั้นตอนนี้การเชื่อต่อบริการไฟร์เบสถือว่าเป็นอันเสร็จ
	\end{itemize}
	
	\subsection{โครงสร้างของการสร้างหน้าเข้าสู่ระบบ}
	\begin{figure}[H]
	{\setstretch{1.0}\begin{lstlisting}
<template>
	<Row type="flex" justify="center" align="middle">
	<Col span="8" class="col">
	<Card style="width:400px">
	<p slot="title">
	<Icon type="ios-person" size="20"></Icon>
	sign in
	</p>
	<a href="#" slot="extra" @click.prevent="SignUp">
	create account
	</a>
	<Form class="form" ref="formInline" :model="formInline" :rules="ruleInline">
	<FormItem prop="email">
		<Input type="text" v-model="formInline.email" placeholder="email">
		<Icon type="ios-email" slot="prepend"></Icon>
		</Input>
	</FormItem>
	<FormItem prop="password">
		<Input type="password" v-model="formInline.password" placeholder="password">
		<Icon type="ios-locked" slot="prepend"></Icon>
		</Input>
	</FormItem>
	<FormItem>
	<Button type="primary" :loading="loading" @click="handleSubmit('formInline')">
	<span v-if="!loading">sign in</span>
	<span v-else>signing in...</span>
	</Button>
	</FormItem>
	</Form>         
	</Card>
	</Col>
	</Row>
</template>
		\end{lstlisting}}
		\caption{การสร้างหน้าจอส่วนติดต่อผู้ใช้ของหน้าเข้าสู่ระบบ SignIn.vue}
		\label{Fig:SignIn}
	\end{figure}
	
	จากภาพที่ \ref{Fig:SignIn} โครงสร้างของการสร้างหน้าจอส่วนติดต่อผู้ใช้ของหน้าเข้าสู่ระบบ สามารถอธิบายการทำงานได้ดังนี้
	\begin{itemize}[label={--}]
		\item บรรทัดที่ 1-33  เป็นเทมเพลตที่ใช้เพื่อสี่อสารกับ Vue.js ให้แปลงข้อมูลดังกล่าวเป็น HTML
		\item บรรทัดที่ 2-32 เป็นการความคุมลักษณะการแสดงผลบนหน้าจอ
		\item บรรทัดที่ 3-31 เป็นการกำหนดขนาดของเนื้อภายใน
		\item บรรทัดที่ 4-30 เป็นการแสดงเนื้อหาในรูปแบบการ์ด (Card)
		\item บรรทัดที่ 5-8 เป็นส่วนที่ใช้สำหรับกำหนดหัวเรื่องของการ์ด
		\item บรรทัดที่ 12 เป็นสร้างฟอร์ม (Form)
		\item บรรทัดที่ 13 เป็นสร้างช่องกรอกข้อมูลอีเมล (e-mail) จากผู้ใช้
		\item บรรทัดที่ 18 เป็นสร้างช่องกรอกข้อมูลรหัสผ่าน (password) จากผู้ใช้
		\item บรรทัดที่ 24 สร้างปุ่มเข้าสู่ระบบ
	\end{itemize}

	\begin{figure}[H]
		{\setstretch{1.0}\begin{lstlisting}
data () {
 return {
  alert: false,
  formInline: {
   email: '',
   password: ''
  },
  ruleInline: {
   email: [
    { required: true, message: 'please fill email', trigger: 'blur' }
   ],
   password: [
    { required: true, message: 'please fill password', trigger: 'blur' }
   ]
  }
 }
}
			\end{lstlisting}}
		\caption{การสร้างหน้าจอส่วนติดต่อผู้ใช้ของหน้าเข้าสู่ระบบ SignIn.vue}
		\label{Fig:SignIn}
	\end{figure}
	จากภาพที่ \ref{Fig:SignIn} โครงสร้างของการสร้างหน้าจอส่วนติดต่อผู้ใช้ของหน้าเข้าสู่ระบบ สามารถอธิบายการทำงานได้ดังนี้
	\begin{itemize}[label={--}]
		\item บรรทัดที่ 1-7  เป็นการสร้างชุดข้อมูลที่ใช้สำหรับการเข้าสู่ระบบ
		\item บรรทัดที่ 3 ค่าที่ใช้เก็บสถานะของการเข้าสู่ระบบ
		\item บรรทัดที่ 4-7 เป็นการเก็บข้อมูลอีเมลและรหัสผ่านในรูปแบบ json
		\item บรรทัดที่ 8-15 เป็นการกำหนดกที่ใช้ในการตรวจสอบความถูกต้องของอีเมลและรหัสผ่าน
	\end{itemize}
	
	\begin{figure}[H]
		{\setstretch{1.0}\begin{lstlisting}
userSignIn({commit}, payload) {
 commit('setLoading', true)
  firebase.auth().signInWithEmailAndPassword(payload.email, payload.password)
  .then(firebaseUser => {
 commit('setUser', firebaseUser)
 commit('setLoading', false)
 commit('setError', null)
})
.catch(error => {
 commit('setError', error.message)
 commit('setLoading', false)
})
}
			\end{lstlisting}}
		\caption{การสร้างลอจิก (logic) ของหน้าเข้าสู่ระบบ SignIn.vue}
		\label{Fig:SignInJs}
	\end{figure}
	จากภาพที่ \ref{Fig:SignInJs} โครงสร้างลอจิกของหน้าเข้าสู่ระบบ สามารถอธิบายการทำงานได้ดังนี้
	\begin{itemize}[label={--}]
		\item บรรทัดที่ 1  เป็นการสร้างฟังก์ชันสำหรับรับข้อมูลที่ในการเข้าสู่ระบบ
		\item บรรทัดที่ 2 เรียกใช้ฟังก์ชันอื่นเพื่อทำการอัพเดทสถานะการเข้าสู่ระบบ
		\item บรรทัดที่ 3-12 เป็นการเรียกใช้บริการไฟร์เบส Authentication พร้อมส่งค่า email และ password เพื่อทำการเข้าสู่ระบบ
		\item บรรทัดที่ 5-7 เป็นการอัพเดทสถานะเมื่อเข้าสู่ระบบสำเร็จ
		\item บรรทัดที่ 9-12 เป็นการอัพเดทสถานะเมื่อเข้าสู่ระบบไม่สำเร็จ 
	\end{itemize}
	
	\subsection{โครงสร้างของการสร้างหน้าข่าวสาร}
	\begin{figure}[H]
		{\setstretch{1.0}\begin{lstlisting}
<template>
<div style="padding: 16px;">
<Row>
<Col span="20" style="padding:16px;">
<Row v-for="(post,index) in postsData" :key="post.id" style="margin-bottom:16px;"> 
<Card>
<p slot="title">
<Icon type="social-rss-outline"></Icon>
post 
<span style="font-size:11px; color: #95a5a6;"> 
{{ post.time }}
</span>
</p>
<a href="#" slot="extra" @click.prevent="showData(index)">
<!-- <Icon type="ios-loop-strong"></Icon> -->
detail
</a>
<p>
{{ post.title }}
</p>
</Card>           
</Row>
</Col>
<Col span="4" style="padding:16px;">
<Timeline>
<TimelineItem v-for="(event, index) in eventsData" :key="index" >
<Icon type="trophy" slot="dot" v-if="index == 0"></Icon>
<p class="time">{{event.time}}</p>
<p class="content">{{event.title}}</p>
</TimelineItem>
</Timeline>
</Col>
</Row>
</div>
</template>
			\end{lstlisting}}
		\caption{การสร้างหน้าจอส่วนติดต่อผู้ใช้ของหน้าข่าวสาร Home.vue}
		\label{Fig:Home}
	\end{figure}
	จากภาพที่ \ref{Fig:Home} โครงสร้างของการสร้างหน้าจอส่วนติดต่อผู้ใช้ของหน้าข่าวสาร  สามารถอธิบายการทำงานได้ดังนี้
	\begin{itemize}[label={--}]
		\item บรรทัดที่ 1-33  เป็นเทมเพลตที่ใช้เพื่อสี่อสารกับ Vue.js ให้แปลงข้อมูลดังกล่าวเป็น HTML
		\item บรรทัดที่ 3-33 	    เป็นการความคุมลักษณะการแสดงผลบนหน้าจอ
		\item บรรทัดที่ 6-21 เป็นการแสดงเนื้อหาในรูปแบบการ์ด (Card)
		\item บรรทัดที่ 10-12 เป็นการแสดงเวลาที่ประกาศข่าว
		\item บรรทัดที่ 18-20 เป็นการแสดงหัวข้อข่าวสาร
		\item บรรทัดที่ 25-31 เป็นการแสดงปฏฺทินกำหนดการขนาดย่อ
	\end{itemize}
	
		\begin{figure}[H]
			{\setstretch{1.0}\begin{lstlisting}
created() {
 var vm = this;
 vm.postsData = [];
 db.collection("Posts")
  .orderBy("time", "desc")
  .get()
  .then(function(querySnapshot) {
   querySnapshot.forEach(function(doc) {
   const data = {
    id: doc.id,
    title: doc.data().title,
    description: doc.data().description,
    time: doc.data().time.toLocaleString(),
    fileUrl: doc.data().fileURL[0]
   }
  if(vm.postsData){
   vm.postsData.push(data);
  }
 })
})
}
				\end{lstlisting}}
			\caption{การสร้างลอจิก(logic)ของหน้าข่าวสาร Home.vue}
			\label{Fig:HomeJs}
		\end{figure}
		จากภาพที่ \ref{Fig:HomeJs} โครงสร้างลอจิกของหน้าข่าวสาร สามารถอธิบายการทำงานได้ดังนี้
		\begin{itemize}[label={--}]
			\item บรรทัดที่ 1  เป็นการฟังก์ชันที่ถูกเรียกทุกครั้งที่ผู้ใช้เปิดหน้าข่าวสาร
			\item บรรทัดที่ 4-20 เรียกใช้บริการ Cloude Firestore เพื่อทำการสืบค้นข้อมูลข่าวสารทั้งหมดพร้อมทั้งเรียงลำดับตามวันที่ประกาศ
			\item บรรทัดที่ 9-18 เป็นการเพิ่มข้อมูลเข้าสู่ลิสต์รายการเพื่อใช้ในการแสดงบนหน้าจอ
		\end{itemize}
		
	\subsection{โครงสร้างของการสร้างหน้าดูรายละเอียดข่าวสาร}
	\begin{figure}[H]
		{\setstretch{1.0}\begin{lstlisting}
<template>
 <div>
  <h2>{{ post.title }}</h2>
  <p style="font-size:14px;">{{ post.description }}</p>
  <br>
  <img :src="post.fileURL"/>
  </div>
</template>
				\end{lstlisting}}
			\caption{การสร้างหน้าจอส่วนติดต่อผู้ใช้ของหน้ารายละเอียดข่าวสาร ViewPost.vue}
			\label{Fig:ViewPost}
		\end{figure}
			จากภาพที่ \ref{Fig:ViewPost} โครงสร้างของการสร้างหน้าจอส่วนติดต่อผู้ใช้ของหน้าดูรายละเอียดข่าวสาร  สามารถอธิบายการทำงานได้ดังนี้
			\begin{itemize}[label={--}]
				\item บรรทัดที่ 1-8  เป็นเทมเพลตที่ใช้เพื่อสี่อสารกับ Vue.js ให้แปลงข้อมูลดังกล่าวเป็น HTML
				\item บรรทัดที่ 2-7	 ครอบทับเนื้อหาทั้งหมดเพื่อให้ง่ายต่อการจัดการการแสดงผล
				\item บรรทัดที่ 3 เป็นการแสดงหัวข้อข่าวสาร
				\item บรรทัดที่ 4 เป็นการแสดงรายละเอียดข่าวสาร
				\item บรรทัดที่ 6 เป็นการแสดงไฟล์แนบ
			\end{itemize}
			
			\begin{figure}[H]
				{\setstretch{1.0}\begin{lstlisting}
created(){
 let id = this.$route.params.id
 db.collection('Posts').doc(id).get().then((doc) => {
  if(doc.exists){
   this.post = doc.data()
  }
 })
}
					\end{lstlisting}}
				\caption{การสร้างลอจิกของหน้าดูรายละเอียดของข่าวสาร ViewPost.vue}
				\label{Fig:HomeJs}
			\end{figure}
			จากภาพที่ \ref{Fig:HomeJs} โครงสร้างลอจิกของหน้าดูรายละเอียดของข่าวสาร สามารถอธิบายการทำงานได้ดังนี้
			\begin{itemize}[label={--}]
				\item บรรทัดที่ 1  เป็นการฟังก์ชันที่ถูกเรียกทุกครั้งที่ผู้ใช้เปิดหน้าดูรายละเอียดข่าวสาร
				\item บรรทัดที่ 2 ดึงค่าไอดีของประกาศที่ถูกส่งมาจากหน้าแสดงข่าวสาร
				\item บรรทัดที่ 3 ทำการสืบค้นข้อมูลข่าวสารจาก Cloud Firestore จากไอดีของประกาศ
				\item บรรทัดที่ 4-6 เป็นการตรวจสอบว่ามีประกาศดังกล่าวอยู่ในฐานข้อมูล Cloude Firestore หรือไม่
			\end{itemize}
			
	\subsection{โครงสร้างของการสร้างหน้าสนทนา}
	\begin{figure}[H]
		{\setstretch{1.0}\begin{lstlisting}
<template>
<Row :gutter="0" type="flex" justify="center" align="middle">
<Col span="16" style="padding: 0px;">
<Card style="min-height: 500px;max-height: 500px;" :padding="0">
<p v-if="!isAdmin" slot="title" style="text-align:center;">
ESP
</p>
<p v-else slot="title" style="text-align:center;">
{{ chatTitle }}
</p>
<Scroll style="background-color: #EEEEEE;">
<ul style="padding:6px;padding-right:8px;">
<li v-for="(item,index) in messages" :key="index" style="margin-bottom:8px;">
<Card v-if="user.uid !== item.id" :padding="6" style="text-align:left;display: inline-block;background-color: #FAFAFA;">
<div>
<p>{{ item.message }}</p> 
</div>
</Card>
<div  v-else  style="text-align:right;">
<Card :padding="6" style="display: inline-block;background-color: #B2E281;">
<p>{{ item.message }}</p>
</Card>
</div>
</li>
</ul>
</Scroll>
<div style="padding:16px;text-align:right;" >
<Input type="textarea" v-model="formItem.message" placeholder="type..." v-on:keyup.enter="send"></Input>
<Button :loading="loading" type="primary" style="margin-top:10px;" size="large" @click="send">send</Button>
</div>
</Card>
</Col>
</Row>
</template>
			\end{lstlisting}}
		\caption{การสร้างหน้าจอส่วนติดต่อผู้ใช้ของหน้าสนทนา Message.vue}
		\label{Fig:Message}
	\end{figure}
	จากภาพที่ \ref{Fig:Message} โครงสร้างของการสร้างหน้าจอส่วนติดต่อผู้ใช้ของหน้าสนทนา สามารถอธิบายการทำงานได้ดังนี้
	\begin{itemize}[label={--}]
		\item บรรทัดที่ 1-34  เป็นเทมเพลตที่ใช้เพื่อสี่อสารกับ Vue.js ให้แปลงข้อมูลดังกล่าวเป็น HTML
		\item บรรทัดที่ 4-22 แสดงหน้าต่างสนทนา
		\item บรรทัดที่ 9 แสดงชื่อคู่สนทนา
		\item บรรทัดที่ 11-26 แสดงข้อความสนทนา
		\item บรรทัดที่ 28 แสดงช่องกรอกข้อความสนทนา
		\item บรรทัดที่ 29 แสดงปุ่มกดส่งข้อความ
	\end{itemize}
	\begin{figure}[H]
		{\setstretch{1.0}\begin{lstlisting}
send() {
 var vm = this;
 vm.loading = true;
 this.formItem.time = new Date();

// check where data shulde update
 let key = "";
 if (this.isAdmin) {
  key = this.chatId;
   if (this.chatTitle === "" || this.chatTitle === null) {
    return;
   }
 } else {
   key = this.formItem.senderId;
 }

 this.formItem.name = this.user.displayName;
 this.formItem.photo = this.user.photoURL

 db
 .collection("Chats")
 .doc(key)
 .collection("messages")
 .add(this.formItem)
 .then(function(docRef) {
   vm.loading = false;
   vm.formItem.message = "";
   db
   .collection("Users")
   .doc(key)
   .update({ lastChat: new Date() })
   .catch(function(error) {
    vm.$Message.error("send message fail");
    vm.loading = false;
   });
 });
}
			\end{lstlisting}}
		\caption{การสร้างลอจิกของหน้าสนทนา Message.vue}
		\label{Fig:MessageJs}
	\end{figure}
	\newpage
	จากภาพที่ \ref{Fig:MessageJs} โครงสร้างลอจิกของหน้าสนทนา สามารถอธิบายการทำงานได้ดังนี้
	\begin{itemize}[label={--}]
		\item บรรทัดที่ 1  ชื่อฟังก์ชัน
		\item บรรทัดที่ 2 ตรวจสอบไอดีของผู้ใช้
		\item บรรทัดที่ 3 ดึงค่าโปรไฟล์ผู้ใช้คนปัจจุบัน
		\item บรรทัดที่ 4-6 เขียนข้อมูลลงฐานข้อมูล Cloud Firestore โดยระบุ path ที่จะทำการจัดเก็บชุดข้อมูล
		\item บรรทัดที่ 28-35 อัพเดทข้อมูลเวลาสนทนาล่าสุดของผู้ใช้
		\item บรรทัดที่ 33 แสดงสถานะการอัพเดทข้อมูล
	\end{itemize}

	\subsection{โครงสร้างของการสร้างหน้าปฏิทินแสดงกำหนดการ}
	\begin{figure}[H]
		{\setstretch{1.0}\begin{lstlisting}
<template>
 <div>
  <Row :gutter="16">
   <Col span="6">
    <Card>
     <h3>search</h3>
     <DatePicker v-model="filterDate" format="d-M-yyyy" type="date" size="large" placeholder="select date" style="margin-top:6px;"></DatePicker> 
    </Card>
   </Col>
   <Col span="18"> 
    <h3>Schedule</h3>
    <Table border 
      :loading="dataLoading" 
      :columns="columnsName" 
      :data="eventsData"
      no-data-text="no schedule"
      style="margin-top:6px;">
    </Table> 
   </Col>
  </Row>
 </div>
</template>
			\end{lstlisting}}
		\caption{การสร้างหน้าจอส่วนติดต่อผู้ใช้ของหน้าปฏิทินกำหนดการ Schedule.vue}
		\label{Fig:Schedule}
	\end{figure}
	จากภาพที่ \ref{Fig:Schedule} โครงสร้างของการสร้างหน้าจอส่วนติดต่อผู้ใช้ของหน้าปฏิทินกำหนดการ สามารถอธิบายการทำงานได้ดังนี้
	\begin{itemize}[label={--}]
		\item บรรทัดที่ 1-22  เป็นเทมเพลตที่ใช้เพื่อสี่อสารกับ Vue.js ให้แปลงข้อมูลดังกล่าวเป็น HTML
		\item บรรทัดที่ 4-9 แสดงหน้าต่างเลือกวันที่เพื่อค้นหา
		\item บรรทัดที่ 11 แสดงชื่อตาราง
		\item บรรทัดที่ 12-18 แสดงตารางกำหนดการ
	\end{itemize}
	\begin{figure}[H]
		{\setstretch{1.0}\begin{lstlisting}
created() {
 var vm = this;
 db.collection("Events")
 .orderBy("time")
 .onSnapshot(function(querySnapshot) {
  vm.dataLoading = false;
  vm.eventsData = [];
  querySnapshot.forEach((doc) => {
   const data = {
    'id': doc.id,
    'title': doc.data().title,
    'description': doc.data().description,
    'time': `${doc.data().time.getDate()}-${doc.data().time.getMonth()}-${doc.data().time.getFullYear()}`
   }
   vm.eventsData.push(data)
  })
 })
}
			\end{lstlisting}}
		\caption{การสร้างลอจิกของหน้าปฏิทินกำหนดการ Schedule.vue}
		\label{Fig:ScheduleJs}
	\end{figure}
	จากภาพที่ \ref{Fig:ScheduleJs} โครงสร้างลอจิกของหน้าปฏิทินกำหนดการ สามารถอธิบายการทำงานได้ดังนี้
	\begin{itemize}[label={--}]
		\item บรรทัดที่ 1-18 ชื่อฟังก์ชันที่จะถูกเรียกทุกครั้งที่หน้าปฏิทินกำหนดการถูกเปิด
		\item บรรทัดที่ 3-17 สืบค้นกำหนดการจากฐานข้อมูลโดยมีการเรียงลำดับจากวันที่ล่าสุดไปยังวันที่ก่อนหน้า
		\item บรรทัดที่ 9-15 จัดเก็บข้อมูลที่สืบค้นได้ เพื่อใช้ในการแสดงผลบนหน้าจอ
	\end{itemize}
	
		\subsection{โครงสร้างของการสร้างหน้าสร้างประกาศ}
		\begin{figure}[H]
			{\setstretch{1.0}\begin{lstlisting}
<Modal  v-model="modalNewPost"
title="add post">
 <Form :model="formItem" :label-width="80">
  <FormItem label="title">
   <Input v-model="formItem.title" placeholder="Enter something..."></Input>
  </FormItem>
  <FormItem label="detail">
   <Input v-model="formItem.description" type="textarea" :autosize="{minRows: 2,maxRows: 5}" placeholder="Enter something..."></Input>
  </FormItem>
  <FormItem label="target">
   <Select v-model="formItem.collection">
    <Option value="public">public</Option>
    <Option value="group">group</Option>
    <Option value="volunteer">volunteer</Option>
   </Select>
  </FormItem>
  <FormItem  label="contact list" v-if="tags.length > 0">
    <Tag closable color="blue" v-for="tag in tags" :key="tag" @on-close="handleClose"> {{ tag }} </Tag>
  </FormItem>
  <FormItem label="file">
   <Upload :before-upload="handleUpload"
     action="https://shielded-earth-61349.herokuapp.com/">
    <Button :type="btnAddPostType" icon="ios-cloud-upload-outline"> {{ uploadBtnTile }} </Button>
   </Upload>
  </FormItem>
 </Form>
 <div slot="footer">
  <Button type="primary" :loading="loading" @click="newPost">save</Button>
  <Button type="ghost" style="margin-left: 8px" @click="modalNewPost = !modalNewPost">cancle</Button>
 </div>
</Modal>
				\end{lstlisting}}
			\caption{การสร้างหน้าจอส่วนติดต่อผู้ใช้ของหน้าสร้างประกาศ MgPost.vue}
			\label{Fig:MgPost}
		\end{figure}
		จากภาพที่ \ref{Fig:ViewPost} โครงสร้างของการสร้างหน้าจอส่วนติดต่อผู้ใช้ของหน้าสร้างประกาศ สามารถอธิบายการทำงานได้ดังนี้
		\begin{itemize}[label={--}]
			\item บรรทัดที่ 1-31  เนื่องจากต้องการให้แสดงหน้าเพิ่มข่าวสารเป็นป๊อปอัพ (Pop up)  จึงใช้แท็ก(tag) <Model></Model>
			\item บรรทัดที่ 2-7	 ครอบทับเนื้อหาทั้งหมดเพื่อให้ง่ายต่อการจัดการการแสดงผล
			\item บรรทัดที่ 3-26 เป็นการสร้างฟอร์มรับข้อมูล
			\item บรรทัดที่ 4-25 เป็นสร้างช่องรับข้อมูลประกาศ
			\item บรรทัดที่ 5 เป็นการรับค่าหัวเรื่องประกาศ
			\item บรรทัดที่ 8 เป็นการรับค่ารายละเอียดประกาศ
			\item บรรทัดที่ 11-15 เป็นการรับค่ากลุ่มเป้าหมายของประกาศนั้นๆ
			\item บรรทัดที่ 21-24 เป็นการสร้างปุ่มอัพโหลดไฟล์
			\item บรรทัดที่ 28 เป็นการสร้างปุ่มบันทึกประกาศ
			\item บรรทัดที่ 29 เป็นการสร้างปุ่มยกเลิกประกาศ
		\end{itemize}
		\begin{figure}[H]
			{\setstretch{1.0}\begin{lstlisting}
newPost(){ 
 let key = '';
 var vm = this;
 vm.formItem.time = new Date();
 vm.formItem.tags = vm.tags
 db.collection("Posts").add(this.formItem)
  .then(function(docRef) {
   key = docRef.id;
   if(vm.file != null){
    let storageRef = storage.ref('Posts/'+key);
    let fileRef = storageRef.child(vm.file.name+"");
    fileRef.put(vm.file).then(function(snapshot) {
    db.collection("Posts").doc(key)
   .update(
    {
     fileURL: snapshot.metadata.downloadURLs[0],
     fileType: vm.file.name.slice(vm.file.name.lastIndexOf('.')) + "",
     fileName: vm.file.name
   }
 ).then(() => {
   vm.$Notice.success({
     title: 'Your post was created',
     desc: ''
   });
})
.catch(function(error) {
  vm.$Notice.warning({
  title: 'create post fail',
  desc:''
});
}
				\end{lstlisting}}
			\caption{การสร้างลอจิกของหน้าหน้าสร้างประกาศ MgPost.vue}
			\label{Fig:MgPostJs}
		\end{figure}
		จากภาพที่ \ref{Fig:MgPostJs} โครงสร้างลอจิกของหน้าสร้างประกาศ สามารถอธิบายการทำงานได้ดังนี้
		\begin{itemize}[label={--}]
			\item บรรทัดที่ 1 ชื่อฟังก์ชัน
			\item บรรทัดที่ 4-5 ข้อมูลของประกาศ
			\item บรรทัดที่ 6-30 เรียกใช้งาน Cloud Firestore เพื่อบันทึกประกาศลงฐานข้อมูล
			\item บรรทัดที่ 10-24 เป็นส่วนที่ใช้ในการอัพโหลดเอกสารแนบไปยังไฟร์เบส Storage
			\item บรรทัดที่ 13-19 เป็นการอัพเดทข้อมูล URL ที่ได้จากการอัพโหลดไฟล์แนบเข้าสู่ฐานข้อมูล Cloud Firestore
		\end{itemize}
		
		\subsection{โครงสร้างของการสร้างหน้าอัพโหลดเอกสารที่เกี่ยวข้อง}
		\begin{figure}[H]
			{\setstretch{1.0}\begin{lstlisting}
 <Modal  v-model="modalNewDoc"
 title="Upload File">
  <Form :model="formItem" :label-width="80">
   <FormItem label="Document title">
    <Input v-model="formItem.title"></Input>
   </FormItem>
   <FormItem label="Document detail">
    <Input v-model="formItem.description" type="textarea" :autosize="{minRows: 2,maxRows: 5}"></Input>
   </FormItem>
   <FormItem label="Select file">
    <Upload
    :before-upload="handleUpload"
    action="https://shielded-earth-61349.herokuapp.com/">
     <Button :type="type" icon="ios-cloud-upload-outline"> {{ uploadBtnTile }} </Button>
    </Upload>
   </FormItem>
  </Form>
  <div slot="footer">
   <Button type="primary" :loading="loading" @click="newDoc">upload</Button>
   <Button type="ghost" style="margin-left: 8px" @click="modalNewDoc = !modalNewDoc">cancle</Button>
  </div>
 </Modal>
				\end{lstlisting}}
			\caption{การสร้างหน้าจอส่วนติดต่อผู้ใช้ของหน้าอัพโหลดเอกสารที่เกี่ยวข้อง MgDocument.vue} %TODO ตัวหนังสือแปลกๆ
			\label{Fig:MgDocument}
		\end{figure}
		จากภาพที่ \ref{Fig:MgDocument} โครงสร้างของการสร้างหน้าจอส่วนติดต่อผู้ใช้ของหน้าอัพโหลดเอกสารที่เกี่ยวข้อง สามารถอธิบายการทำงานได้ดังนี้
		\begin{itemize}[label={--}]
			\item บรรทัดที่ 1-22 สร้างป๊อปอัพแสดงหน้าอัพโหลดไฟล์เอกสาร
			\item บรรทัดที่ 3-17 สร้างฟอร์ม
			\item บรรทัดที่ 11-15 สร้างปุ่มอัพโหลดเอกสาร
			\item บรรทัดที่ 19 สร้างปุ่มบันทึกเอกสาร
			\item บรรทัดที่ 20 สร้างปุ่มยกเลิกการอัพโหลดเอกสาร  
		\end{itemize}
		\begin{figure}[H]
			{\setstretch{1.0}\begin{lstlisting}
newDoc(){
 var vm = this;
 vm.loading = true;
 let key = '';
 this.formItem.time = new Date();
 db.collection("Docs").add(this.formItem)
 .then(function(docRef) {
  key = docRef.id;
  if(vm.file != null){
   let storageRef = storage.ref('Docs/'+key);
   let fileRef = storageRef.child(vm.file.name+"");
   fileRef.put(vm.file).then(function(snapshot) {
    db.collection("Docs").doc(key).update({
      fileURL: snapshot.metadata.downloadURLs[0],
      fileType: vm.file.name.slice(vm.file.name.lastIndexOf('.'))
    }).then(() => {
     vm.$Notice.success({
       title: 'Success',
      desc: ''
    });
  }).catch(function(error) {
   vm.$Notice.warning({
     title: 'Fail',
     desc:''
   });
 });
}

				\end{lstlisting}}
			\caption{การสร้างลอจิกของหน้าหน้าอัพโหลดเอกสารที่เกี่ยวข้อง MgDocument.vue}
			\label{Fig:MgDocumentJs}
		\end{figure}
		จากภาพที่ \ref{Fig:MgDocumentJs} โครงสร้างลอจิกของหน้าสร้างประกาศ สามารถอธิบายการทำงานได้ดังนี้
		\begin{itemize}[label={--}]
			\item บรรทัดที่ 6-26 เป็นการอัพเดทข้อมูลเอกสารเข้าสู่ฐานข้อมูล Cloud Firestore
			\item บรรทัดที่ 10-12 เป็นการเรียกใช้ไฟร์เบส Storage เพื่อทำการอัพโหลดไฟล์เอกสาร
			\item บรรทัดที่ 13-20 ใช้ในการอัพเดทข้อมูล URL ไปยังฐานข้อมูล Cloude Firestore
		\end{itemize}
		
		\subsection{โครงสร้างของการสร้างหน้าสร้างกำหนดการจองคิวส่งเอกสาร}
		\begin{figure}[H]
			{\setstretch{1.0}\begin{lstlisting}
<Modal v-model="modalNewQueue">
 <p slot="header" style="color:#3498db;text-align:center">
  <Icon type="information-circled"></Icon>
  <span>Create sumbit document date</span>
 </p>
 <div>
  <Form :model="formItem" :label-width="80">
   <FormItem label="title">
    <Input v-model="formItem.title"></Input>
   </FormItem>
   <FormItem label="date">
    <DatePicker  v-model="formItem.date" format="d-MMMM-yyyy" type="daterange" placement="bottom-end" placeholder="select" style="width: 200px"></DatePicker>
   </FormItem>
   <FormItem label="time">
    <TimePicker v-model="formItem.time" format="HH:mm" type="timerange" placement="bottom-end" placeholder="select" style="width: 200px"></TimePicker>
   </FormItem>
   <FormItem label="students per hr">
    <InputNumber :max="100" :min="1" v-model="formItem.count" style="width: 200px"></InputNumber>
   </FormItem>
  </Form>
 </div>
 <div slot="footer">
   <Button type="success" size="large" long @click="saveQueue">save</Button>
 </div>
</Modal>
				\end{lstlisting}}
			\caption{การสร้างหน้าจอส่วนติดต่อผู้ใช้ของหน้าสร้างกำหนดการจองคิวส่งเอกสาร MgQueue.vue}
			\label{Fig:MgQueue}
		\end{figure}
		จากภาพที่ \ref{Fig:MgQueue} โครงสร้างของการสร้างหน้าจอส่วนติดต่อผู้ใช้ของหน้าสร้างกำหนดการจองคิวส่งเอกสาร สามารถอธิบายการทำงานได้ดังนี้
		\begin{itemize}[label={--}]
			\item บรรทัดที่ 1-25 การสร้างหน้าต่างป๊อปอัพ
			\item บรรทัดที่ 4 ชื่อหน้าต่างป๊อปอัพ
			\item บรรทัดที่ 7-20 เป็นการสร้างฟอร์ม
			\item บรรทัดที่ 23 เป็นการสร้างปุ่มบันทึกกำหนดการ
		\end{itemize}
		\begin{figure}[H]
			{\setstretch{1.0}\begin{lstlisting}
saveQueue(){
 var vm = this;
 db.collection("Queue").add(this.formItem)
 .then(function(docRef) {
   vm.modalNewQueue = false;
   vm.$Notice.success({
     title: 'success',
     desc: ''
   });
 }).catch(function(){
   vm.modalNewQueue = false;
   vm.$Notice.warning({
     title: 'fial',
     desc:''
  });
 })
}
				\end{lstlisting}}
			\caption{การสร้างลอจิกของหน้าสร้างกำหนดการจองคิวส่งเอกสาร MgQueue.vue}
			\label{Fig:MgQueueJs}
		\end{figure}
		จากภาพที่ \ref{Fig:MgQueueJs} โครงสร้างลอจิกของหน้าสร้างกำหนดการจองคิวส่งเอกสาร สามารถอธิบายการทำงานได้ดังนี้
		\begin{itemize}[label={--}]
			\item บรรทัดที่ 3-15 ที่ได้จากการอัพโหลดไฟล์แนบเข้าสู่ฐานข้อมูล Cloud Firestore
		\end{itemize}

\section{การพัฒนาเว็บแอปพลิเคชันแอนดรอยด์แอปพลิเคชัน}
		การพัฒนาเว็บแอปพลิเคชันแอนดรอยด์แอปพลิเคชันระบบกองทุนเงินให้กู้ยืมเพื่อการศึกษาคณะวิทยาศาสตร์ มหาวิทยาลัยอุบลราชธานี มีวัตถุประสงค์หลักเพื่ออำนวยความสะดวกต่อนักศึกษาที่โดยส่วนใหญ่ใช้อุปกรณ์พกพาและต้องการใช้งานฮาร์ดแวร์(Hardware)เช่น กล้องที่ใช้ในการถ่ายภาพสำเนาเอกสาร เป็นต้น
	\subsection{โครงสร้างของการสร้างหน้า MainActivity}
		\begin{figure}[H]
			{\setstretch{1.0}\begin{lstlisting}
private FirebaseAuth mAuth;
private FeedFragment feedFrag = FeedFragment.newInstance();
private ChatFragment chatFrag = ChatFragment.newInstance();
private DocumentsFragment docFrag = DocumentsFragment.newInstance();
private ScheduleFragment sheduleFrag = ScheduleFragment.newInstance();
private SubmitFragment submitFrag = SubmitFragment.newInstance();
private UserChatFragment userChatFrag = UserChatFragment.newInstance();
private CheckinFragment checkinFrag = CheckinFragment.newInstance();
				\end{lstlisting}}
			\caption{ตัวแปรในคลาส MainActivity}
			\label{Fig:MainActivity}
		\end{figure}
		จากภาพที่ \ref{Fig:MainActivity} ตัวแปรที่ประกาศขึ้นเพื่อใช้ในการทำงานของคลาส MainActivity สามารถอธิบายได้ดังนี้
		\begin{itemize}[label={--}]
			\item บรรทัดที่ 1 ตัวแปร mAuth ใช้ในการจัดเก็บสถานะและข้อมูลของผู้ใช้
			\item บรรทัดที่ 2 ตัวแปร feedFrag ใช้แสดงผลหน้าจอข่าวสาร  
			\item บรรทัดที่ 3 ตัวแปร chatFrag ใช้แสดงผลหน้าจอสนทนาสำหรับเจ้าหน้าที่
			\item บรรทัดที่ 4 ตัวแปร chatFrag ใช้แสดงผลหน้าจอเอกสารที่เกี่ยวข้อง
			\item บรรทัดที่ 5 ตัวแปร scheduleFrag ใช้แสดงผลหน้าจอปฏิทินกำหนดการ
			\item บรรทัดที่ 6 ตัวแปร submitFrag ใช้แสดงผลหน้าจอส่งสำเนาเอกสาร 
			\item บรรทัดที่ 7 ตัวแปร userChatFrag ใช้แสดงผลหน้าจอสนทนาสำหรับนักศึกษา
			\item บรรทัดที่ 8 ตัวแปร checkinFrag ใช้แสดงผลหน้าจอจองคิวส่งเอกสาร
		\end{itemize}
		\begin{figure}[H]
			{\setstretch{1.0}\begin{lstlisting}
new DrawerBuilder()
.addDrawerItems(
   feed, chat, event, doc, submit, checkin, faq, about, setting, account, logout
).withOnDrawerItemClickListener(new Drawer.OnDrawerItemClickListener() {
@Override
  public boolean onItemClick(View view, int position, IDrawerItem drawerItem) {
    long id = drawerItem.getIdentifier();
   if (id == 1) {
     getSupportFragmentManager().beginTransaction()
     .replace(R.id.contentContainer, feedFrag)
     .commit();
   } else if (id == 2) {
    if (currentUser.getEmail().contains("tapgabee")) {
      getSupportFragmentManager().beginTransaction()
      .replace(R.id.contentContainer, chatFrag)
      .commit();
    } else {
       getSupportFragmentManager().beginTransaction()
       .replace(R.id.contentContainer, userChatFrag)
       .commit();
    }
   } else if (id == 3) {
      getSupportFragmentManager().beginTransaction()
      .replace(R.id.contentContainer, sheduleFrag)
      .commit();
   } else if (id == 4) {
      getSupportFragmentManager().beginTransaction()
      .replace(R.id.contentContainer, docFrag)
      .commit();
   }else if (id == 5) {
      getSupportFragmentManager().beginTransaction()
      .replace(R.id.contentContainer, submitFrag)
      .commit();
   } else if (id == 6) {
      getSupportFragmentManager().beginTransaction()
      .replace(R.id.contentContainer, checkinFrag)
      .commit();
   } else if (id == 11) {
      mAuth.signOut();
      startActivity(new Intent(MainActivity.this, MainActivity.class));
      finish();
   }
}
				\end{lstlisting}}
			\caption{โค๊ดส่วนที่ใช้ในการสร้างเมนูนำทางหลักภายในคลาส MainActivity}
			\label{Fig:MainActivity2}
		\end{figure}

		จากภาพที่ \ref{Fig:MainActivity2} สามารถอธิบายการทำงานโค๊ดส่วนที่ใช้ในการสร้างเมนูนำทางหลักภายในคลาส MainActivity ได้ดังนี้
		\begin{itemize}[label={--}]
			\item บรรทัดที่ 1 เป็นการสร้างเมนูนำทาง
			\item บรรทัดที่ 2-3 เป็นการเพิ่ม Fragment ต่างๆ เข้าไปยังเมนูนำทาง
			\item บรรทัดที่ 4-6 เป็นการเพิ่มการดักจับอีเวนต์ (Event) เพื่อสลับหน้าจอการแสดงผลที่เกิดขึ้นเมื่อผู้ใช้กดที่เมนูนำทาง
			\item บรรทัดที่ 8-11 เป็นการแสดงผลหน้าข่าวสาร
			\item บรรทัดที่ 13-16 เป็นการแสดงผลหน้าสนทนาสำหรับเจ้าหน้าที่
			\item บรรทัดที่ 17-20 เป็นการแสดงผลหน้าสนทนาสำหรับนักศึกษา
			\item บรรทัดที่ 22-25 เป็นการแสดงผลหน้าปฏิทินกำหนดการ
			\item บรรทัดที่ 26-29 เป็นการแสดงผลหน้าดาวน์โหลดเอกสาร
			\item บรรทัดที่ 30-33 เป็นการแสดงผลหน้าส่งสำเนาเอกสาร
			\item บรรทัดที่ 34-37 เป็นการแสดงผลหน้าจองคิวส่งเอกสาร
			\item บรรทัดที่ 38-41 เป็นการรีเฟรช(refresh)หน้าจอเมื่อผู้ใช้กดปุ่มออกจากระบบ
		\end{itemize}
	
	\subsection{โครงสร้างของการสร้างหน้า FeedFragment}
	\begin{figure}[H]
		{\setstretch{1.0}\begin{lstlisting}
private RecyclerView recyclerView;
private FirebaseFirestore db;
private ArrayList<Post> posts;
private FeedItemAdapter adapter;
			\end{lstlisting}}
		\caption{ตัวแปรในคลาส FeedFragment}
		\label{Fig:FeedFragment}
	\end{figure}
	จากภาพที่ \ref{Fig:FeedFragment} ตัวแปรที่ประกาศขึ้นเพื่อใช้ในการทำงานของคลาส FeedFragment สามารถอธิบายได้ดังนี้
	\begin{itemize}[label={--}]
		\item บรรทัดที่ 1 ตัวแปร recyclerView ใช้ในการแสดงข้อมูลลิสต์รายการข่าวสาร
		\item บรรทัดที่ 2 ตัวแปร db ใช้ในการสืบค้นข้อมูลจากข่าวสารจาก Cloud Firestore 
		\item บรรทัดที่ 3 ตัวแปร posts ใช้ในการเก็บชุดข้อมูลที่ได้จากการสืบค้นข้อมูล
		\item บรรทัดที่ 4 ตัวแปร adapter ใช้ในการแปลงชุดข้อมูลเป็นลิสต์รายการเพื่อแสดงบน recyclerView
	\end{itemize}
	\begin{figure}[H]
		{\setstretch{1.0}\begin{lstlisting}
db.collection("Posts")
.orderBy(getString(R.string.key_time), Query.Direction.DESCENDING)
.get()
.addOnCompleteListener(new OnCompleteListener<QuerySnapshot>() {
  @Override
  public void onComplete(@NonNull Task<QuerySnapshot> task) {
    if (task.isSuccessful() && isAdded()) {
      for (DocumentSnapshot document : task.getResult()) {
        Log.d(TAG, document.getId() + " => " + document.getData());
        Map<String, Object> data = document.getData();
        Post post = new Post();
        post.setTitle(data.get(getString(R.string.key_title)).toString());
        post.setCollection(data.get(getString(R.string.key_collection)).toString());
        post.setDate((Date)data.get(getString(R.string.key_time)));
        post.setDescription(data.get(getString(R.string.key_description)) == null ? "" : data.get(getString(R.string.key_description)).toString());
        post.setFileURL(data.get(getString(R.string.key_fileURL)) == null ? "" : data.get(getString(R.string.key_fileURL)).toString());
        post.setFileName(data.get(getString(R.string.key_fileName)) == null ? "" : data.get(getString(R.string.key_fileName)).toString());
       posts.add(post);
    }
   recyclerView.setLayoutManager(new LinearLayoutManager(getActivity()));
   recyclerView.setAdapter(adapter);
   adapter.notifyDataSetChanged();
  } else {
    Log.w(TAG, "Error getting documents.", task.getException());
  }
 }
});
			\end{lstlisting}}
		\caption{โค๊ดส่วนที่ใช้ในการสืบค้นข้อมูลจาก Cloude Firestore ภายในคลาส FeedFragment}
		\label{Fig:FeedFragment2}
	\end{figure}
	จากภาพที่ \ref{Fig:FeedFragment2} โค๊ดส่วนที่ใช้ในการสืบค้นข้อมูลจาก Cloude Firestore ภายในคลาส FeedFragment สามารถอธิบายได้ดังนี้
	\begin{itemize}[label={--}]
		\item บรรทัดที่ 1-3 เริ่มทำการสืบค้นข้อมูลประกาศทั้งหมดพร้อมทั้งเรียงลำดับข้อมูลจากประกาศล่าสุดก่อน
		\item บรรทัดที่ 4-18 รับผลการสืบค้นพร้อมทั้งเพิ่มข้อมูลที่ได้แต่ละแถวเข้าไว้ที่ตัวแปร posts 
		\item บรรทัดที่ 20-22 ทำการอัพเดทข้อมูลที่แสดงอยู่บน recyclerView
	\end{itemize}
	\begin{figure}[H]
		{\setstretch{1.0}\begin{lstlisting}
@Override
public void recyclerViewListClicked(View v, int position) {
  Intent intent = new Intent(getActivity(), PostDetailActivity.class);
  intent.putExtra(getString(R.string.key_title), posts.get(position).getTitle());
  intent.putExtra(getString(R.string.key_collection), posts.get(position).getCollection());
  intent.putExtra(getString(R.string.key_time), posts.get(position).getDate());
  intent.putExtra(getString(R.string.key_description), posts.get(position).getDescription());
  intent.putExtra(getString(R.string.key_fileURL), posts.get(position).getFileURL());
  intent.putExtra(getString(R.string.key_fileName), posts.get(position).getFileName());
  if (getActivity() != null)
    getActivity().startActivity(intent);
}
			\end{lstlisting}}
		\caption{โค๊ดส่วนที่ใช้ในการดักอีเวนต์เมื่อผู้ใช้กดที่แถวประกาศในคลาส FeedFragment}
		\label{Fig:FeedFragment3}
	\end{figure}
	จากภาพที่ \ref{Fig:FeedFragment3}{โค๊ดส่วนที่ใช้ในการดักอีเวนต์เมื่อผู้ใช้กดที่แถวประกาศในคลาส FeedFragment สามารถอธิบายได้ดังนี้
	\begin{itemize}[label={--}]
		\item บรรทัดที่ 1 ประกาศตัวแปร intent ประเภทตัวแปร Intent เพื่อใช้กำหนดแอคทิวิตี้ปลายทางซึ่งในที่นี้คือ PostDetailActivity
		\item บรรทัดที่ 4-9 เป็นการเพิ่มข้อมมูลเข้าเก็บไว้ที่ตัวแปร intent โดยดึงข้อมูลใน posts มาจากตำแหน่งแถวที่ถูกผู้ใช้กด
		\item บรรทัดที่ 11 เริ่มการทำงานของแอคทิวิตี้ PostDetailActivity
	\end{itemize}

	\subsection{โครงสร้างของการสร้างหน้า PostDetailActivity}
	\begin{figure}[H]
		{\setstretch{1.0}\begin{lstlisting}
private TextView tvTitle, tvDescription, tvCollection, tvDate;
private String strTitle, strDescription, strDate, strCollection, strFileURl, strFileName;
private FloatingActionButton fab;
private DownloadManager downloadManager;

strTitle = getIntent().getStringExtra(getString(R.string.key_title));
strDescription = getIntent().getStringExtra(getString(R.string.key_description));
strDate = getIntent().getStringExtra(getString(R.string.key_time));
strCollection = getIntent().getStringExtra(getString(R.string.key_collection));
strFileURl = getIntent().getStringExtra(getString(R.string.key_fileURL));
strFileName = getIntent().getStringExtra(getString(R.string.key_fileName));
tvTitle.setText(strTitle);
tvDescription.setText(strDescription);
tvDate.setText(strDate);
tvCollection.setText(strCollection);
			\end{lstlisting}}
		\caption{โค๊ดส่วนที่ใช้ในการแสดงผลรายละเอียดประกาศของคลาส PostDetailActivity}
		\label{Fig:PostDetailActivity}
	\end{figure}
	จากภาพที่ \ref{Fig:PostDetailActivity} โค๊ดส่วนที่ใช้ในการแสดงผลรายละเอียดประกาศของคลาส  PostDetailActivity สามารถอธิบายได้ดังนี้
	\begin{itemize}[label={--}]
		\item บรรทัดที่ 1-4 เป็นการประกาศตัวแปรที่ใช้ในการเก็บข้อมูลประกาศ
		\item บรรทัดที่ 6-11 เป็นการดึงค่าที่ถูกส่งมาจากคลาส FeedFragment ผ่านทาง Intent
		\item บรรทัดที่ 12-15 เป็นการแสดงผลข้อมูลต่างๆ ออกทางหน้าจอแสดงผล
	\end{itemize}
	\begin{figure}[H]
		{\setstretch{1.0}\begin{lstlisting}
@Override
public void onClick(View v) {
  int id = v.getId();
  if (id == R.id.fab) {
    downloadManager = (DownloadManager) getSystemService(Context.DOWNLOAD_SERVICE);
    Uri uri = Uri.parse(strFileURl);
    DownloadManager.Request request = new DownloadManager.Request(uri);
    request.setNotificationVisibility(DownloadManager.Request.VISIBILITY_VISIBLE_NOTIFY_COMPLETED);
    request.setDestinationInExternalPublicDir(Environment.DIRECTORY_DOWNLOADS, strFileName);
   downloadManager.enqueue(request);
  }
}
			\end{lstlisting}}
		\caption{โค๊ดส่วนที่ใช้ในการดาวน์โหลดเอกสารของคลาส PostDetailActivity}
		\label{Fig:PostDetailActivity2}
	\end{figure}
	จากภาพที่ \ref{Fig:PostDetailActivity2} โค๊ดส่วนที่ใช้ในการดาวน์โหลดเอกสารของคลาส PostDetailActivity สามารถอธิบายได้ดังนี้
	\begin{itemize}[label={--}]
		\item บรรทัดที่ 1-2 เช็คว่าอีเวนต์ที่เกิดขึ้นมาจากปุ่มดาวน์โหลดเอกสารหรือไม่
		\item บรรทัดที่ 5-9 เป็นการเตรียมความพร้อมสำหรับการดาวน์โหลดเอกสาร
		\item บรรทัดที่ 10 ทำการเริ่มดาวน์โหลดเอกสาร
	\end{itemize}
	
	\subsection{โครงสร้างของการสร้างหน้า ChatActivity}
	\begin{figure}[H]
		{\setstretch{1.0}\begin{lstlisting}
private FirebaseFirestore db;
private MessagesListAdapter adapter;
private MessagesList messagesList;
private String senderId;
private String name;
private String avatar;
private FirebaseAuth mAuth;
private EditText tvMessage;
private FirebaseUser currentUser;
private Button btnSend;
			\end{lstlisting}}
		\caption{โค๊ดส่วนที่ใช้ในการแสดงผลรายละเอียดประกาศของคลาส ChatActivity}
		\label{Fig:ChatActivity}
	\end{figure}
	จากภาพที่ \ref{Fig:ChatActivity} โค๊ดส่วนที่ใช้ในการแสดงผลรายละเอียดประกาศของคลาส ChatActivity สามารถอธิบายได้ดังนี้
	\begin{itemize}[label={--}]
		\item บรรทัดที่ 1 ตัวแปร db ใช้ในการสืบค้นช้อมูลจาก Cloud Firestore
		\item บรรทัดที่ 2 ตัวแปร adapter ใช้ในการแปลงชุดข้อมูลที่ได้จากการสืบค้นเป็นลิสต์รายการ
		\item บรรทัดที่ 3 ตัวแปร messagesList ใช้ในการแสดงบทสนทนา
		\item บรรทัดที่ 4 ตัวแปร senderId ใช้ในการจัดเก็บไอดีของผู้ใช้
		\item บรรทัดที่ 5 ตัวแปร name ใช้ในการจัดเก็บชื่อของผู้ใช้
		\item บรรทัดที่ 6 ตัวแปร avatar ใช้ในการจัดเก็บ url รูปภาพของผู้ใช้
		\item บรรทัดที่ 7 ตัวแปร mAuth ใช้ในการจัดเก็บสถานะและข้อมูลของผู้ใช้
		\item บรรทัดที่ 8 ตัวแปร tvMessage ใช้ในการกรอกข้อความ
		\item บรรทัดที่ 9 ตัวแปร currentUser ใช้ในการจัดเก็บสถานะและข้อมูลของผู้ใช้คนปัจจุบัน
		\item บรรทัดที่ 10 ตัวแปร btnSend เป็นปุ่มที่ใช้สำหรับกดส่งข้อความ
	\end{itemize}
	\begin{figure}[H]
		{\setstretch{1.0}\begin{lstlisting}
db.collection("Chats").document(key).collection("messages")
.orderBy("time")
.addSnapshotListener(new EventListener<QuerySnapshot>() {
@Override
public void onEvent(QuerySnapshot documentSnapshots, FirebaseFirestoreException e) {
  if (e != null) {
     System.err.println("Listen failed:" + e);
     return;
  }

  String id;
  String usrId;
  String text;
  Date createdAt;
  adapter.clear();
  for (DocumentSnapshot document : documentSnapshots) {
     Map<String, Object> data = document.getData();
     id = document.getId();
     text = data.get("message").toString();
     createdAt = (Date) data.get("time");
     usrId = data.get("senderId").toString();
     Message m = new Message(id, text, createdAt, new User(usrId, name, avatar));
     adapter.addToStart(m, true);
     adapter.notifyDataSetChanged();
  }
 }
}
);
			\end{lstlisting}}
		\caption{โค๊ดส่วนที่ใช้ในการสืบค้นประวัติการสนทนาของคลาส ChatActivity}
		\label{Fig:ChatActivity2}
	\end{figure}
	จากภาพที่ \ref{Fig:ChatActivity2} โค๊ดส่วนที่ใช้ในการสืบค้นประวัติการสนทนาของคลาส ChatActivity สามารถอธิบายได้ดังนี้
	\begin{itemize}[label={--}]
		\item บรรทัดที่ 1-2 ทำการสืบค้นข้อมูลประวัติการสนทนาจาก Cloud Firestore 
		\item บรรทัดที่ 11-14 สร้างตัวแปรเพื่อใช้ในการจัดเก็บข้อมูล
		\item บรรทัดที่ 16-24 ทำการอ่านค่าและเพิ่มเข้าลิสต์รายการ
	\end{itemize}
	\begin{figure}[H]
		{\setstretch{1.0}\begin{lstlisting}
final Map<String, Object> map = new HashMap<>();
map.put("message", tvMessage.getText().toString());
map.put("time", new Date());
map.put("senderId", currentUser.getUid());
map.put("name", currentUser.getDisplayName());
map.put("photo", currentUser.getPhotoUrl().toString());
tvMessage.setText("");
db.collection("Chats").document(senderId)
.collection("messages")
.add(map)
.addOnCompleteListener(new OnCompleteListener<DocumentReference>() {
 @Override
 public void onComplete(@NonNull Task<DocumentReference> task) {
    Map<String, Object> map1 = new HashMap<>(); 
    map1.put("lastChat", new Date());
    db.collection("Users")
    .document(senderId)
    .update(map1);
  }
 }
);
			\end{lstlisting}}
		\caption{โค๊ดส่วนที่ใช้ในการส่งข้อความของคลาส ChatActivity}
		\label{Fig:ChatActivity3}
	\end{figure}
	จากภาพที่ \ref{Fig:ChatActivity3} โค๊ดส่วนที่ใช้ในการส่งข้อความของคลาส ChatActivity สามารถอธิบายได้ดังนี้
	\begin{itemize}[label={--}]
		\item บรรทัดที่ 1-6 สร้าง HashMap เพื่อใช้จัดเก็บข้อความ
		\item บรรทัดที่ 8-11 เป็นการเพิ่มชุดข้อมูลเข้าสู่ Cloud Firestore
		\item บรรทัดที่ 14-18 เป็นการอัพเดทเวลาสนทนาล่าสุดของผู้ใช้
	\end{itemize}
	
	\subsection{โครงสร้างของการสร้างหน้า SignInActivity}
	\begin{figure}[H]
		{\setstretch{1.0}\begin{lstlisting}
private FirebaseAuth mAuth;
private  String email, password;
private EditText userName, userPassword;
private ProgressBar simpleProgressBar;
			\end{lstlisting}}
		\caption{โค๊ดส่วนที่ใช้ในการแสดงผลหน้าเข้าสู่ระบบของคลาส SignInActivity}
		\label{Fig:SignInActivity}
	\end{figure}
	จากภาพที่ \ref{Fig:SignInActivity} โค๊ดส่วนที่ใช้ในการแสดงผลหน้าเข้าสู่ระบบของคลาส SignInActivity สามารถอธิบายได้ดังนี้
	\begin{itemize}[label={--}]
		\item บรรทัดที่ 1 ตัวแปร mAuth ใช้ในการจัดเก็บสถานะการเข้าสู่ระบบของผู้ใช้
		\item บรรทัดที่ 2 ตัวแปร email จัดเก็บอีเมลของผู้ใช้
		\item บรรทัดที่ 2 ตัวแปร password จัดเก็บรหัสผ่านของผู้ใช้
		\item บรรทัดที่ 3 ตัวแปร userName ใช้ในการรับค่าอีเมลจากผู้ใช้
		\item บรรทัดที่ 3 ตัวแปร userPassword ใช้ในการรับค่ารหัสผ่านจากผู้ใช้   
	\end{itemize}
	\begin{figure}[H]
		{\setstretch{1.0}\begin{lstlisting}
 email = userName.getText().toString();
 password = userPassword.getText().toString();
 if(email.isEmpty() || email == null || password.isEmpty() || password == null ){
   Toast.makeText(SignInActivity.this, "Please fill data!", Toast.LENGTH_SHORT).show();
   return;
 }
 simpleProgressBar.setVisibility(View.VISIBLE);
 mAuth.signInWithEmailAndPassword(email, password)
 .addOnCompleteListener(this, new OnCompleteListener<AuthResult>() {
  @Override
  public void onComplete(@NonNull Task<AuthResult> task) {
    if (task.isSuccessful()) {
      startActivity(new Intent(SignInActivity.this,MainActivity.class));
      finish();
    } else {
      Toast.makeText(SignInActivity.this, "Authentication failed.",
      Toast.LENGTH_SHORT).show();
    }
    simpleProgressBar.setVisibility(View.INVISIBLE);
   }
  }
 );
			\end{lstlisting}}
		\caption{โค๊ดส่วนที่ใช้ในการเข้าสู่ระบบของคลาส SignInActivity}
		\label{Fig:SignInActivity2}
	\end{figure}
	จากภาพที่ \ref{Fig:SignInActivity2} โค๊ดส่วนที่ใช้ในการเข้าสู่ระบบของคลาส SignInActivity สามารถอธิบายได้ดังนี้
	\begin{itemize}[label={--}]
		\item บรรทัดที่ 1 ตัวแปร email จัดเก็บอีเมลของผู้ใช้
		\item บรรทัดที่ 2 ตัวแปร password จัดเก็บรหัสผ่านของผู้ใช้
		\item บรรทัดที่ 3 ตรวจสอบความถูกต้องของข้อมูล
		\item บรรทัดที่ 8-21 เป็นการเรียกใช้ไฟร์เบส Authentication เพื่อเข้าใช้งานระบบ   
	\end{itemize}
	
	\subsection{โครงสร้างของการสร้างหน้า ScheduleFragment}
	\begin{figure}[H]
		{\setstretch{1.0}\begin{lstlisting}
private RecyclerView recyclerView;
private FirebaseFirestore db;
private ArrayList<Event> events;
private EventItemAdapter adapter;
private DatePickerTimeline datePicker;
			\end{lstlisting}}
		\caption{โค๊ดส่วนที่ใช้ในการแสดงผลหน้าปฏิทินของคลาส ScheduleFragment}
		\label{Fig:ScheduleFragment}
	\end{figure}
	จากภาพที่ \ref{Fig:ScheduleFragment} โค๊ดส่วนที่ใช้ในการแสดงผลหน้าปฏิทินของคลาส ScheduleFragment สามารถอธิบายได้ดังนี้
	\begin{itemize}[label={--}]
		\item บรรทัดที่ 1 ตัวแปร recyclerView ใช้ในการแสดงลิสต์รายการกำหนดการ
	    \item บรรทัดที่ 2 ตัวแปร db ใช้ในการสืบค้นช้อมูลจาก Cloud Firestore
		\item บรรทัดที่ 3 ตัวแปร events จัดเก็บข้อมูลกำหนดการ
		\item บรรทัดที่ 4 ตัวแปร adapter ใช้ในการแปลงชุดข้อมูลกำหนดการเป็นลิสต์รายการเพื่อใช้แสดงใน recyclerView
		\item บรรทัดที่ 5 ตัวแปร datePicker ใช้ในการแสดงปฏิทิน   
	\end{itemize}
	\begin{figure}[H]
		{\setstretch{1.0}\begin{lstlisting}
db.collection("Events")
.orderBy(getString(R.string.key_time), Query.Direction.DESCENDING)
.get()
.addOnCompleteListener(new OnCompleteListener<QuerySnapshot>() {
 @Override
 public void onComplete(@NonNull Task<QuerySnapshot> task) {
    if (task.isSuccessful() && isAdded()) {
      for (DocumentSnapshot document : task.getResult()) {
        Log.d(TAG, document.getId() + " => " + document.getData());
        Map<String, Object> data = document.getData();
        Date dbDate = (Date) data.get(getString(R.string.key_time));
        if (dbDate.getDate() == datePicker.getSelectedDay() && dbDate.getMonth() == datePicker.getSelectedMonth()) {
           Event event = new Event();
           event.setTitle(data.get(getString(R.string.key_title)).toString());
           event.setDescription(data.get(getString(R.string.key_description)).toString());
           event.setTime((Date) data.get(getString(R.string.key_time)));
           events.add(event);
       }
    }
  recyclerView.setLayoutManager(new LinearLayoutManager(getActivity()));
  recyclerView.setAdapter(adapter);
  adapter.notifyDataSetChanged();
    } else {
      Log.w(TAG, "Error getting documents.", task.getException());
    }
  }
 }
);
			\end{lstlisting}}
		\caption{โค๊ดส่วนที่ใช้ในการสืบค้นข้อมูลกำหนดการของคลาส ScheduleFragment}
		\label{Fig:ScheduleFragment2}
	\end{figure}
	จากภาพที่ \ref{Fig:ScheduleFragment2} โค๊ดส่วนที่ใช้ในการสืบค้นข้อมูลกำหนดการของคลาส ScheduleFragment สามารถอธิบายได้ดังนี้
	\begin{itemize}[label={--}]
		\item บรรทัดที่ 1-4 เป็นการสืบค้นข้อมูลกำหนดการโดยเรียงลำดับข้อมูลล่าสุดก่อน
		\item บรรทัดที่ 7-19 เป็นการนำข้อมูลที่ได้จากการสืบค้นแปลงเป็นลิสต์รายการและแสดงผล
		\item บรรทัดที่ 20-22 เป็นการอัพเดทลิสต์รายการ
	\end{itemize}
	
	\subsection{โครงสร้างของการสร้างหน้า ScheduleFragment}
	\begin{figure}[H]
		{\setstretch{1.0}\begin{lstlisting}
private RecyclerView recyclerView;
private FirebaseFirestore db;
private ArrayList<Event> events;
private EventItemAdapter adapter;
private DatePickerTimeline datePicker;
			\end{lstlisting}}
		\caption{โค๊ดส่วนที่ใช้ในการแสดงผลหน้าปฏิทินของคลาส ScheduleFragment}
		\label{Fig:ScheduleFragment}
	\end{figure}
	จากภาพที่ \ref{Fig:ScheduleFragment} โค๊ดส่วนที่ใช้ในการแสดงผลหน้าปฏิทินของคลาส ScheduleFragment สามารถอธิบายได้ดังนี้
	\begin{itemize}[label={--}]
		\item บรรทัดที่ 1 ตัวแปร recyclerView ใช้ในการแสดงลิสต์รายการกำหนดการ
		\item บรรทัดที่ 2 ตัวแปร db ใช้ในการสืบค้นช้อมูลจาก Cloud Firestore
		\item บรรทัดที่ 3 ตัวแปร events จัดเก็บข้อมูลกำหนดการ
		\item บรรทัดที่ 4 ตัวแปร adapter ใช้ในการแปลงชุดข้อมูลกำหนดการเป็นลิสต์รายการเพื่อใช้แสดงใน recyclerView
		\item บรรทัดที่ 5 ตัวแปร datePicker ใช้ในการแสดงปฏิทิน   
	\end{itemize}
