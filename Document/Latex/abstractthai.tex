\begin{thaiabstract}
    แอปพลิเคชันสูงวัย มายเฟรนด์ (OLD MY FRIENDS) ได้ถูกพัฒนาขึ้นโดยสามารถทำงานได้ทั้งบนระบบปฏิบัติการแอนดรอยด์ (Android) และไอโอเอส (IOS) 
    ซึ่งจะเป็นแอพพลิเคชั่น ที่ช่วยอำนวยความสะดวกสำหรับกลุ่มผู้สูงอายุ โดยประกอบด้วยฟังก์ชันการทำงานที่จำเป็น ซึ่งได้แก่ระบบแชทบอทเพื่อถามตอบเรื่องโรค 
    จะช่วยให้เข้าใจในโรคที่เกิดกับผู้สูงอายุจำนวน 5 โรค โดยสามารถให้ข้อมูลสาเหตุ อาการ วิธีป้องกันและดูแลรักษาเบื้องต้นได้ 
    และระบบยังสามารถแจ้งเตือนการทานยาได้ด้วย
    นอกจากนี้ยังมีส่วนให้ผู้สูงอายุสามารถโพสต์กระทู้หรือหัวข้อสนทนา เพื่อช่วยแชร์เรื่องราวที่สนใจกับเพื่อนหรือกลุ่มได้ 
    ทั้งยังมีฟังก์ชันของการติดตามตำแหน่งของผู้ใช้งานเพื่อการติดตามหรือแจ้งเตือนกรณีที่หลงทาง โดยในการพัฒนาจะใช้ ionic 3 
    และ dialogflow ซึ่งเป็นส่วนในการสร้างแชทบอทเพื่อการพูดคุยโต้ตอบ โดยประโยชน์ของแอปพลิเคชันสูงวัย มายเฟรนด์ (OLD MY FRIENDS) 
    คือช่วยให้คำแนะนำการส่งเสริมพฤติกรรมสุขภาพที่ดีผ่านระบบแชทบอท และสร้างกลุ่มเพื่อนผ่านการพูดคุยทางสังคมออนไลน์ทำให้ช่วยลดความเหงาและสร้างกิจกรรมให้กับผู้สูงอายุได้

\noindent
\\คำสำคัญ: แชทบอท ไอโอนิก แอนดรอยด์ ไอโอเอส ผู้สูงอายุ
\end{thaiabstract}
