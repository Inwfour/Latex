\begin{thaiabstract}
    แอปพลิเคชันสูงวัย มายเฟรนด์ (OLD MY FRIENDS) ได้ถูกพัฒนาขึ้นโดยสามารถทำงานได้ทั้งบนระบบปฏิบัติการแอนดรอยด์ (Android) และไอโอเอส (IOS) 
    ซึ่งเป็นแอพพลิเคชั่นที่ช่วยอำนวยความสะดวกสำหรับกลุ่มผู้สูงอายุ โดยประกอบด้วยระบบแชทบอทเพื่อถามตอบเรื่องโรคที่
    เกิดกับผู้สูงอายุจำนวน 5 โรค ซึ่งสามารถให้ข้อมูลสาเหตุ อาการ วิธีป้องกันและดูแลรักษาเบื้องต้นได้ 
    และระบบยังสามารถแจ้งเตือนการทานยาได้ด้วย
    นอกจากนี้ยังมีระบบโพสต์กระทู้หรือหัวข้อสนทนา เพื่อช่วยแชร์เรื่องราวที่น่าสนใจกับเพื่อนหรือกลุ่มได้ 
    ทั้งยังมีระบบการระบุตำแหน่งของผู้ใช้งานเพื่อช่วยติดตามหรือแจ้งเตือนกรณีที่หลงทาง โดยในการพัฒนานั้นจะใช้ ionic 3 
    และ dialogflow เพื่อสร้างแชทบอทสำหรับโต้ตอบ โดยประโยชน์ของแอปพลิเคชัน 
    คือช่วยให้คำแนะนำการส่งเสริมพฤติกรรมสุขภาพที่ดีผ่านระบบแชทบอท และสร้างกลุ่มเพื่อนผ่านการพูดคุยทางสังคมออนไลน์ทำให้ช่วยลดความเหงาและสร้างกิจกรรมให้กับผู้สูงอายุได้

\noindent
\\คำสำคัญ: แชทบอท ไอโอนิก แอนดรอยด์ ไอโอเอส ผู้สูงอายุ dialogflow
\end{thaiabstract}
