\chapter{บทนำ}

\section{ที่มาและเหตุผล }
เนื่องจากระบบการทำงานเดิม มีข้อจำกัดดังนี้ XX และในปัจจุบัน เทคโนโลยี XX สามารถทำให้ดีขึ้น ดังนี้ XX

ดังนั้น ผู้พัฒนาจึงเสนอระบบงาน XX โดยใช้เทคโนโลยี XX เพื่อแก้ปัญหา โดย มีความสามารถดังต่อไปนี้ XX 


\section{วัตถุประสงค์}
\begin{enumerate}
	\item เพื่อพัฒนาระบบแอปพลิเคชัน 
	\item  เพื่ออำนวยสร้างสิ่งอำนวยความสะดวกสำหรับ 
\end{enumerate}
\section{ขอบเขตของโครงงาน}
\begin{enumerate}[label=1.3.\arabic*]
	\item เจ้าหน้าที่
	 \begin{itemize}
		 	\item สามารถจัดการ X
		 	\item สามารถเพิ่ม X
		 	\item สามารถตรวจสอบอนุมัติ X
	 \end{itemize}
	\item นักศึกษา
	\begin{itemize}
		\item สามารถดู X ได้
		\item สามารถกำหนด X ได้
		\item สามารถดาวน์โหลด X ได้
		\item สามารถจอง X ได้
		\item สามารถส่ง X ได้
		\item สามารถสนทนากับ X ผ่านระบบได้
	\end{itemize}
\end{enumerate}
\section{ประโยชน์ที่คาดว่าจะได้รับ}
\begin{enumerate}
	\item ช่วยอำนวยความสะดวก XX
	\item ช่วยกระจายงานของ XX 
  \item ช่วยจัดระบบ XX
\end{enumerate}
\section{เครื่องมือที่ใช้ในการพัฒนา (Development tools)}
\subsection{ฮาร์ดเเวร์}
\begin{enumerate}
	\item สมาร์ทโฟน (Smart phone)
		\begin{itemize}
			\item ทำงานบนระบบปฏิบัติการแอนดรอย์เวอร์ชัน 5.0 หรือ API Level 21
			\item หน่วยประมาลผลกลาง Mediatek MT6753 Octa-core ความเร็ว 1.3 กิกะเฮิร์ตซ์ (Gigahertz, GHz)
			\item หน่วยประมวลผลกราฟฟิกอย่างน้อย Mali-T720MP3
			\item หน่วยความจำหลักอย่างน้อย 2 กิกะไบต์ (Gigabyte, GB)
			\item หน่วยความจำสำรองอย่างน้อย 16 กิกะไบต์ (Gigabyte, GB)
			\item หน้าจอแสดงผลความละเอียดอย่างน้อย 1080 x 1920 พิกเซล  (Pixel)
			\item หน้าจอแสดงผลขนาดอย่างน้อย 5 นิ้ว
			\item กล้องถ่ายรูปความละเอียดอย่างน้อย 13 เมกกะพิกเซล (Magapixel)
		\end{itemize}
	
	\item เครื่องคอมพิวเตอร์ส่วนบุคคล (Personal computer)
		\begin{itemize}
			\item  ทำงานบนระบบปฏิบัติการ Elementary os พื้นฐานการทำงานบน Linux
			\item  หน่วยประมวลผลกลาง Intel Core i3-3217U ความเร็ว 1.80 กิกะเฮิร์ตซ์ (Gigahertz, GHz)
			\item  หน่วยประมวลผลกราฟฟิก NVIDIA GeForce GT 720M ความจำ 2 กิกะไบต์ (Gigabyte, GB) 
			\item  หน่วยความจำหลัก 4 กิกะไบต์ (Gigabyte, GB)
			\item  หน่วยความจำสำรอง 120 กิกะไบต์ (Gigabyte, GB)
		\end{itemize}
\end{enumerate}

\subsection{ซอฟต์แวร์ (Software)}
\begin{enumerate}
	\item XX.js ซึ่งเป็น Frontend Framework สำหรับพัฒนาเว็บไชต์ 
	\item Node.js คือ Cross Platform Runtime Environment หรือเรียกอีกอย่างว่า Backend Framework ใช้สำหรับเป็นเว็บเซิฟเวอร์ (Web Server) ซึ่งเขียนด้วยภาษา JavaScript 
	\item Android Studio เป็น IDE (Integrated Development Environment) ใช้พัฒนาแอปพลิเคชันสำหรับระบบปฏิบัติการแอนดรอยด์
	\item Android SDK ชุดของเครื่องมือที่ใช้ในการพัฒนาโปรแกรมสำหรับระบบปฏิบัติการแอนดรอยด์
	\item Visual Studio Code เครื่องมือสำหรับพัฒนาเว็บแอปพลิเคชัน
\end{enumerate}

\newpage
\subsection{แผนการดำเนินการ}
	ในการสร้างระบบ XX ผู้พัฒนาได้แบ่งขั้นตอนการดำเนินงานไว้ด้วยกัน 7 ขั้นตอน ดังต่อไปนี้

%\begin{landscape}
%\sffamily
\begin{table}[H]
	\noindent
	\caption{ขั้นตอนการดำเนินงาน}
	\begin{ganttchart}[
		canvas/.append style={fill=none, draw=black!5, line width=.75pt},
		vgrid={*2{draw=black!7, line width=.75pt}},
		title label font=\bfseries\footnotesize,
		bar label node/.append style={
			align=left,
			text width=width("7. Functional Testing On")},
		bar/.append style={draw=none, fill=black!63}
		]{1}{22}
		\gantttitle{2560}{10}
		\gantttitle{2561}{12}\\
		\gantttitle{ต.ค.}{2}
		\gantttitle{พ.ย.}{2}
		\gantttitle{ธ.ค.}{2} 
		\gantttitle{ม.ค.}{2}
		\gantttitle{ก.พ.}{2}
		\gantttitle{มี.ค.}{2}
		\gantttitle{เม.ย.}{2}
		\gantttitle{พ.ค.}{2}
		\gantttitle{มิ.ย.}{2} 
		\gantttitle{ก.ค.}{2}
		\gantttitle{ส.ค.}{2} \\
		\ganttbar{1.ศึกษาความเป็นไปได้}{1}{6} \\
		\ganttbar{2.เสนอหัวข้อโครงงาน}{7}{8} \\
		\ganttbar{3.ศึกษาค้นคว้าข้อมูล}{7}{9} \\
		\ganttbar{4.ศึกษาการใช้เครื่องมือ}{8}{10} \\
		\ganttbar{5.วิเคราะห์และออกแบบ}{9}{12} \\
		\ganttbar{6.เขียนโปรแกรม}{9}{14} \\
		\ganttbar{7.ทดสอบและแก้ปัญหา}{13}{14} \\
		\ganttbar{8.จัดทำเอกสาร}{13}{16} \\
	\end{ganttchart}
	\label{tab:ganttchart}
\end{table}
%\end{landscape}
%TODO แก้เทมเพลตเอาชื่อตารางไว้ด้านบน
