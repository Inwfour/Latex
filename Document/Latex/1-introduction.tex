\chapter{บทนำ}

\section{ที่มาและเหตุผล}
ในภาวะสังคมปัจจุบัน ประเทศไทยในปี 2561 มีผู้สูงอายุทั้งเพศและเพศหญิงเฉลี่ยประมาณ 16.06 เปอร์เซ็นต์ ของประชากรทั้งหมดในประเทศไทย \cite{dop} 
ผู้สูงอายุส่วนใหญ่อยู่บ้านเพียงลำพัง อันเนื่องสาเหตุมาจากลูกหลานต้องออกไปทำงานที่ต่างจังหวัด หรือในกรุงเทพมหานคร ส่งผลให้ผู้สูงอายุ
สนใจที่จะใช้โซเชียลในการติดต่อสื่อสาร หรือแชร์เรื่องราวที่น่าสนใจในคนวัยเดียวกัน ซึ่งในปัจจุบันมีเทคโนโลยีในการสื่อสารมากมาย เช่น เฟสบุ๊ค ไลน์
แต่แอปพลิเคชันเหล่านี้ถูกออกแบบสำหรับคนรุ่นใหม่มากกว่าผู้สูงอายุ ปัญหาคือการใช้งานมีความซับซ้อน ตัวหนังสือที่มีขนาดเล็ก 
รวมไปถึงไม่มีฟังก์ชันที่ช่วยอำนวยความสะดวกให้ผู้สูงอายุในกรณีฉุกเฉินได้เฉพาะเจาะจง นอกจากนี้ผู้สูงอายุยังไม่เข้าใจสาเหตุ วิธีดูแล ที่ถูกต้องในเรื่องโรคที่มักเกิดกับผู้สูงอายุ 
เนื่องจากการค้นหาข้อมูลต่าง ๆ ในปัจจุบันมีความซับซ้อน ทำให้เกิดความสับสนในการใช้งานสำหรับผู้สูงอายุ

ดังนั้นผู้พัฒนาจึงต้องการเสนอแอปพลิเคชันสูงวัยมายเฟรนด์ (OLD MY FRIENDS) ที่ช่วยให้ผู้สูงอายุสามารถสื่อสารหรือแชร์เรื่องราวที่น่าสนใจ
กับคนวัยเดียวกัน และยังสามารถให้ความรู้เบื้องต้น เช่น สาเหตุ อาการ วิธีป้องกัน วิธีรักษา อาการแทรกซ้อน ในเรื่องโรคที่เกิดขึ้นบ่อยกับผู้สูงอายุจำนวน 5 โรค \cite{thaihealth} ได้แก่ 
โรคเบาหวาน โรคซึมเศร้า โรคความดันโลหิตสูง โรคข้อเสื่อม โรคอัลไซเมอร์ ที่จะเป็นประโยชน์สำหรับผู้สูงอายุ และผู้สูงอายุยังสามารถดู
ตำแหน่งปัจจุบันของคนในครอบครัว และยังสามารถแจ้งเหตุเมื่อเกิดเหตุฉุกเฉินด้วยการโทรหรือส่งข้อความไปยังคนในครอบครัวได้ นอกจากนี้
แอปพลิเคชัน ยังประกอบด้วยฟังก์ชันแจ้งเตือนการทานยา ซึ่งจะช่วยให้ผู้สูงอายุสามารถทานยาได้ถูกต้อง และตรงตามเวลา

\section{วัตถุประสงค์}
\begin{enumerate}
	\item เพื่อพัฒนาแอปพลิเคชันให้ความรู้พื้นฐานเรื่องโรคที่เกิดกับผู้สูงอายุผ่านระบบแชทบอทได้
	\item เพื่อพัฒนาระบบแอปพลิเคชันสำหรับผู้สูงอายุให้สามารถแบ่งปันเรื่องราวที่น่าสนใจได้
	\item เพื่อพัฒนาแอปพลิเคชันอำนวยความสะดวกในกรณีฉุกเฉิน
\end{enumerate}
\section{ขอบเขตของโครงงาน}
แอปพลิเคชันสูงวัยมายเฟรนด์มีของเขตการทำงานดังนี้
\begin{itemize}
		\item ผู้ใช้งานสามารถเข้าสู่ระบบได้
		\item ผู้ใช้งานสามารถสนทนากับแชทบอทแบบเรียลไทม์ เพื่อสอบถามเรื่องโรคมีจำนวน 5 โรคได้แก่ โรคเบาหวาน, โรคซึมเศร้า, โรคความดันโลหิตสูง, โรคข้อเสื่อม, โรคอัลไซเมอร์
		\item ผู้ใช้งานสามารถเพิ่ม ลบ แก้ไข โพสท์และคอมเมนท์ได้
		\item ผู้ใช้งานสามารถดู เพิ่ม แก้ไขตำแหน่ง และลบสมาชิกในครอบครัวได้
		\item ผู้ใช้สามารถจัดการการแจ้งเตือน เช่น การแจ้งเตือนการทานยาและแจ้งเตือนฉุกเฉินได้
		\item ผู้ใช้สามารถดูวิธีใช้งานโปรแกรมได้
		\item ผู้ใช้สามารถเพิ่ม ลบ แก้ไข กลุ่มและเพื่อนได้
		\item ผู้ใช้สามารถจัดการแก้ไขข้อมูลส่วนตัวได้
\end{itemize}
\section{ประโยชน์ที่คาดว่าจะได้รับ}
\begin{enumerate}[label=\arabic*]
	\item ช่วยให้ผู้สูงอายุได้รับความรู้ของโรคพื้นฐานที่พบในผู้สูงอายุในรูปแบบแชทบอทแบบเรียลไทม์
	\item ช่วยให้ผู้สูงอายุสามารถโพสท์แบ่งปันประสบการณ์สร้้างกลุ่มเพื่อนในคนวัยเดียวกันได้
	\item ช่วยอำนวยความสะดวกในการติดต่อสื่อสารกับสมาชิกในครอบครัวได้
\end{enumerate}
\section{เครื่องมือที่ใช้ในการพัฒนา (Development tools)}
\subsection{ฮาร์ดเเวร์}
\begin{enumerate}
			\item ทำงานบนระบบปฏิบัติการแอนดรอยด์เวอร์ชัน 4.4 ขึ้นไป
			\item ทำงานบนระบบปฏิบัติการไอโอเอสเวอร์ชัน 10 ขึ้นไป
\end{enumerate}

\subsection{ซอฟต์แวร์ (Software)}
\begin{enumerate}
	\item Ionic 3 ซึ่งเป็น Frontend Framework สำหรับพัฒนาเว็บแอปพลิเคชัน
	\item Firebase คือ Platform ที่รวบรวมเครื่องมือต่าง ๆ ที่ใช้จัดการ Backend หรือ Server side
	\item Dialogflow หรือ Api.ai เป็นแพลตฟอร์มที่ใช้ในการสร้างแชทบอทที่รองรับการทำ Natural Language Processing (NLP)
	\item Node Package Manager หรือ NPM เป็นซอฟต์แวร์ที่มาพร้อมกับ Node ที่ช่วยให้สามารถนำเข้าโมดูลต่าง ๆ ภายใน Node ได้
	\item Library moment.js เป็น JavaScript Library สำหรับจัดการ Date & Time
	\item Google Maps APIs เป็น API ของ Google ไว้สำหรับเรียกใช้แผนที่
	\item Visual Studio Code เครื่องมือสำหรับพัฒนาเว็บแอปพลิเคชัน
\end{enumerate}

\subsection{แผนการดำเนินการ}
	ในการสร้างแอปพลิเคชันสูงวัยมายเฟรนด์ ผู้พัฒนาได้แบ่งขั้นตอนการดำเนินงานไว้ด้วยกัน 8 ขั้นตอน ดังต่อไปนี้

%\begin{landscape}
%\sffamily
\begin{table}[H]
	\noindent
	\caption{ขั้นตอนการดำเนินงาน}
	\begin{ganttchart}[
		canvas/.append style={fill=none, draw=black!5, line width=.75pt},
		vgrid={*2{draw=black!7, line width=.75pt}},
		title label font=\bfseries\footnotesize,
		bar label node/.append style={
			align=left,
			text width=width("10. Functional Testing On")},
		bar/.append style={draw=none, fill=black!63}
		]{1}{12}
		\gantttitle{2561}{4}
		\gantttitle{2562}{8}\\
		\gantttitle{พ.ย.}{2}
		\gantttitle{ธ.ค.}{2} 
		\gantttitle{ม.ค.}{2}
		\gantttitle{ก.พ.}{2}
		\gantttitle{มี.ค.}{2}
		\gantttitle{เม.ย.}{2} \\
		\ganttbar{1.ศึกษาความเป็นไปได้}{1}{2} \\
		\ganttbar{2.เสนอหัวข้อโครงงาน}{3}{3} \\
		\ganttbar{3.ศึกษาค้นคว้าข้อมูล}{4}{7} \\
		\ganttbar{4.ศึกษาการใช้เครื่องมือ}{4}{6} \\
		\ganttbar{5.วิเคราะห์และออกแบบ}{4}{5} \\
		\ganttbar{6.เขียนโปรแกรม}{6}{10} \\
		\ganttbar{7.ทดสอบและแก้ปัญหา}{6}{10} \\
		\ganttbar{8.จัดทำเอกสาร}{9}{12} \\
	\end{ganttchart}
	\label{tab:ganttchart}
\end{table}
%\end{landscape}
%TODO แก้เทมเพลตเอาชื่อตารางไว้ด้านบน
