\chapter{การวิเคราะห์และออกแบบระบบ}

การวิเคราะห์และออกแบบระบบก่อนดำเนินการจริงเป็นอีกหนึ่งขั้นตอนที่มีความสำคัญมาก เพราะการวิเคราะห์และออกแบบระบบนั้นเป็นการกระทำที่ทำให้ผู้พัฒนาเห็นรายละเอียดส่วนย่อยของงานทั้งหมด เพิ่มประสิทธิภาพในการวางแผน การทำงาน และยังช่วยลดปัญหาที่อาจจะเกิดขึ้นในระหว่างพัฒนา เพื่อให้ระบบมีความสมบูรณ์มากยิ่งขึ้น เนื่องจากการวิเคราะห์และออกแบบระบบนั้นจะช่วยให้ให้บริการ จัดการทรัพยากรได้อย่างคุ้มค่าและตรงตามความต้องการของระบบ

การวิเคราะห์และออกแบบแอปพลิเคชันสูงวัยมายเฟรนด์ ในบทนี้จะแบ่งออกเป็น 6 ขั้นตอนเพื่อให้เห็นการดำเนินงานอย่างมีระบบ ในหัวข้อแรกจะนำเสนอภาพรวมของระบบ ก่อนจะนำเสนอเอกสารแสดงความต้องการของระบบซึ่งจะทำให้เห็นที่มาของเพจต่าง ๆ ในขั้นตอนของการออกแบบในหัวข้อที่สาม ส่วนหัวข้อที่เหลือจะแสดงแผนภาพการการทำงานของระบบโดยใช้ UML diagram ซึ่งประกอบไปด้วย Use Case, Class และ Sequence Diagram เพื่อแสดงรายละเอียดของระบบก่อนนำไปเขียนคำสั่งด้วยภาษาโปรแกรมในบทต่อไป

\begin{enumerate}[label=3.\arabic*]
	\item โครงสร้างภาพรวมของระบบ (System Architecture)
	\item การวิเคราะห์ความต้องการของระบบ (System Requirements)
	\item การออกแบบส่วนติดต่อผู้ใช้ (User Interface Design)
	\item แผนภาพไดอะแกรม
	\begin{enumerate}[label=3.3.\arabic*]
		\item ยูสเคสไดอะแกรม (Use Case Diagram)
		\item คลาสไดอะแกรม (Class Diagram)
		\item ซีเควสไดอะแกรม (Sequence Diagram)
	\end{enumerate}	
\end{enumerate}	

%  เริ่มต้น ภาพรวมของระบบ

\section{โครงสร้างภาพรวมของระบบ}
    ความหมายของ System Architecture \cite{architecture} หมายถึง กรอบโครงสร้างของระบบที่อธิบายความสัมพันธ์ขององค์ประกอบต่าง ๆ ไปจนถึงขั้นการเชื่อมต่อกันของระบบย่อยต่าง ๆ โดยจัดกลุ่มองค์ประกอบไว้ในหลาย ๆ ลักษณะเพื่อให้ผู้เกี่ยวข้อง (Stakeholder) จากพื้นฐานสาขาอาชีพที่แตกต่าง กันสามารถทำความเข้าใจได้ง่าย เช่น การจัดแบ่งองค์ประกอบตามลักษณะการทำงานของระบบ (functional components) เป็นต้น
    
    การออกแบบ System architecture แสดงภาพรวมและเทคโนโลยีของแอปพลิเคชันสูงวัยมายเฟรนด์ มีรายละเอียดดังรูปที่ \ref{Fig:architecture}
   	\begin{figure}[H]
   		\centering
   		\includegraphics[width=\textwidth]{Figures/3/architecture/structure_app}
   		\caption{System architecture แอปพลิเคชันสูงวัยมายเฟรนด์}
   		\label{Fig:architecture}
   	\end{figure}
   
   จากรูปที่ \ref{Fig:architecture} สามารถอธิบายโครงสร้างและเทคโนโลยีของระบบโดยแบ่งเป็น 3 ส่วนหลัก ดังนี้
   \begin{enumerate}
   	\item Database
   	 ระบบใช้บริการฐานข้อมูลแบบ NoSQL ของไฟร์เบสชื่อ Cloud Firestore
	  \item Server
	   กระบวนการทำงานในส่วนของเซิฟเวอร์ (server) แบ่งเป็น 3 ส่วนได้แก่
	   \begin{itemize}
	   	\item Dialogflow เป็น Platform ไว้สำหรับจัดการการโต้ตอบอัตโนมัติหรือแชทบอท
	   	\item Google Maps Api เป็น Api ของ Google ไว้สำหรับเรียกใช้งาน Google Maps เพื่อใช้งานแผนที่
	   	\item ชุดบริการไฟร์เบส Api ใช้สำหรับการทำงานกับบริการต่าง ๆ ของไฟร์เบสบนแฟลตฟอร์มที่แตกต่างกัน เช่น Authentication ใช้สำหรับการจัดการข้อมูลผู้ใช้หรือไฟร์เบส Storage ที่ใช้สำหรับจัดเก็บไฟล์เอกสารและรูปภาพต่าง ๆ เป็นต้น
	   \end{itemize}
	   \item Client
	    App Moblie แอปพลิเคชันทำงานบนอุปกรณ์พกพาสามารถใช้ได้ทั้งแอนดรอยด์และไอโอเอส
   \end{enumerate}

%  สิ้นสุด ภาพรวมของระบบ

%  เริ่มต้น การวิเคราะห์ความต้องการของระบบ

\section{การวิเคราะห์ความต้องการของระบบ}
\subsection{ความต้องการหลักของระบบ (Functional Requirements)}
	แอปพลิเคชันสูงวัยมายเฟรนด์ แบ่งความสามารถของระบบดังนี้
	\begin{enumerate}
		\item ผู้ใช้งาน
			\begin{itemize}[label={--}]
				\item สามารถสมัครสมาชิกและเข้าสู่ระบบได้
				\item สามารถดู สร้าง แก้ไข ลบ โพสท์ได้
				\item สามารคุยโต้ตอบกับแชทบอทได้
				\item สามารถดูตำแหน่งของคนในครอบครัวได้
				\item สามารถส่งแจ้งเตือนรูปแบบการโทร หรือข้อความ ไปหาคนในครอบครัวได้
				\item สามารถเพิ่ม ลบ เพื่อนได้
				\item สามารถรับการแจ้งเตือนการทานยาได้
				\item สามารถดูและแก้ไขข้อมูลส่วนตัวได้
			\end{itemize}
	\end{enumerate}

\subsection{Non-functional Requirements}
\begin{enumerate}
		\item แอปพลิเคชัน
		\begin{itemize}[label={--}]
			\item แชทบอทสามารถตอบโต้ได้ตรงประเด็น
			\item มีความรวดเร็วในการกดไลท์
			\item สามารถใช้งานได้รวดเร็ว
			\item สามารถดูตำแหน่งของคนในครอบครัวแบบ Realtime
			\item สามารถแจ้งเตือนการทานยาได้เมื่อปิดแอปพลิเคชันอย่างสมบูรณ์
		\end{itemize}
	\end{enumerate}
%  สิ้นสุด การวิเคราะห์ความต้องการของระบบ
	
%  เริ่มต้น User Interface Design

\section{การออกแบบส่วนติดต่อผู้ใช้}
ในการออกแบบส่วนติดต่อผู้ใช้ของแอปพลิเคชันสูงวัยมายเฟรนด์ ประกอบด้วยส่วนต่าง ๆ ดังนี้
	\subsection{หน้าเข้าสู่ระบบ}
		\begin{figure}[H]
			\centering
			\includegraphics[width=0.6\textwidth]{Figures/3/UI/login}
			\caption{หน้าจอเข้าสู่ระบบ}
			\label{Fig:หน้าจอเข้าสู่ระบบ}
		\end{figure}
		จากภาพที่ \ref{Fig:หน้าจอเข้าสู่ระบบ} การออกแบบหน้าจอเข้าสู่ระบบประกอบไปด้วย 6 ส่วนดังนี้
		\begin{itemize}
			\item ส่วนที่ 1 โลโก้ของแอปพลิเคชัน
			\item ส่วนที่ 2 ช่องสำหรับกรอกชื่อผู้ใช้
			\item ส่วนที่ 3 ช่องสำหรับกรอกรหัสผ่าน
			\item ส่วนที่ 4 ปุ่มสำหรับเข้าสู่ระบบ
			\item ส่วนที่ 5 ปุ่มสำหรับสมัครสมาชิก
			\item ส่วนที่ 6 ข้อความสำหรับกดเมื่อลืมรหัสผ่าน
		\end{itemize}

		\begin{figure}[H]
			\centering
			\includegraphics[width=0.6\textwidth]{Figures/3/UI/main}
			\caption{หน้าจอหลัก}
			\label{Fig:หน้าจอหลัก}
		\end{figure}
		จากภาพที่ \ref{Fig:หน้าจอหลัก} การออกแบบหลักประกอบไปด้วย 13 ส่วนดังนี้
		\begin{itemize}
			\item ส่วนที่ 1 ข้อมูลของผู้ใช้ประกอบไปด้วยรูปประจำตัวและชื่อผู้ใช้
			\item ส่วนที่ 2 ปุ่มสำหรับออกจากระบบ
			\item ส่วนที่ 3 ปุ่มสำหรับไปยังหน้าปู่จอห์นที่เป็นแชทบอท
			\item ส่วนที่ 4 ปุ่มสำหรับไปยังหน้ากระดานข่าว
			\item ส่วนที่ 5 ปุ่มสำหรับไปยังหน้าครอบครัว
			\item ส่วนที่ 6 ปุ่มสำหรับไปยังหน้าตำแหน่งของครอบครัว
			\item ส่วนที่ 7 ปุ่มสำหรับไปยังหน้าฉุกเฉิน
			\item ส่วนที่ 8 ปุ่มไปยังหน้าวิธีใช้งาน
			\item ส่วนที่ 9 ปุ่มไปยังหน้าหลักเราจะอยู่หน้านี้ทุกครั้งเมื่อเราเข้าสู่แอปพลิเคชัน
			\item ส่วนที่ 10 ปุ่มสำหรับไปยังหน้าเพื่อนและครอบครัว
			\item ส่วนที่ 11 ปุ่มเพื่อโพสท์ข้อความหรือรูปภาพ
			\item ส่วนที่ 12 ปุ่มสำหรับไปหน้าตั้งค่า
			\item ส่วนที่ 13 ปุ่มสำหรับไปยังหน้าโปรไฟล์หรือข้อมูลส่วนตัว
		\end{itemize}

		\begin{figure}[H]
			\centering
			\includegraphics[width=0.6\textwidth]{Figures/3/UI/chatbot}
			\caption{หน้าแชทบอทคุยกับปู่จอห์น}
			\label{Fig:แชทบอท}
		\end{figure}
		จากภาพที่ \ref{Fig:แชทบอท} การออกแบบหลักประกอบไปด้วย 8 ส่วนดังนี้
		\begin{itemize}
			\item ส่วนที่ 1 ปุ่มลูกศรกลับไปยังหน้าจอหลัก
			\item ส่วนที่ 2 รูปประจำตัวของปู่จอห์น
			\item ส่วนที่ 3 ข้อความที่ปู่จอห์นตอบกลับผู้ใช้
			\item ส่วนที่ 4 ข้อความของผู้ใช้
			\item ส่วนที่ 5 รูปประจำตัวของผู้ใช้
			\item ส่วนที่ 6 ปุ่มสำหรับพิมพ์ข้อความด้วยเสียง (Speech to text)
			\item ส่วนที่ 7 ช่องสำหรับกรอกข้อความที่จะแสดงในส่วนที่ 4
			\item ส่วนที่ 8 ปุ่มสำหรับยืนยันข้อความส่วนที่ 7
		\end{itemize}

		\begin{figure}[H]
			\centering
			\includegraphics[width=0.6\textwidth]{Figures/3/UI/board}
			\caption{หน้าจอกระดานข่าว}
			\label{Fig:กระดานข่าว}
		\end{figure}
		จากภาพที่ \ref{Fig:กระดานข่าว} การออกแบบหลักประกอบไปด้วย 11 ส่วนดังนี้
		\begin{itemize}
			\item ส่วนที่ 1 ปุ่มสำหรับเพิ่มรูปภาพสำหรับโพสท์มีสองส่วนคือ ถ่ายรูปและเลือกรูปจากแกลอรี่
			\item ส่วนที่ 2 ช่องสำหรับกรอกข้อความที่จะโพสท์
			\item ส่วนที่ 3 ปุ่มสำหรับยืนยันการโพสท์หลังจากคลิกจะมีการเลือกประเภทได้ 3 แบบคือ กีฬา, ศาสนา, ดนตรี
			\item ส่วนที่ 4 ลิสท์สำหรับเลือกแสดงเฉพาะประเภทที่ต้องการมี 4 แบบได้แก่ ทั้งหมด, กีฬา, ศาสนา, ดนตรี
			\item ส่วนที่ 5 ปุ่มสำหรับแก้ไขโพสท์
			\item ส่วนที่ 6 ปุ่มสำหรับลบโพสท์
			\item ส่วนที่ 7 ข้อความโพสท์ของผู้ใช้
			\item ส่วนที่ 8 รูปภาพโพสท์ของผู้ใช้
			\item ส่วนที่ 9 ปุ่มสำหรับกดถูกใจ
			\item ส่วนที่ 10 ปุ่มสำหรับแสดงความคิดเห็น
			\item ส่วนที่ 11 ข้อความแสดงระยะเวลาหลังจากโพสท์ถูกสร้าง
		\end{itemize}

		\begin{figure}[H]
			\centering
			\includegraphics[width=0.6\textwidth]{Figures/3/UI/family}
			\caption{หน้าครอบครัว}
			\label{Fig:ครอบครัว}
		\end{figure}
		จากภาพที่ \ref{Fig:ครอบครัว} การออกแบบหลักประกอบไปด้วย 5 ส่วนดังนี้
		\begin{itemize}
			\item ส่วนที่ 1 ปุ่มสำหรับเพิ่มสมาชิกในครอบครัว
			\item ส่วนที่ 2 รูปภาพของสมาชิกในครอบครัว
			\item ส่วนที่ 3 ชื่อของสมาชิกในครอบครัว
			\item ส่วนที่ 4 สถานะของครอบครัว
			\item ส่วนที่ 5 ปุ่มสำหรับจัดการสมาชิกครอบครัวได้แก่ แก้ไขสถานะ ลบสมาชิก
		\end{itemize}

		\begin{figure}[H]
			\centering
			\includegraphics[width=0.6\textwidth]{Figures/3/UI/maps}
			\caption{หน้าแสดงตำแหน่งของครอบครัว}
			\label{Fig:แผนที่}
		\end{figure}
		จากภาพที่ \ref{Fig:แผนที่} การออกแบบหลักประกอบไปด้วย 2 ส่วนดังนี้
		\begin{itemize}
			\item ส่วนที่ 1 แสดงตำแหน่งของผู้ใช้
			\item ส่วนที่ 2 แสดงตำแหน่งของสมาชิกในครอบครัว
		\end{itemize}

		\begin{figure}[H]
			\centering
			\includegraphics[width=0.6\textwidth]{Figures/3/UI/help}
			\caption{หน้าฉุกเฉิน}
			\label{Fig:ฉุกเฉิน}
		\end{figure}
		จากภาพที่ \ref{Fig:ฉุกเฉิน} การออกแบบหลักประกอบไปด้วย 2 ส่วนดังนี้
		\begin{itemize}
			\item ส่วนที่ 1 ปุ่มสำหรับแสดงเสียงเพื่อขอความช่วยเหลือ
			\item ส่วนที่ 2 ปุ่มสำหรับไปยังหน้าเลือกครอบครัวเพื่อขอความช่วยเหลือจากครอบครัว
		\end{itemize}

		\begin{figure}[H]
			\centering
			\includegraphics[width=0.6\textwidth]{Figures/3/UI/helpfamily}
			\caption{หน้าเลือกครอบครัว}
			\label{Fig:เลือกครอบครัว}
		\end{figure}
		จากภาพที่ \ref{Fig:เลือกครอบครัว} การออกแบบหลักประกอบไปด้วย 5 ส่วนดังนี้
		\begin{itemize}
			\item ส่วนที่ 1 รูปประจำตัวของสมาชิกในครอบครัว
			\item ส่วนที่ 2 ชื่อของสมาชิกในครอบครัว
			\item ส่วนที่ 3 สถานะของครอบครัว
			\item ส่วนที่ 4 ปุ่มสำหรับโทรไปยังหมายเลขของสมาชิกในครอบครัว
			\item ส่วนที่ 5 ปุ่มสำหรับส่งข้อความไปยังสมาชิกในครอบครัวมีข้อความดังนี้ "ช่วยด้วย !!! นี่ [ชื่อผู้ใช้] เอง ตอนนี้มีปัญหาช่วยติดต่อกลับมาที่ [เบอร์ผู้ใช้] ด้วยนะ ด่วนๆ"
		\end{itemize}

		\begin{figure}[H]
			\centering
			\includegraphics[width=0.6\textwidth]{Figures/3/UI/howto}
			\caption{หน้าวิธีใช้งาน}
			\label{Fig:วิธีใช้งาน}
		\end{figure}
		จากภาพที่ \ref{Fig:วิธีใช้งาน} การออกแบบหลักประกอบไปด้วย 4 ส่วนดังนี้
		\begin{itemize}
			\item ส่วนที่ 1 รูปภาพตัวอย่างแสดงหน้าจอนั้นๆ
			\item ส่วนที่ 2 ชือของหน้าจอนั้นๆ
			\item ส่วนที่ 3 รายละเอียดเพิ่มเติมของหน้าจอนั้นๆ
			\item ส่วนที่ 4 จุดสำหรับแสดงตำแหน่งที่เราอยู่ ณ ปัจจุบัน
		\end{itemize}

		\begin{figure}[H]
			\centering
			\includegraphics[width=0.6\textwidth]{Figures/3/UI/group}
			\caption{หน้าจอแสดงกลุ่ม}
			\label{Fig:กลุ่ม}
		\end{figure}
		จากภาพที่ \ref{Fig:กลุ่ม} การออกแบบหลักประกอบไปด้วย 7 ส่วนดังนี้
		\begin{itemize}
			\item ส่วนที่ 1 ปุ่มสำหรับเพิ่มเพื่อน
			\item ส่วนที่ 2 ปุ่มสำหรับไปยังหน้าแสดงกลุ่ม
			\item ส่วนที่ 3 ปุ่มสำหรับไปยังหน้าจอแสดงเพื่อน
			\item ส่วนที่ 4 รูปภาพของกลุ่มที่เราเป็นคนสร้าง
			\item ส่วนที่ 5 ชื่อของกลุ่มที่เราเป็นคนสร้าง
			\item ส่วนที่ 6 รูปภาพของกลุ่มที่เราเป็นสมาชิก
			\item ส่วนที่ 7 ชื่อของกลุ่มที่เราเป็นสมาชิก
		\end{itemize}

		\begin{figure}[H]
			\centering
			\includegraphics[width=0.6\textwidth]{Figures/3/UI/friend}
			\caption{หน้าจอแสดงเพื่อน}
			\label{Fig:เพื่อน}
		\end{figure}
		จากภาพที่ \ref{Fig:เพื่อน} การออกแบบหลักประกอบไปด้วย 8 ส่วนดังนี้
		\begin{itemize}
			\item ส่วนที่ 1 ปุ่มสำหรับเพิ่มเพื่อน
			\item ส่วนที่ 2 ปุ่มสำหรับไปยังหน้าแสดงกลุ่ม
			\item ส่วนที่ 3 ปุ่มสำหรับไปยังหน้าจอแสดงเพื่อน
			\item ส่วนที่ 4 รูปภาพของเพื่อนที่รอการยืนยัน
			\item ส่วนที่ 5 ชื่อของเพื่อนที่รอการยืนยัน
			\item ส่วนที่ 6 ปุ่มสำหรับยกเลิกคำขอร้องเป็นเพื่อน
			\item ส่วนที่ 7 รูปภาพของเพื่อน
			\item ส่วนที่ 8 ชื่อของเพื่อน
			\item ส่วนที่ 9 ปุ่มสำหรับแชทกับเพื่อน
			\item ส่วนที่ 10 ปุ่มสำหรับลบเพื่อน
		\end{itemize}

		\begin{figure}[H]
			\centering
			\includegraphics[width=0.6\textwidth]{Figures/3/UI/notification}
			\caption{หน้าแสดงรายการแจ้งเตือนการทานยา}
			\label{Fig:รายการทานยา}
		\end{figure}
		จากภาพที่ \ref{Fig:รายการทานยา} การออกแบบหลักประกอบไปด้วย 4 ส่วนดังนี้
		\begin{itemize}
			\item ส่วนที่ 1 ข้อความแสดงชื่อของยา
			\item ส่วนที่ 2 ข้อความแสดงจำนวนการรับประทานต่อครั้ง
			\item ส่วนที่ 3 ข้อความแสดงเวลาในการแจ้งเตือน
			\item ส่วนที่ 4 ปุ่มสำหรับเพิ่มการแจ้งเตือนการทานยา
		\end{itemize}

		\begin{figure}[H]
			\centering
			\includegraphics[width=0.6\textwidth]{Figures/3/UI/addnotification}
			\caption{หน้าแสดงการเพิ่มการแจ้งเตือนทานยา}
			\label{Fig:เพิ่มทานยา}
		\end{figure}
		จากภาพที่ \ref{Fig:เพิ่มทานยา} การออกแบบหลักประกอบไปด้วย 5 ส่วนดังนี้
		\begin{itemize}
			\item ส่วนที่ 1 ลิสท์สำหรับเลือกประเภทการรับประทานมี 2 ประเภทได้แก่ สำหรับรับประทานและสำหรับฉีด
			\item ส่วนที่ 2 ช่องสำหรับกรอกชื่อยา ลักษณะของยา หรือสีของยา
			\item ส่วนที่ 3 ช่องสำหรับกรอกจำนวนครั้งในการทานยา
			\item ส่วนที่ 4 ลิสท์สำหรับกำหนดเวลาการแจ้งเตือนการทานยา
			\item ส่วนที่ 5 ปุ่มสำหรับยืนยันการเพิ่มการแจ้งเตือนทานยา
		\end{itemize}

		\begin{figure}[H]
			\centering
			\includegraphics[width=0.6\textwidth]{Figures/3/UI/profile}
			\caption{หน้าจอแสดงข้อมูลส่วนตัวของผู้ใช้}
			\label{Fig:โปรไฟล์}
		\end{figure}
		จากภาพที่ \ref{Fig:โปรไฟล์} การออกแบบหลักประกอบไปด้วย 4 ส่วนดังนี้
		\begin{itemize}
			\item ส่วนที่ 1 รูปประจำตัวของผู้ใช้
			\item ส่วนที่ 2 ปุ่มสำหรับเปลี่ยนรูปประจำตัว
			\item ส่วนที่ 3 แสดงจำนวนโพสท์ เพื่อน ครอบครัว และปุ่มแก้ไขข้อมูลส่วนตัว
			\item ส่วนที่ 4 โพสท์ของผู้ใช้
		\end{itemize}
\newpage

%  สิ้นสุด User Interface Design

%  เริ่มต้น Diagram

\section{แผนภาพไดอะแกรม}

%  เริ่มต้น Use Case Diagram

\subsection{แผนภาพยูสเคส (Use Case Diagram)}
	Use Case Diagram เป็นแผนผังเพื่อแสดงฟังก์ชันแสดงการทำงานของระบบโดยรวม แสดงส่วนประกอบในระบบและกิจกรรมที่เกิดขึ้นในระบบซึ่งในระบบระบบกองทุนเงินให้กู้ยืมเพื่อการศึกษา คณะวิทยาศาสตร์ มหาวิทยาลัยอุบลราชธานี ผู้ใช้จำเป็นต้องเข้าสู่ระบบเพื่อใช้งานระบบ สัญลักษณ์ที่ใช้ในการเขียน Use Case Diagram แสดงในตารางที่ \ref{tab:use-case2}
	\begin{table}[H]
		\caption{สัญลักษณ์ของ Use case Diagram}
		\label{tab:use-case2}
		\begin{tabular}{|c|p{10cm}|}
		\hline
		\textbf{สัญลักษณ์} & \multicolumn{1}{c|}{\textbf{การใช้งาน}} \\ \hline
		\raisebox{-\totalheight}{Use case}
		& \setstretch{1.5} {Use case คือส่วนย่อยของระบบงาน แทนด้วยวงรีและชื่อของ Use case ภายในวงรี} \\ \hline
		\raisebox{-\totalheight}{\includegraphics[height=1.5cm]{Figures/table/use-case/2}}
		& \setstretch{1.5} {Actor คือบุคคลหรือระบบงานอื่นที่ใช้งานระบบหรือได้รับประโยชน์จากระบบซึ่งอยู่ภายนอกระบบ แทนด้วยรูปคนและมีชื่อบทบาทการใช้งานระบบ} \\ \hline
		\raisebox{-\totalheight}{\includegraphics[width=3cm]{Figures/table/use-case/3}}
		& \setstretch{1.5} {เส้นตรงที่แสดงถึงการใช้งาน Use case ของผู้กระทำ} \\ \hline
		\raisebox{-\totalheight}{\includegraphics[width=0.3\textwidth]{Figures/table/use-case/4}}
		& \setstretch{1.5} {กรอบสี่เหลี่ยมแสดงถึงขอบเขตของระบบโดยแสดงชื่อระบบภายในหรือด้านบนกรอกสี่เหลี่ยม Use case อยู่ภายในกรอบสี่เหลี่ยม และ actor อยู่ภายนอกกรอบสี่เหลี่ยม} \\ \hline
		\raisebox{-\totalheight}{\includegraphics[width=0.3\textwidth]{Figures/table/use-case/5}}
		& \setstretch{1.5} {ความสัมพันธ์แบบ <<includes>> แสดงว่า Use case หนึ่งดำเนินการตามขั้นตอนของ Use case อื่น โดยแทนด้วยสัณลักษณ์ลูกศรเส้นประ ซึ่ง Use case ที่หางลูกศรเรียกใช้งาน Use case ที่หัวลูกศรทุกครั้งที่มีการทำงาน} \\ \hline
		\raisebox{-\totalheight}{\includegraphics[width=0.3\textwidth]{Figures/table/use-case/6}}
		& \setstretch{1.5} {ความสัมพันธ์แบบ <<extend>> แสดงว่า Use case หนึ่งดำเนินการตามขั้นตอนของ Use case อื่น โดยแทนด้วยสัญลักษณ์ลูกศรเส้นประ ซึ่ง Use case ที่หัวลูกศรเรียกใช้งาน Use case ที่หางลูกศร แต่การใช้งานไม่จำเป็นต้องเกิดขึ้นทุกครั้งขึ้นอยู่กับเงื่อนไขระหว่างการทำงาน} \\ \hline
		\end{tabular}
	\end{table}

	\begin{figure}[H]
		\includegraphics[width=1.1\textwidth]{Figures/3/Usecase/Usecase}
		\caption{Use Case Diagram ของแอปพลิเคชันสูงวัยมายเฟรนด์}
		\label{Fig:usecase}
	\end{figure}
	
	\begin{table}[H]
		\centering
		\caption{อธิบาย Use Case หน้าที่ของแอปพลิเคชันสูงวัยมายเฟรนด์ในภาพที่ \ref{Fig:usecase}}
		\label{tab:usecase}
		\resizebox{\totalheight}{!}{\columnwidth}{%
			\begin{tabular}{|c|p{10cm}|}
				\hline
				\multicolumn{1}{|c|}{\textbf{Use Case}} & \multicolumn{1}{c|}{\textbf{คำอธิบาย}} \\ \hline
				เข้าสู่ระบบ & ใช้สำหรับให้ผู้ใช้เข้าใช้งานแอปพลิเคชัน \\ \hline
				ออกจากระบบ & ผู้ใช้ที่เข้าสู่ระบบแล้วสามารถออกจากระบบ \\ \hline
				จัดการข้อมูลส่วนตัว & ใช้งานเพื่อให้ผู้ใช้ แก้ไขข้อมูลที่ต้องการเปลี่ยนแปลง \\ \hline
				ดู เพิ่ม แก้ไข หรือลบโพสท์ & ใช้งานเพื่อให้ผู้ใช้สามารถดู เพิ่ม แก้ไข หรือลบโพสท์ของตัวเอง \\ \hline
				โทรหรือส่งข้อความหาสมาชิกในครอบครัว & ใช้สำหรับให้ผู้ใช้สามารถโทรหรือส่งข้อความหาคนในครอบครัว และสามารถเปิดเสียงฉุกเฉิน \\ \hline
				ดูตำแหน่งสมาชิกในครอบครัว & ใช้งานเพื่อให้ผู้ใช้สามารถดูตำแหน่งปัจจุบันของสมาชิกในครอบครัวด้วยแผนที่ \\ \hline
				จัดการครอบครัว & ใช้สำหรับเพิ่ม แก้ไข หรือลบข้อมูลสมาชิกครอบครัว \\ \hline
				จัดการกลุ่ม & ใช้สำหรับเพิ่ม แชท แก้ไข หรือลบกลุ่มได้ และเพิ่ม แก้ไข หรือลบสมาชิกในกลุ่มได้ \\ \hline
				จัดการเพื่อน & ใช้สำหรับยืนยันคำร้องขอเป็นเพื่อน รวมถึงพูดคุย เพิ่มและลบเพื่อน \\ \hline
				เพิ่ม หรือลบการแจ้งเตือน & ใช้สำหรับเพิ่ม หรือลบการแจ้งเตือนการทานยา \\ \hline
				ดูวิธีการใช้งาน & ใช้สำหรับให้ผู้ใช้ดูรายละเอียดของหน้าต่าง ๆ ในแอปพลิเคชัน \\ \hline
				คุยกับแชทบอท & ใช้งานเพื่อให้ผู้ใช้สามารถถามเรื่องโรคหรือเรื่องที่สนใจ \\ \hline
			\end{tabular}%
		}
	\end{table}

	\begin{table}[H]
		\centering
		\caption{Use Case เข้าสู่ระบบ}
		\label{tab:usecase}
		\resizebox{\totalheight}{!}{\textwidth}{%
			\begin{tabular}{|p{10cm}|p{10cm}|}
				\hline
				\textbf{Use Case Title : เข้าสู่ระบบ} & \multicolumn{1}{|c|}{\textbf{Use case Id : 1 }} \\ \hline
				\multicolumn{2}{|p{\linewidth}|}{Primary Actor : ผู้ใช้งาน} \\ \hline
			    \multicolumn{2}{|p{\linewidth}|}{Stakeholder Actor : -} \\ \hline
			    \multicolumn{2}{|p{\linewidth}|}{Main Flow : สามารถเข้าสู่ระบบเพื่อใช้งานแอปพลิเคชัน} \\ \hline
				\multicolumn{2}{|p{\linewidth}|}{Exceptional Flow ที่ 1 : หากผู้ใช้ไม่เชื่อมต่ออินเทอร์เน็ต จะไม่สามารถเข้าสู่ระบบได้} \\ \hline
			\end{tabular}%
		}
	\end{table}
	\begin{table}[H]
		\centering
		\caption{Use Case ออกจากระบบ}
		\label{tab:usecase}
		\resizebox{\totalheight}{!}{\textwidth}{%
			\begin{tabular}{|p{10cm}|p{10cm}|}
				\hline
				\textbf{Use Case Title : ออกจากระบบ} & \multicolumn{1}{|c|}{\textbf{Use case Id : 2 }} \\ \hline
				\multicolumn{2}{|l|}{Primary Actor : ผู้ใช้งาน} \\ \hline
				\multicolumn{2}{|l|}{Stakeholder Actor : -} \\ \hline
				\multicolumn{2}{|p{\linewidth}|}{Main Flow : สามารถออกจากระบบได้} \\ \hline
				\multicolumn{2}{|p{\linewidth}|}{Exceptional Flow ที่ 1 : หากยังไม่เข้าสู่ระบบจะไม่สามารถออกจากระบบได้} \\ \hline
			\end{tabular}%
		}
	\end{table}

	\begin{table}[H]
		\centering
		\caption{Use Case จัดการข้อมูลส่วนตัว}
		\label{tab:usecase}
		\resizebox{\totalheight}{!}{\textwidth}{%
			\begin{tabular}{|p{10cm}|p{10cm}|}
				\hline
				\textbf{Use Case Title : จัดการข้อมูลส่วนตัว} & \multicolumn{1}{|c|}{\textbf{Use case Id : 3 }} \\ \hline
				\multicolumn{2}{|l|}{Primary Actor : ผู้ใช้งาน} \\ \hline
				\multicolumn{2}{|l|}{Stakeholder Actor : -} \\ \hline
				\multicolumn{2}{|p{\linewidth}|}{Main Flow : สามารถดู เพิ่ม แก้ไขข้อมูลส่วนตัวได้} \\ \hline
				\multicolumn{2}{|p{\linewidth}|}{Exceptional Flow ที่ 1 : หากผู้ใช้ไม่เข้าสู่ระบบจะไม่สามารถดู เพิ่ม แก้ไขข้อมูลส่วนตัวได้} \\ \hline
			\end{tabular}%
		}
		\end{table}
		
		\begin{table}[H]
			\centering
			\caption{Use Case ดู เพิ่ม แก้ไข หรือลบโพสท์}
			\label{tab:usecase}
			\resizebox{\totalheight}{!}{\textwidth}{%
				\begin{tabular}{|p{10cm}|p{10cm}|}
					\hline
					\textbf{Use Case Title : ดู เพิ่ม แก้ไข หรือลบโพสท์} & \multicolumn{1}{c|}{\textbf{Use case Id : 4 }} \\ \hline
					\multicolumn{2}{|l|}{Primary Actor : ผู้ใช้งาน} \\ \hline
					\multicolumn{2}{|l|}{Stakeholder Actor : -} \\ \hline
					\multicolumn{2}{|p{\linewidth}|}{Main Flow : สามารถดู เพิ่ม แก้ไข หรือลบโพสท์ได้} \\ \hline
					\multicolumn{2}{|p{\linewidth}|}{Exceptional Flow ที่ 1 : หากผู้ใช้ไม่เข้าสู่ระบบจะไม่สามารถดู เพิ่ม แก้ไข หรือลบโพสท์} \\ \hline
				\end{tabular}%
			}
		\end{table}
		
		\begin{table}[H]
			\centering
			\caption{Use Case โทรหรือส่งข้อความหาสมาชิกครอบครัว}
			\label{tab:usecase}
			\resizebox{\totalheight}{!}{\textwidth}{%
				\begin{tabular}{|p{10cm}|p{10cm}|}
					\hline
					\textbf{Use Case Title : โทรหรือส่งข้อความหาสมาชิกครอบครัว} & \multicolumn{1}{c|}{\textbf{Use case Id : 5 }} \\ \hline
					\multicolumn{2}{|l|}{Primary Actor : ผู้ใช้งาน} \\ \hline
					\multicolumn{2}{|l|}{Stakeholder Actor : -} \\ \hline
					\multicolumn{2}{|p{\linewidth}|}{Main Flow : สามารถโทรหรือส่งข้อความหาสมาชิกครอบครัว และสามารถเปิดเสียงฉุกเฉินได้} \\ \hline
					\multicolumn{2}{|p{\linewidth}|}{Exceptional Flow ที่ 1 : หากผู้ใช้ไม่เข้าสู่ระบบจะไม่สามารถโทรหรือส่งข้อความหาสมาชิกครอบครัว และสามารถเปิดเสียงฉุกเฉินได้} \\ \hline
				\end{tabular}%
			}
		\end{table}
	
		\begin{table}[H]
			\centering
			\caption{Use Case ดูตำแหน่งสมาชิกครอบครัว}
			\label{tab:usecase}
			\resizebox{\totalheight}{!}{\textwidth}{%
				\begin{tabular}{|p{10cm}|p{10cm}|}
					\hline
					\textbf{Use Case Title : ดูตำแหน่งสมาชิกครอบครัว} & \multicolumn{1}{c|}{\textbf{Use case Id : 6 }} \\ \hline
					\multicolumn{2}{|l|}{Primary Actor : ผู้ใช้งาน} \\ \hline
					\multicolumn{2}{|l|}{Stakeholder Actor : -} \\ \hline
					\multicolumn{2}{|p{\linewidth}|}{Main Flow : สามารถดูตำแหน่งสมาชิกในครอบครัวได้ด้วยแผนที่} \\ \hline
					\multicolumn{2}{|p{\linewidth}|}{Exceptional Flow ที่ 1 : หากผู้ใช้ไม่เชื่อมต่ออินเทอร์เน็ต จะไม่สามารถดูตำแหน่งสมาชิกในครอบครัวได้} \\ \hline
				\end{tabular}%
			}
		\end{table}	
		
		\begin{table}[H]
			\centering
			\caption{Use Case จัดการครอบครัว}
			\label{tab:usecase}
			\resizebox{\totalheight}{!}{\textwidth}{%
				\begin{tabular}{|p{10cm}|p{10cm}|}
					\hline
					\textbf{Use Case Title : จัดการครอบครัว} & \multicolumn{1}{c|}{\textbf{Use case Id : 7 }} \\ \hline
					\multicolumn{2}{|l|}{Primary Actor : ผู้ใช้งาน} \\ \hline
					\multicolumn{2}{|l|}{Stakeholder Actor : -} \\ \hline
					\multicolumn{2}{|p{\linewidth}|}{Main Flow : สามารถเพิ่ม แก้ไข ลบสมาชิกในครอบครัวได้} \\ \hline
					\multicolumn{2}{|p{\linewidth}|}{Exceptional Flow ที่ 1 : หากผู้ใช้ไม่เข้าสู่ระบบจะไม่สามารถเพิ่ม แก้ไข ลบสมาชิกในครอบครัวได้} \\ \hline
					\multicolumn{2}{|p{\linewidth}|}{Exceptional Flow ที่ 2 : หากผู้ใช้ไม่มีผู้ใช้งานเป็นเพื่อน จะไม่สามารถเพิ่มเข้ามาในครอบครัวได้} \\ \hline
				\end{tabular}%
			}
		\end{table}	

		\begin{table}[H]
			\centering
			\caption{Use Case จัดการกลุ่ม}
			\label{tab:usecase}
			\resizebox{\totalheight}{!}{\textwidth}{%
				\begin{tabular}{|p{10cm}|p{10cm}|}
					\hline
					\textbf{Use Case Title : จัดการกลุ่ม} & \multicolumn{1}{c|}{\textbf{Use case Id : 8 }} \\ \hline
					\multicolumn{2}{|l|}{Primary Actor : ผู้ใช้งาน} \\ \hline
					\multicolumn{2}{|l|}{Stakeholder Actor : -} \\ \hline
					\multicolumn{2}{|p{\linewidth}|}{Main Flow : สามารถเพิ่ม พูดคุย แก้ไข หรือลบกลุ่ม เพิ่ม แก้ไข หรือลบสมาชิกในกลุ่มได้} \\ \hline
					\multicolumn{2}{|p{\linewidth}|}{Exceptional Flow ที่ 1 : หากผู้ใช้ไม่เข้าสู่ระบบจะไม่สามารถเพิ่ม พูดคุย แก้ไข หรือลบกลุ่ม เพิ่ม แก้ไข หรือลบสมาชิกในกลุ่มได้} \\ \hline
					\multicolumn{2}{|p{\linewidth}|}{Exceptional Flow ที่ 2 : หากผู้ใช้ไม่เชื่อมต่ออินเทอร์เน็ตจะไม่สามารถเพิ่ม พูดคุย แก้ไข หรือลบกลุ่ม เพิ่ม แก้ไข หรือลบสมาชิกในกลุ่มได้} \\ \hline
				\end{tabular}%
			}
		\end{table}	

		 \begin{table}[H]
		 	\centering
		 	\caption{Use Case จัดการเพื่อน}
		 	\label{tab:usecase}
		 	\resizebox{\totalheight}{!}{\textwidth}{%
		 		\begin{tabular}{|p{10cm}|p{10cm}|}
		 			\hline
		 			\textbf{Use Case Title : จัดการเพื่อน} & \multicolumn{1}{c|}{\textbf{Use case Id : 9 }} \\ \hline
		 			\multicolumn{2}{|l|}{Primary Actor : ผู้ใช้งาน} \\ \hline
		 			\multicolumn{2}{|l|}{Stakeholder Actor : -} \\ \hline
					 \multicolumn{2}{|p{\linewidth}|}{Main Flow : สามารถเพิ่ม พูดคุย แก้ไข หรือลบเพื่อนได้} \\ \hline
					 \multicolumn{2}{|p{\linewidth}|}{Exceptional Flow ที่ 1 : หากผู้ใช้ไม่เข้าสู่ระบบจะไม่สามารถเพิ่ม พูดคุย แก้ไข หรือลบเพื่อนได้} \\ \hline
		 			\multicolumn{2}{|p{\linewidth}|}{Exceptional Flow ที่ 2 : หากผู้ใช้ไม่เชื่อมต่ออินเทอร์เน็ตจะไม่สามารถเพิ่ม พูดคุย แก้ไข หรือลบเพื่อนได้} \\ \hline
		 		\end{tabular}%
		 	}
		 \end{table}	

		  \begin{table}[H]
		  	\centering
		  	\caption{Use Case เพิ่ม หรือลบการแจ้งเตือน}
		  	\label{tab:usecase}
		  	\resizebox{\totalheight}{!}{\textwidth}{%
		  		\begin{tabular}{|p{10cm}|p{10cm}|}
		  			\hline
		  			\textbf{Use Case Title : เพิ่ม หรือลบการแจ้งเตือน} & \multicolumn{1}{c|}{\textbf{Use case Id : 10 }} \\ \hline
		  			\multicolumn{2}{|l|}{Primary Actor : ผู้ใช้งาน} \\ \hline
		  			\multicolumn{2}{|l|}{Stakeholder Actor : -} \\ \hline
		  			\multicolumn{2}{|p{\linewidth}|}{Main Flow : สามารถเพิ่ม หรือลบการแจ้งเตือนการทานยาได้} \\ \hline
		  			\multicolumn{2}{|p{\linewidth}|}{Exceptional Flow ที่ 1 : หากผู้ใช้ไม่เข้าสู่ระบบจะไม่สมารถเพิ่ม หรือลบการแจ้งเตือนได้} \\ \hline
		  		\end{tabular}%
		  	}
		  \end{table}	
		  \begin{table}[H]
			    	\centering
			    	\caption{Use Case ดูวิธีการใช้งาน}
			    	\label{tab:usecase}
			    	\resizebox{\totalheight}{!}{\textwidth}{%
			    		\begin{tabular}{|p{10cm}|p{10cm}|}
			    			\hline
			    			\textbf{Use Case Title : ดูวิธีการใช้งาน} & \multicolumn{1}{c|}{\textbf{Use case Id : 11 }} \\ \hline
			    			\multicolumn{2}{|l|}{Primary Actor : ผู้ใช้งาน} \\ \hline
			    			\multicolumn{2}{|l|}{Stakeholder Actor : -} \\ \hline
			    			\multicolumn{2}{|p{\linewidth}|}{Main Flow : สามารถดูรายละเอียดของหน้าจอต่าง ๆ ได้} \\ \hline
			    			\multicolumn{2}{|p{\linewidth}|}{Exceptional Flow ที่ 1 : หากผู้ใช้ไม่เข้าสู่ระบบจะไม่สามารถดูรายละเอียดของหน้าจอต่าง ๆ ได้} \\ \hline
			    		\end{tabular}%
			    	}
		  \end{table}	
		    \begin{table}[H]
		    	\centering
		    	\caption{Use Case คุยกับแชทบอท}
		    	\label{tab:usecase}
		    	\resizebox{\totalheight}{!}{\textwidth}{%
		    		\begin{tabular}{|p{10cm}|p{10cm}|}
		    			\hline
		    			\textbf{Use Case Title : คุยกับแชทบอท} & \multicolumn{1}{c|}{\textbf{Use case Id : 12 }} \\ \hline
		    			\multicolumn{2}{|l|}{Primary Actor : ผู้ใช้งาน} \\ \hline
		    			\multicolumn{2}{|l|}{Stakeholder Actor : -} \\ \hline
		    			\multicolumn{2}{|p{\linewidth}|}{Main Flow : สามารถคุยโต้ตอบกับแชทบอทได้} \\ \hline
						\multicolumn{2}{|p{\linewidth}|}{Exceptional Flow ที่ 1 : หากผู้ใช้งานไม่เชื่อมต่ออินเทอร์เน็ตจะไม่คุยกับแชทบอทได้} \\ \hline
						\multicolumn{2}{|p{\linewidth}|}{Exceptional Flow ที่ 2 : หากผู้ใช้งานไม่เข้าสู่ระบบจะไม่คุยกับแชทบอทได้} \\ \hline
		    		\end{tabular}%
		    	}
		    \end{table}	
\newpage

%  สิ้นสุด Use Case Diagrame

%  เริ่มต้น Class Dialgram

\subsection{แผนภาพคลาส (Class Diagram)}
	Class Diagram คือแผนภาพที่ใช้แสดงคลาสและความสัมพันธ์ในแบบต่างๆ ระหว่างคลาส สัญลักษณ์ที่ใช้ในการเขียน Class Diagram แสดงในตารางที่ \ref{tab:class2} 
	\begin{center}
	\begin{table}[H]
		\centering
		\caption{สัญลักษณ์ของ Class Diagram}
		\label{tab:class2}
		\begin{tabular}{|c|p{10cm}|}
			\hline
			\textbf{สัญลักษณ์} & \multicolumn{1}{c|}{\textbf{การใช้งาน}} \\ \hline
			\raisebox{-\totalheight}{\includegraphics[width=0.3\textwidth]{Figures/table/class/11}}
			& \setstretch{1.5} {คลาส สัญลักษณ์แทนด้วยสี่เหลี่ยมแบ่งเป็น 3 ส่วน 
				ส่วนบน เป็นชื่อของ class ส่วนกลาง เป็นชื่อ Attribute และส่วนล่างเป็น Operation Name หรือ Method ใช้สำหรับเขียนฟังก์ชันในการทำงานของคลาสนั้น ๆ
				ชนิดของ Visibility ของ Method และ Attribute
				แบ่งเป็น 3 ชนิด ได้แก่
				\begin{enumerate}
					\item Public แทนสัญลักษณ์ด้วยเครื่องหมายบวก (+)
					\item Private แทนสัญลักษณ์ด้วยเครื่องหมายลบ (-)
					\item Protected แทนสัญลักษณ์ด้วยเครื่องหมายชาร์ป (#)
				\end{enumerate}
			} \\ \hline
			\raisebox{-\totalheight}{\includegraphics[width=0.3\textwidth]{Figures/table/class/1}}
			& \setstretch{1.5} {Dependency Relationship หมายความว่า คลาสที่อยู่ฝั่งต้นลูกศรสามารถเรียกใช้คลาสที่อยู่ฝั่งหัวลูกศร}
			\\ \hline
			\raisebox{-\totalheight}{\includegraphics[width=0.35\textwidth]{Figures/3/Class/aggre}}
			& \setstretch{1.5} {Composition Relationship เป็นความสัมพันธ์ระหว่างออบเจ็กต์หรือคลาสแบบขึ้นต่อกันและมีความเกี่ยวข้องกันเสมอ} \\ \hline
			\raisebox{-\totalheight}{\includegraphics[width=0.3\textwidth]{Figures/3/Class/implement}}
			& \setstretch{1.5} {Realization Relationship เป็นความสัมพันธ์ระหว่าง Object หรือ Class ในลักษณะของการสืบทอดคุณสมบัติจาก Class หนึ่ง (Super class) ไปยังอีก Class หนึ่ง (Subclass)} \\ \hline
			\raisebox{-\totalheight}{\includegraphics[width=50,height=50]{Figures/table/class/connector}}
			& \setstretch{1.5} {Connector เป็นสัญลักษณ์แทนด้วยรูปห้าเหลี่ยมและมีชื่ออยู่ตรงกลาง จะสร้างสัญลักษณ์นี้ไว้เมื่อต้องการเชื่อมต่อคลาสที่อยู่คนละหน้า} \\ \hline
		\end{tabular}
	\end{table}
	\end{center}

\newpage
  %IMAGE of class
	Class Diagram แสดงความสัมพันธ์ในรูปแบบต่างๆ ระหว่างคลาสของแอปพลิเคชันสูงวัยมายเฟรนด์ อธิบายได้ตามภาพที่ \ref{Fig:MainActivity20C} ดังต่อไปนี้

\begin{sidewaysfigure}
	\begin{figure}[H]
		\includegraphics[width=1.0\columnwidth]{Figures/3/Class/classdiagrams}
		\caption{Class Diagram ของแอปพลิเคชันสูงวัยมายเฟรนด์}
		\label{Fig:classdiagram}
	\end{figure}
\end{sidewaysfigure}


	% TABLE of class
\newpage	
	จากรูปภาพที่ \ref{Fig:classdiagram} สามารถอธิบายแผนภาพ Class Diagram ได้ดังนี้
	\begin{table}[H]
		\centering
		\caption{อธิบาย Class Diagram ของแอปพลิเคชันสูงวัยมายเฟรนด์}
		\label{tab:class}
		\begin{tabular}{|c|p{10cm}|}
			\hline
			\textbf{Class Diagram} & \multicolumn{1}{c|}{\textbf{คำอธิบาย}} \\ \hline
			\raisebox{-\totalheight}{LoginPage}
			& \setstretch{1.5} {คลาส LoginPage เป็นคลาสที่ใช้เพื่อให้ผู้ใช้ที่ได้ลงทะเบียนกับระบบไว้แล้วเข้าระบบเพื่อใช้งานแอปพลิเคชัน } \\ \hline
			\raisebox{-\totalheight}{TabsPage}
			& \setstretch{1.5} {คลาส TabsPage เป็นคลาสที่ถูกเรียกเพื่อจัดการ Tabs} \\ \hline
			\raisebox{-\totalheight}{TabsfriendsPage}
			& \setstretch{1.5} {คลาส TabsfriendsPage เป็นคลาสที่ใช้แสดงสมาชิกกลุ่ม คำร้องขอเพิ่มเพื่อนและเพื่อน} \\ \hline
			\raisebox{-\totalheight}{FriendsPage}
			& \setstretch{1.5} {คลาส FriendsPage เป็นคลาสที่ใช้แสดงคำร้องขอเพื่อนและเพื่อน} \\ \hline
			\raisebox{-\totalheight}{connreq}
			& \setstretch{1.5} {คลาส connreq เป็นคลาสที่ใช้เก็บข้อมูลคำร้องขอเพื่อน} \\ \hline
			\raisebox{-\totalheight}{BuddiesPage}
			& \setstretch{1.5} {คลาส BuddiesPage เป็นคลาสที่ใช้สำหรับการเพิ่มเพื่อน} \\ \hline
			\raisebox{-\totalheight}{BuddychatPage}
			& \setstretch{1.5} {คลาส BuddychatPage เป็นคลาสที่ใช้สำหรับแชทกับเพื่อน} \\ \hline
			\raisebox{-\totalheight}{BuddychatPage}
			& \setstretch{1.5} {คลาส BuddychatPage เป็นคลาสที่ใช้สำหรับแชทกับเพื่อน} \\ \hline
			\raisebox{-\totalheight}{GroupsPage}
			& \setstretch{1.5} {คลาส GroupsPage เป็นคลาสที่แสดงกลุ่มของฉันและกลุ่มที่เป็นสมาชิก} \\ \hline
			\raisebox{-\totalheight}{NewgroupPage}
			& \setstretch{1.5} {คลาส NewgroupPage เป็นคลาสที่ใช้สร้างกลุ่ม} \\ \hline
			\raisebox{-\totalheight}{GroupchatPage}
			& \setstretch{1.5} {คลาส GroupchatPage เป็นคลาสสำหรับการแชทกลุ่ม} \\ \hline
			\raisebox{-\totalheight}{SettingallalarmPage}
			& \setstretch{1.5} {คลาส SettingallalarmPage เป็นคลาสที่แสดงการแจ้งเตือนการทานยาทั้งหมด} \\ \hline
			\raisebox{-\totalheight}{SettingsalarmPage}
			& \setstretch{1.5} {คลาส SettingsalarmPage เป็นคลาสที่ใช้บันทึกการแจ้งเตือนการทานยา} \\ \hline
			\raisebox{-\totalheight}{TabsCameraPage}
			& \setstretch{1.5} {คลาส TabsCameraPage เป็นคลาสที่ใช้สำหรับสร้างโพสท์} \\ \hline
			\raisebox{-\totalheight}{HomePage}
			& \setstretch{1.5} {คลาส HomePage เป็นคลาสสำหรับการแสดงหน้าเมนูหลักของแอปพลิเคชัน} \\ \hline
			\raisebox{-\totalheight}{ChatbotPage}
			& \setstretch{1.5} {คลาส ChatbotPage เป็นคลาสที่ใช้สำหรับพูดคุยกับแชทบอท} \\ \hline
			\raisebox{-\totalheight}{FeedPage}
			& \setstretch{1.5} {คลาส FeedPage เป็นคลาสที่แสดงโพสท์ทั้งหมด} \\ \hline
			\raisebox{-\totalheight}{FamilyPage}
			& \setstretch{1.5} {คลาส FamilyPage เป็นคลาสที่แสดงสมาชิกในครอบครัว} \\ \hline
			\raisebox{-\totalheight}{GooglemapPage}
			& \setstretch{1.5} {คลาส GooglemapPage เป็นคลาสที่แสดงตำแหน่งของสมาชิกในครอบครัวด้วย Google Map} \\ \hline
			\raisebox{-\totalheight}{HelpPage}
			& \setstretch{1.5} {คลาส HelpPage เป็นคลาสที่แสดงรูปภาพขอความช่วยเหลือ 2 แบบ คือ แบบติดต่อครอบครัว และแบบส่งเสียงฉุกเฉิน} \\ \hline
			\raisebox{-\totalheight}{HelpfamilyPage}
			& \setstretch{1.5} {คลาส HelpfamilyPage เป็นคลาสแสดงสมาชิกในครอบครัวเพื่อขอความช่วยเหลือมี 2 แบบได้แก่ โทร และส่งข้อความ} \\ \hline
	\end{tabular}
\end{table}

\begin{table}[H]
	\centering
	\caption{อธิบาย Class Diagram ของแอปพลิเคชันสูงวัยมายเฟรนด์}
	\label{tab:class}
	\begin{tabular}{|c|p{10cm}|}
		\hline
		\textbf{Class Diagram} & \multicolumn{1}{c|}{\textbf{คำอธิบาย}} \\ \hline
		\raisebox{-\totalheight}{FamilyPage}
		& \setstretch{1.5} {คลาส FamilyPage เป็นคลาสที่แสดงสมาชิกในครอบครัว} \\ \hline
		\raisebox{-\totalheight}{GooglemapPage}
		& \setstretch{1.5} {คลาส GooglemapPage เป็นคลาสที่แสดงตำแหน่งของสมาชิกในครอบครัวด้วย Google Map} \\ \hline
		\raisebox{-\totalheight}{HelpPage}
		& \setstretch{1.5} {คลาส HelpPage เป็นคลาสที่แสดงรูปภาพขอความช่วยเหลือ 2 แบบ คือ แบบติดต่อครอบครัว และแบบส่งเสียงฉุกเฉิน} \\ \hline
		\raisebox{-\totalheight}{HelpfamilyPage}
		& \setstretch{1.5} {คลาส HelpfamilyPage เป็นคลาสแสดงสมาชิกในครอบครัวเพื่อขอความช่วยเหลือมี 2 แบบได้แก่ โทร และส่งข้อความ} \\ \hline
\end{tabular}
\end{table}

\newpage

%  สิ้นสุด Class Diagram

%  เริ่มต้น Squence Diagram

\section{ซีเควนไดอะแกรม (Sequence Diagram)}
	Sequence Diagram เป็น Diagram ที่แสดงขั้นตอนการทำงานของแต่ละ Use Case ระหว่าง Object ต่างๆ ที่ส่งข้อความถึงกันและกัน โดย Sequence Diagram จะช่วยให้มองเห็นการทำงานของภาพรวมของระบบ ส่วนประกอบสัญลักษณ์ที่ใช้ในการเขียน Sequence Diagram 
	แสดงดังตารางที่ \ref{tab:Sequences}
	
	\begin{table}[H]
		\centering
		\caption{สัญลักษณ์ของ Sequence Diagram}
		\label{tab:Sequences}
		\begin{tabular}{| c	| p{10cm} |}
		\hline
		\textbf{สัญลักษณ์} & \multicolumn{1}{c|}{\textbf{การใช้งาน}} \\ \hline
		\raisebox{-\totalheight}{\includegraphics[width=0.17\textwidth]{Figures/table/Sequence/Sequence1}}
		& \setstretch{1.5} {Class แสดงถึงการทำงานของ Use Case ในการส่งหรือรับข้อความ แทนด้วยสัญลักษณ์สี่เหลี่ยมมีชื่อคลาสอยู่ภายใน} \\ \hline
		\raisebox{-\totalheight}{\includegraphics[height=0.08\textheight]{Figures/table/Sequence/Sequence2}}
		& \setstretch{1.5} {Lifeline หรือเส้นอายุขัย แสดงช่วงเวลาตั้งแต่เริ่มสร้าง object ในคลาสนั้น จนกระทั่ง object นั้นถูกทำลาย สัญลักษณ์แทนด้วยเส้นประ} \\ \hline
		\raisebox{-\totalheight}{\includegraphics[height=0.08\textheight]{Figures/table/Sequence/Sequence3}}
		& \setstretch{1.5} {Focus of control หรือจุดควบคุม เป็นจุดควบคุมที่ object ใช้ทำการส่งหรือรับข้อความ สัญลักษณ์แทนด้วยสี่เหลี่ยม} \\ \hline
		\raisebox{-\totalheight}{\includegraphics[width=0.3\textwidth]{Figures/table/Sequence/Sequence4}}
		& \setstretch{1.5} {Message คือ ข้อความที่รับส่งระหว่าง Object สัญลักษณ์แทนด้วยลูกศรและประกอบด้วย 2 ส่วน คือ ข้อมูล (Data) และฟังก์ชัน (Function)} \\ \hline
		\raisebox{-\totalheight}{\includegraphics[width=0.3\textwidth]{Figures/table/Sequence/Sequence5}}
		& \setstretch{1.5} {Return Message เป็นข้อมูลที่ส่งกลับหลังจากทำงานเสร็จ} \\ \hline
		\raisebox{-\totalheight}{\includegraphics[height=0.08\textheight]{Figures/3/selfcall}}
		& \setstretch{1.5} {Self call เป็นการเรียกฟังชันก์การทำงานภายในตัวเอง} \\ \hline
		\raisebox{-\totalheight}{\includegraphics[height=0.1\textheight,width=0.3\textwidth]{Figures/3/frame}}
		& \setstretch{1.5} {สร้างกรอบการทำงานของโปรแกรม เพื่อให้รู้ขอบเขตของการทำงานเช่น ลูป(loop)} \\ \hline
		\end{tabular}
	\end{table}
%
%	Sequence Diagram ที่ใช้อธิบายการทำงานของระบบกองทุนเงินให้กู้ยืมเพื่อการศึกษา คณะวิทยสศาสตร์ มหาวิทยาลัยอุบลราชธานี มีรายละเอียดดังต่อไปนี้

\newpage
	\begin{landscape}
	\begin{figure}[H]
		\centering
		\includegraphics[width=0.95\columnwidth]
		{Figures/3/Sequence/feed}
		\caption{Sequence Diagram การแสดงข่าวสาร}
		\label{Fig:Sequence-feed}
	\end{figure}
  \end{landscape}

	จากภาพที่ \ref{Fig:Sequence-feed} สามารถอธิบายแผนภาพ Sequence Diagram แสดงข่าวสาร ได้ดังนี้ เมื่อ
	ผู้ใช้เปิดโปรแกรมระบบจะเรียกใช้เมธอด onCreate() ที่คลาส MainActivity ระบบจะทำการสร้าง
	Fragment ขึ้นมาโดยใช้เมธอด onCreate() ที่คลาส FeedFragment เมื่อ FeedFragment ถูกติดตั้งบน MainActivity เมธอด callData() จะสืบค้นข้อมูลจากฐานข้อมูลบน Firebase FireStore และส่งข้อมูลที่ได้ไปแปลงที่คลาส FeedItemAdapter โดยมีการคืนค่าเป็นข้อมูลข่าวสารแต่ละแถวและในขั้นตอนสุดท้ายคลาส FeedFragment จะทำการแสดงรายการข้อมูลข่าวสารทั้งหมดออกทางหน้าจอ หากผู้ใช้มีการกดเลือกข่าวสารบางแถวคลาส FeedFragment จะทำการเรียกใช้ PostDetailActivity เพื่อแสดงรายละเอียดข้อมูลข่าวสารของแถวที่ถูกเลือก
	\begin{sidewaysfigure}
	\begin{figure}[H]
		\centering
		\includegraphics[width=0.8\columnwidth]
		{Figures/3/Sequence/calendar}
		\caption{Sequence Diagram การแสดงปฏิทินกำหนดการ}
		\label{Fig:Sequence-calendar}
	\end{figure}
	\end{sidewaysfigure}
	\newpage
	จากภาพที่ \ref{Fig:Sequence-calendar} สามารถอธิบายแผนภาพ Sequence Diagram แสดงปฏิทินกำหนดการ ได้ดังนี้ เมื่อ
	ผู้ใช้เปิดโปรแกรมระบบจะเรียกใช้เมธอด onCreate() ที่คลาส MainActivity ระบบจะทำการสร้าง
	Fragment ขึ้นมาโดยใช้เมธอด onCreate() ที่คลาส ScheduleFragment เมื่อ ScheduleFragment ถูกติดตั้งบน MainActivity เมธอด callData() จะสืบค้นข้อมูลกำหนดการของวันปัจจุบันจากฐานข้อมูลบน Firebase FireStore และส่งข้อมูลที่ได้ไปแปลงที่คลาส Schedule-ItemAdapter โดยมีการคืนค่าเป็นข้อมูลกำหนดการแต่ละแถวและในขั้นตอนสุดท้ายคลาส Schedule-Fragment จะทำการแสดงรายการกำหนดการวันปัจจุบันออกทางหน้าจอ หากผู้ใช้มีการกดเลือกวันที่ที่ต้องการทราบกำหนดการจากปฏิทินคลาส ScheduleFragment จะทำการเรียกใช้ callData() อีกครั้งโดยสืบค้นข้อมูลกำหนดการของวันที่ถูกเลือกจากฐานข้อมูลบน Firebase FireStore และส่งข้อมูลที่ได้ไปแปลงที่คลาส ScheduleItemAdapter โดยมีการคืนค่าเป็นข้อมูลแต่กำหนดการละแถวและในขั้นตอนสุดท้ายคลาส ScheduleFragment จะทำการแสดงรายการกำหนดการวันที่ผู้ใช้เลือกออกทางหน้าจอ

	\begin{sidewaysfigure}
	\begin{figure}[H]
		\centering
		\includegraphics[width=0.8\columnwidth]
		{Figures/3/Sequence/doc}
		\caption{Sequence Diagram การแสดงดาวน์โหลดเอกสาร}
		\label{Fig:Sequence-doc}
	\end{figure}
	\end{sidewaysfigure}
	\newpage
	จากภาพที่ \ref{Fig:Sequence-doc} สามารถอธิบายแผนภาพ Sequence Diagram แสดงดาวน์โหลดเอกสาร ได้ดังนี้ เมื่อผู้ใช้เปิดโปรแกรมระบบจะเรียกใช้เมธอด onCreate() ที่คลาส MainActivity ระบบจะทำการสร้าง
	Fragment ขึ้นมาโดยใช้เมธอด onCreate() ที่คลาส DocumentsFragment เมื่อ DocumentsFragment ถูกติดตั้งบน MainActivity เมธอด initInstances() จะสืบค้นข้อมูลเอกสารทั้งหมดจากฐานข้อมูลบน Firebase FireStore และส่งข้อมูลที่ได้ไปแปลงที่คลาส DocItem-Adapter โดยมีการคืนค่าเป็นข้อมูลเอกสารแต่ละแถวและในขั้นตอนสุดท้ายคลาส Documents-Fragment จะทำการแสดงรายการกำหนดการวันปัจจุบันออกทางหน้าจอ

	\begin{sidewaysfigure}
	\begin{figure}[H]
		\centering
		\includegraphics[width=0.8\columnwidth]
		{Figures/3/Sequence/chat}
		\caption{Sequence Diagram การแสดงบทสนทนา}
		\label{Fig:Sequence-chat}
	\end{figure}
	\end{sidewaysfigure}
	\newpage
	จากภาพที่ \ref{Fig:Sequence-chat} สามารถอธิบายแผนภาพ Sequence Diagram แสดงการสนทานา ได้ดังนี้ เมื่อผู้ใช้เปิดโปรแกรมระบบจะเรียกใช้เมธอด onCreate() ที่คลาส MainActivity ระบบจะทำการสร้าง
	Fragment ขึ้นมาโดยใช้เมธอด onCreate() ที่คลาส UserChatFragment เมื่อ UserChatFrag-ment ถูกติดตั้งบน MainActivity เมธอด getMessage() จะสืบค้นข้อมูลประวัติการสนทนาของผู้ใช้คนปัจจุบันทั้งหมดจากฐานข้อมูลบน Firebase FireStore และส่งข้อมูลที่ได้ไปแปลงที่คลาส MessagesListAdapter โดยมีการคืนค่าเป็นข้อมูลรายการประวัติการสนทนาทั้งหมดและในขั้นตอนสุดท้ายคลาส User-ChatFragment จะทำการแสดงรายการประวัติการสนทนาทั้งหมดออกทางหน้าจอ เมื่อผู้ใช้พิมพ์ข้อความและกดปุ่มส่งระบบจะเรียกให้เมธอด send() เพื่อทำการบันทึกข้อมูลไว้บน Firebase FireStore และทำการแสดงข้อมูลรายการประวัติการสนทนาทั้งหมดที่ถูกอัพเดท

	\begin{sidewaysfigure}
	\begin{figure}[H]
		\centering
		\includegraphics[width=0.8\columnwidth]
		{Figures/3/Sequence/submit}
		\caption{Sequence Diagram แสดงส่งเอกสารตรวจสอบ}
		\label{Fig:Sequence-submit}
	\end{figure}
	\end{sidewaysfigure}
	\newpage
	จากภาพที่ \ref{Fig:Sequence-submit} สามารถอธิบายแผนภาพ Sequence Diagram แสดงส่งเอกสารตรวจสอบ ได้ดังนี้ เมื่อผู้ใช้เปิดโปรแกรมระบบจะเรียกใช้เมธอด onCreate() ที่คลาส MainActivity ระบบจะทำการสร้าง
	Fragment ขึ้นมาโดยใช้เมธอด onCreate() ที่คลาส SubmitFragment เมื่อ Submit-Fragment ถูกติดตั้งบน MainActivity เมธอด initInstances() จะถูกเรียกเพื่อสร้างหน้าจอแสงดผลเมื่อผู้ใช้กดปุ่มถ่ายรูประบบจะเรียกใช้ไลบรารี่ ScanConstants เพื่อถ่ายภาพเอกสารและรอให้ผู้ใช้ถ่ายครบทั้งสองแผ่นจึงจะแสดงปุ่มกดส่งเอกสารเพื่อตรวจสอบ
\newpage	 
	
\section{โครงสร้างฐานข้อมูลไฟร์เบส(Firebase Database Stucture)}
Firebase Database นั้นเป็น Database แบบ NoSQL และเป็น JSON database ที่มีโครงสร้างที่เป็น Key และ Value จัดเก็บข้อมูลในลักษณะโหนด หากต้องการเรียกงานจะเรียกใช้โดย
การท่องไปยังโหนดที่ต้องการ ส่วนประกอบสัญลักษณ์ที่ใช้ในการเขียนโครงสร้างฐานข้อมูลแบบ Firebase
แสดงดังตารางที่ \ref{tab:DB}

\begin{table}[H]
	\centering
	\caption{สัญลักษณ์ของโครงสร้างฐานข้อมูลแบบ Firebase}
	\label{tab:DB}
	\begin{tabular}{| c	| p{10cm} |}
		\hline
		\textbf{สัญลักษณ์} & \multicolumn{1}{c|}{\textbf{คำอธิบาย}} \\ \hline
		\raisebox{-\totalheight}{\includegraphics[width=0.1\textwidth]{Figures/3/DB/dbroot}}
		& \setstretch{1.5} {Database เป็นการเรียกชื่อแทนโหนด(Node)บนสุดที่ใช้ในการเก็บข้อมูล} \\ \hline
		\raisebox{-\totalheight}{\includegraphics[width=0.1\textwidth]{Figures/3/DB/dbcollection}}
		& \setstretch{1.5} {Collection เป็นการเรียกชื่อแทนของการเก็บหลาย ๆ เอกสารไว้ด้วยกัน} \\ \hline
		\raisebox{-\totalheight}{\includegraphics[width=0.1\textwidth]{Figures/3/DB/dbdoc}}
		& \setstretch{1.5} {Document เป็นการเรียกชื่อแทนหน่วยการเก็บของข้อมูลใน Cloud Firestore ภายในจะประกอบไปด้วย ชื่อของ Document  ชื่อของคีย์ (key) และ ค่าข้อมูล (value) โดยชื่อของ Document ห้ามซ้ำกัน ซึ่งใน Cloud Firestore สามารถระบุประเภทของข้อมูลได้ 9 ประเภทได้แก่ boolean, number, string, geo point, timestamp, array, object, reference และ null} \\ \hline
	\end{tabular}
\end{table}
	\begin{figure}[H]
	\centering
	\includegraphics[width=0.7\columnwidth]
	{Figures/3/DB/DB1}
	\caption{โครงสร้างฐานข้อมูลแบบ Firebase}
	\label{Fig:DB1}
	\end{figure}

	\begin{figure}[H]
	\centering
	\includegraphics[width=0.9\columnwidth]
	{Figures/3/DB/DB2}
	\caption{โครงสร้างฐานข้อมูลแบบ Firebase(ต่อ)}
	\label{Fig:DB2}
\end{figure}
	\begin{figure}[H]
	\centering
	\includegraphics[width=0.55\columnwidth]
	{Figures/3/DB/DB3}
	\caption{โครงสร้างฐานข้อมูลแบบ Firebase(ต่อ)}
	\label{Fig:DB3}
\end{figure}
	\begin{figure}[H]
	\centering
	\includegraphics[width=0.7\columnwidth]
	{Figures/3/DB/DB4}
	\caption{โครงสร้างฐานข้อมูลแบบ Firebase(ต่อ)}
	\label{Fig:DB4}
\end{figure}

\newpage
จากรูที่ \ref{Fig:DB1}-\ref{Fig:DB4} สามารถอธิบายโครงสร้างของข้อมูลได้ดังนี้
\begin{figure}[H]
\centering
\includegraphics[width=0.5\columnwidth]
{Figures/3/DB/nodePost}
\caption{โหนดเก็บข้อมูลประกาศ}
\label{Fig:DB4}
\end{figure}
\begin{table}[H]
	\centering
	\caption{อธิบายโหนดที่ใช้เก็บข้อมูลประกาศ}
	\label{my-label1}
	\begin{tabular}{|c|p{10cm}|}
		\hline
		\multicolumn{1}{|c|}{\textbf{Key}} & \multicolumn{1}{c|}{\textbf{คำอธิบาย}} \\ \hline
		Posts & โหนดสำหรับเก็บข้อมูลประกาศทั้งหมด \\ \hline
		Post &  สำหรับเก็บข้อมูลแต่ละประกาศ \\ \hline
		title & สำหรับเก็บชื่อหัวข้อประกาศ \\ \hline
		description & สำหรับเก็บรายละเอียดประกาศ  \\ \hline
		collection & สำหรับเก็บประเภทของประกาศได้แก่ สาธารณะและเฉพาะบุคคล \\ \hline
		fileURL & สำหรับเก็บ url ของเอกสารแนบประกาศ \\ \hline
		id & สำหรับเก็บรหัสของประกาศ \\ \hline
		time & สำหรับเก็บเวลาที่ประกาศ \\ \hline
	\end{tabular}
\end{table}

\newpage
\begin{figure}[H]
	\centering
	\includegraphics[width=0.5\columnwidth]
	{Figures/3/DB/nodeDoc}
	\caption{โหนดเก็บข้อมูลเอกสารที่เกี่ยวข้อง}
	\label{Fig:DB4}
\end{figure}
\begin{table}[H]
	\centering
	\caption{อธิบายโหนดที่ใช้เก็บข้อมูลเอกสารที่เกี่ยวข้อง}
	\label{my-label1}
	\begin{tabular}{|c|p{10cm}|}
		\hline
		\multicolumn{1}{|c|}{\textbf{Key}} & \multicolumn{1}{c|}{\textbf{คำอธิบาย}} \\ \hline
		Docs & โหนดสำหรับเก็บข้อมูลของเอกสารที่เกี่ยวข้องทั้งหมด \\ \hline
		Doc &  สำหรับเก็บข้อมูลเอกสารแต่ละฉบับ \\ \hline
		title & สำหรับเก็บชื่อหัวเรื่องของเอกสาร \\ \hline
		description & สำหรับเก็บรายละเอียดของเอกสาร \\ \hline
		fileType & สำหรับนามสกุลไฟล์เอกสาร เช่น .pdf .png เป็นต้น \\ \hline
		fileURL & สำหรับเก็บ url ของเอกสาร\\ \hline
		time & สำหรับเก็บเวลาที่ถูกอัพโหลดเข้าสู่ระบบโดยเจ้าหน้าที่\\ \hline
	\end{tabular}
\end{table}

\newpage
\begin{figure}[H]
	\centering
	\includegraphics[width=0.4\columnwidth]
	{Figures/3/DB/nodeChat}
	\caption{โหนดเก็บข้อมูลประวัติการสนทนา}
	\label{Fig:DB4}
\end{figure}
\begin{table}[H]
	\centering
	\caption{อธิบายโหนดที่ใช้เก็บข้อมูลประวัติการสนทนา}
	\label{my-label1}
	\begin{tabular}{|c|p{10cm}|}
		\hline
		\multicolumn{1}{|c|}{\textbf{Key}} & \multicolumn{1}{c|}{\textbf{คำอธิบาย}} \\ \hline
		Chats & โหนดสำหรับเก็บข้อมูลประวัติการสนทนาทั้งหมด \\ \hline
		User\_id &  สำหรับเก็บประวัติการสนทนาของผู้ใช้แต่ละคน \\ \hline
		Messages & สำหรับเก็บประวัติการสนทนาทั้งหมดของผู้ใช้ \\ \hline
		Message & สำหรับเก็บข้อมูลของแต่ละข้อความ \\ \hline
		message & สำหรับเก็บข้อความ \\ \hline
		name & สำหรับเก็บชื่อของผู้ส่งข้อความ\\ \hline
		photo & สำหรับเก็บ url รูปภาพของผู้ส่งข้อความ\\ \hline
		senderId & สำหรับเก็บรหัสของผู้ส่งข้อความ\\ \hline
		time & สำหรับเก็บเวลาที่ข้อความถูกส่ง\\ \hline
	\end{tabular}
\end{table}

\newpage
\begin{figure}[H]
	\centering
	\includegraphics[width=0.5\columnwidth]
	{Figures/3/DB/nodeEvent}
	\caption{โหนดเก็บข้อมูลกำหนดการ}
	\label{Fig:DB4}
\end{figure}
\begin{table}[H]
	\centering
	\caption{อธิบายโหนดที่ใช้เก็บข้อมูลกำหนดการ}
	\label{my-label1}
	\begin{tabular}{|c|p{10cm}|}
		\hline
		\multicolumn{1}{|c|}{\textbf{Key}} & \multicolumn{1}{c|}{\textbf{คำอธิบาย}} \\ \hline
		Events & โหนดสำหรับเก็บข้อมูลของกำหนดการทั้งหมด \\ \hline
		Event & สำหรับเก็บข้อมูลของแต่ละกำหนดการ \\ \hline
		title & สำหรับเก็บชื่อหัวข้อของกำหนดการ \\ \hline
		description & สำหรับเก็บรายละเอียดของกำหนดการ\\ \hline
		time & สำหรับเก็บเวลาของกำหนดการ\\ \hline
	\end{tabular}
\end{table}

\newpage
\begin{figure}[H]
	\centering
	\includegraphics[width=0.4\columnwidth]
	{Figures/3/DB/nodeReq}
	\caption{โหนดเก็บข้อมูลการยื่นสำเนาเอกสารเพื่อตรวจสอบของนักศึกษา}
	\label{Fig:DB4}
\end{figure}
\begin{table}[H]
	\centering
	\caption{อธิบายโหนดที่ใช้เก็บข้อมูลการยื่นสำเนาเอกสารเพื่อตรวจสอบของนักศึกษา}
	\label{my-label1}
	\begin{tabular}{|c|p{10cm}|}
		\hline
		\multicolumn{1}{|c|}{\textbf{Key}} & \multicolumn{1}{c|}{\textbf{คำอธิบาย}} \\ \hline
		RusetSubmitDocs & โหนดสำหรับเก็บข้อมูลการยื่นสำเนาเอกสารเพื่อตรวจสอบของนักศึกษาทั้งหมด \\ \hline
		User\_id & สำหรับเก็บข้อมูลของแต่ละสำเนาเอกสารของนักศึกษาแต่ละคน \\ \hline
		doc2 & สำหรับเก็บ url ของภาพถ่ายสำเนาเอกสารฉบับที่ 1\\ \hline
		doc2 & สำหรับเก็บ url ของภาพถ่ายสำเนาเอกสารฉบับที่ 2\\ \hline
		status & สำหรับเก็บผลการตรวจสอบของเจ้าหน้าที่ \\ \hline
		time & สำหรับเก็บเวลาที่สำเนาเอกสารถูกเพิ่มเข้าสู่ระบบ \\ \hline
	\end{tabular}
\end{table}

\newpage
\begin{figure}[H]
	\centering
	\includegraphics[width=0.35\columnwidth]
	{Figures/3/DB/nodeUser}
	\caption{โหนดเก็บข้อมูลของนักศึกษา}
	\label{Fig:DB4}
\end{figure}
\begin{table}[H]
	\centering
	\caption{อธิบายโหนดที่ใช้เก็บข้อมูลของนักศึกษา}
	\label{my-label1}
	\begin{tabular}{|c|p{10cm}|}
		\hline
		\multicolumn{1}{|c|}{\textbf{Key}} & \multicolumn{1}{c|}{\textbf{คำอธิบาย}} \\ \hline
		Users & โหนดสำหรับเก็บข้อมูลของนักศึกษา \\ \hline
		User\_id & สำหรับเก็บข้อมูลของนักศึกษาแต่ละคน \\ \hline
		depart & สำหรับเก็บภาควิชาของนักศึกษา\\ \hline
		major & สำหรับเก็บสาขาของนักศึกษา\\ \hline
		sid & สำหรับเก็บรหัสประจำตัวนักศึกษา \\ \hline
		name & สำหรับเก็บชื่อของนักศึกษา \\ \hline
		year & สำหรับเก็บชั้นปีของนักศึกษา \\ \hline
		lastChat & สำหรับเก็บเวลาที่สนทนากับเจ้าหน้าที่ล่าสุด \\ \hline
		photoUrl & สำหรับเก็บ url รูปภาพโปรไฟล์ (Profile) \\ \hline
	\end{tabular}
\end{table}

\newpage
\begin{figure}[H]
	\centering
	\includegraphics[width=0.5\columnwidth]
	{Figures/3/DB/nodeQueue}
	\caption{โหนดเก็บข้อมูลการจองคิวของนักศึกษา}
	\label{Fig:DB4}
\end{figure}
\begin{table}[H]
	\centering
	\caption{อธิบายโหนดที่ใช้เก็บข้อมูลการจองคิวของนักศึกษา}
	\label{my-label1}
	\begin{tabular}{|c|p{10cm}|}
		\hline
		\multicolumn{1}{|c|}{\textbf{Key}} & \multicolumn{1}{c|}{\textbf{คำอธิบาย}} \\ \hline
		Queue & โหนดสำหรับเก็บข้อมูลการจองคิวของนักศึกษาทั้งหมด \\ \hline
		q\_id &  สำหรับเก็บข้อมูลของการจองคิวแต่ละครั้งที่เปิดจองคิว \\ \hline
		Date &  สำหรับเก็บวันที่สำหรับส่งเอกสาร\\ \hline
		Time &  สำหรับเก็บรายชื่อของนักศึกษาที่ทำการจองคิวในส่งเอกสารเวลานั้น ๆ\\ \hline
		User\_id & สำหรับเก็บรหัสของนักศึกษา \\ \hline
		title & สำหรับเก็บชื่อหัวเรื่องกำหนดการการจองคิว \\ \hline
		studentPerHr & สำหรับเก็บจำนวนนักศึกษาต่อชั่วโมง \\ \hline
	\end{tabular}
\end{table}

\newpagedr
\begin{figure}[H]
	\centering
	\includegraphics[width=0.4\columnwidth]
	{Figures/3/DB/nodeFaq}
	\caption{โหนดเก็บข้อมูลคำถามที่พบบ่อย}
	\label{Fig:DB4}
\end{figure}
\begin{table}[H]
	\centering
	\caption{อธิบายโหนดที่ใช้เก็บข้อมูลคำถามที่พบบ่อย}
	\label{my-label1}
	\begin{tabular}{|c|p{10cm}|}
		\hline
		\multicolumn{1}{|c|}{\textbf{Key}} & \multicolumn{1}{c|}{\textbf{คำอธิบาย}} \\ \hline
		Queue & โหนดสำหรับเก็บข้อมูลคำถามที่พบบ่อยทั้งหมด \\ \hline
		Faq\_id & สำหรับเก็บข้อมูลคำถามที่พบบ่อยแต่ละรายการ \\ \hline
		title & สำหรับเก็บคำถาม \\ \hline
		description & สำหรับเก็บคำตอบ \\ \hline
	\end{tabular}
\end{table}
