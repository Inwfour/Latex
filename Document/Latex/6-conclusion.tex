\chapter{สรุปและข้อเสนอแนะ}

การดำเนินโครงงานเพื่อพัฒนาแอปพลิเคชันสูงวัยมายเฟรนด์ นี้ พบว่าระบบสามารถทำงานได้ตามที่วิเคราะห์และออกแบบไว้ แต่ก็พบปัญหาและอุปสรรคระหว่างการพัฒนา ในบทนี้ผู้พัฒนาจึงขอสรุปความสามารถของระบบ ชี้แจงปัญหาและอุปสรรค พร้อมเสนอแนวทางในการพัฒนาแอปพลิเคชันสูงวัยมายเฟรนด์ต่อตามลำดับ

\section{สรุปความสามารถของระบบ}
	แอปพลิเคชันสูงวัยมายเฟรนด์ ทำงานได้ดังนี้
		\begin{itemize}
			\item ระบบสามารถทำงานได้ทั้งแอนดรอยด์และไอโอเอส
			\item ผู้ใช้สามารถจัดการข้อมูลผู้ใข้ได้
			\item ผู้สามารถจัดการโพสท์และคอมเมนท์ได้
			\item ผู้ใช้สามารถโต้ตอบกับแชทบอทได้
			\item ผู้ใช้สามารถจัดการครอบครัวได้
			\item ผู้ใช้สามารถจัดการการแจ้งเตือนได้
		\end{itemize}
	
\section{ปัญหาและอุปสรรคในการพัฒนา}
  \begin{enumerate}[label=\arabic*)]

	\item LocalNotification (Plugin) การแจ้งเตือนไม่สามารถแจ้งเตือนได้ถ้าเราปิดแอปพลิเคชัน \\ 
   แนวทางการแก้ไข : ใช้ Background Mode Plugin เพื่อรองรับการปิดแอปพลิเคชันแบบไม่สมบูรณ์

   \item Chatbot ยังไม่สามารถตอบคำถามได้ครอบคลุมทุกคำถาม \\ 
   แนวทางการแก้ไข : ต้องเพิ่มข้อมูลในเรื่องการสนทนาให้กับบอทมาก ๆ
    
  \end{enumerate}

\section{แนวทางการพัฒนาต่อ}
\begin{enumerate}[label=\arabic*)]
	\item การพัฒนาในส่วนแชทบอท จะต้องมีการเทรนข้อมูลที่ครอบคลุมและถูกต้องแม่นยำมากยิ่งขึ้น และสามารถแชทเรื่องสภาพอากาศได้ในอนาคต
\end{enumerate}




